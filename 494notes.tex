\documentclass{amsart}
\usepackage{amsfonts}
\usepackage{amsmath}
\usepackage{amsthm}
\usepackage{amssymb}
\usepackage{mathtools}
\usepackage{enumerate}
\usepackage{graphicx}
\usepackage{dsfont}
\usepackage{bbm}
\usepackage{tikz}
\usepackage{tikz-cd}
\usepackage{xifthen}
\usepackage{cancel}
\usepackage{hyperref}

\setlength{\textwidth}{\paperwidth}
\addtolength{\textwidth}{-2in}
\calclayout

\newcommand{\NN}{\mathbb{N}}
\newcommand{\RR}{\mathbb{R}}
\newcommand{\HH}{\mathbb{H}}
\newcommand{\QQ}{\mathbb{Q}}
\newcommand{\ZZ}{\mathbb{Z}}
\newcommand{\Zn}[1]{\mathbb{Z} / #1 \mathbb{Z}}
\newcommand{\CC}{\mathbb{C}}
\newcommand{\FF}{\mathbb{F}}
\newcommand{\PP}{\mathbb{P}}
\newcommand{\dd}[2][ ]{\frac{\partial #1}{\partial #2}} %partial derivative
\newcommand{\dsq}[3][ ]{\frac{\partial^2 #1}             %Second partial
{\ifthenelse { \equal {#2} {#3} }{\partial #2^2}{\partial #2 \partial #3}}}
\newcommand{\defarr}					    %definition iff arrow
{\overset{\textrm{def}}{\Longleftrightarrow}}
\newcommand{\defeq}						    %definition equality sign
{\overset{\textrm{def}}{=}}
\newcommand{\argeq}[1]						%definition equality sign
{\overset{\textrm{#1}}{=}}

\DeclareMathOperator{\lcm}{lcm}
\DeclareMathOperator{\orb}{orb}
\DeclareMathOperator{\im}{im}
\DeclareMathOperator{\supp}{supp}
\DeclareMathOperator{\stab}{Stab}
\DeclareMathOperator{\sgn}{sign}
\DeclareMathOperator{\spn}{span}
\DeclareMathOperator{\tr}{tr}

\newtheorem{thm}{Theorem}[section]
\newtheorem{lemma}[thm]{Lemma}
\newtheorem*{unnumlemma}{Lemma}
\newtheorem{fact}[thm]{Fact}
\newtheorem{prop}[thm]{Propostion}
\newtheorem*{claim}{Claim}
\newtheorem{cor}{Corollary}

\theoremstyle{definition}
\newtheorem{defn}[thm]{Definition}
\theoremstyle{remark}
\newtheorem*{rmk}{Remark}
\newtheorem*{ex}{Example}
%==============================================================================
% MATH 494 Collaborative Notes
% This is a collaborative notesheet that consists of notes from each day in Math
% 494, transcribed into Tex format for convenience and security from the
% Notebook Thief. General format:
% Contribute by adding a new section, titled "Month Day, Year"
% This is done in the format \section{Date}
% Tex notes using the numbering system used by Tasho himself. I've set up the
% counters on the theorems, definitions, and "facts" to run in tandem with the
% section that the content is in.
% Make sure to properly wrap all proofs, facts, and definitions using the above
% environments.
% Enjoy!
% Current Collaborators:
% 1. Pranav
% 2. Nick
% 3. Andrew
% 4. Ben
% 5. 
%==============================================================================
\begin{document}
\title{Math 494: Honors Algebra II}
\maketitle
\tableofcontents
\section{January 4, 2017} %Pranav
\noindent \textbf{Rings}
\begin{defn} \hspace{0.5cm}
    \begin{enumerate}[a)]
    \item A \textbf{ring} is a tuple $(R, +, \cdot, 0)$ where:
    \begin{itemize}
        \item $R$ is a set
        \item $0 \in R$
        \item $+,\cdot: R \times R \rightarrow R$, $\quad$  $(a,b) \mapsto a + b, a \cdot b$
    \end{itemize}
    subject to:
    \begin{itemize}
        \item $(R, +, 0)$ is an abelian group
        \item $(a \cdot b) \cdot c = a \cdot (b \cdot c)$
        \item $(a + b) \cdot c = a \cdot c + b \cdot c$
        \item $a \cdot (b + c) = a \cdot b + a \cdot c$
    \end{itemize}
    \item A \textbf{ring with unity} is a tuple $(R, +, \cdot, 0, 1)$, where
    $(R,+,\cdot,0)$ is a ring, and $1 \in R$ is subject to $1 \cdot a = a \cdot 1 = a$
    for all $a \in R$.
    \item A ring $(R, +, \cdot, 0)$ is called \textbf{commutative} if $ab = ba$ for all
    $a, b \in R$.
    \item A \textbf{field} is a commutative ring with unity $(R,+,\cdot,0,1)$ such
    that $(R \backslash \{0\}, \cdot, 1)$ is a group.
    \end{enumerate}
\end{defn}
\begin{rmk} \hspace{0.5cm}
    \begin{itemize}
        \item We don't really need to include 0,1 in notation: they are unique
        if they exist
        \item There is a notion of a \textbf{skew field}: ring with unity
        $(R,+,\cdot,0,1)$ such that $(R \backslash \{0\}, \cdot , 1)$ is a group.
        (This drops the commutative condition from the definition of a field).
        \item In French: \textit{corps} is a skew field, and \textit{corps commutatif} is a field.
    \end{itemize}
\end{rmk}
\begin{fact}\label{fact:0prod}
 Let $R$ be a ring. For all $a \in R$, $0 \cdot a = 0$.
\end{fact}
\begin{proof}
    $(0 \cdot a) = (0 + 0) \cdot a = 0 \cdot a + 0 \cdot a \Rightarrow 0 = 0 \cdot a$
\end{proof}
\begin{ex} \hspace{0.5cm}
    \begin{itemize}
        \item $\ZZ$ is a ring, commutative, with unity
        \item $\QQ, \RR, \CC$ are fields
        \item $\HH = \{a + bi + cj + dk \mid a,b,c,d \in \RR \}$ where $i^2 = j^2 = k^2 = ijk = -1$ are
        called the \textbf{Hamiltonian Quaternions} and are a skew-field
        \item $\mathcal{C}_{C}(\RR) = $ functions on $\RR$ with compact
        support \\
        ($\supp(f) = \overline{\{x \in \RR \mid f(x) \neq 0\}}$) is a
        commutative ring without unity
        \item $R = \{\star\}, 0 = 1 = \star$ is the \textbf{zero ring}.
    \end{itemize}
\end{ex}
\begin{fact}
    If $(R,+,\cdot,0,1)$ is a ring with unity and $0 = 1$, then $R$ is the zero ring.
\end{fact}
\begin{proof}
    Take $a \in R$. Then $a = a \cdot 1 = a \cdot 0 = 0$ by Fact \ref{fact:0prod}.
\end{proof}
\noindent \underline{Convention}: Unless otherwise noted, ring will refer to
a commutative ring with 1.
\begin{defn}
    Let $R$ be a ring. Its \textbf{group of units} is
    $$
    R^\times = \{a \in R \mid \exists \, b \in R: ab = 1\}
    $$
\end{defn}
\begin{fact} \hspace{0.5cm}
    \begin{itemize}
        \item For $a \in R^\times$, there is a unique $b \in R$ such that $ab = 1$.
        Write $b = a^{-1}$.
        \item For $a,b \in R^\times$, $a \cdot b \in R^\times$.
    \end{itemize}
\end{fact}
\begin{proof} \hspace{0.5cm}
    \begin{itemize}
        \item Given $b, b^\prime$, we have $b = b \cdot 1 = b(ab^\prime) = (ba)b^\prime = 1 \cdot b^\prime
         = b^\prime$.
         \item $(a \cdot b) \cdot (b^{-1} \cdot a^{-1}) = 1$
    \end{itemize}
\end{proof}
\begin{ex}
    $\RR^\times = \RR \backslash \{0\}$, $\ZZ^\times = \{1, -1\}$
\end{ex}
\begin{defn}
    Let $R, S$ be rings. A \textbf{morphism} $\phi:R \rightarrow S$ is a map of
    sets $\varphi:R \rightarrow S$ satisfying
    \begin{itemize}
        \item $\varphi(a + b) = \varphi(a) + \varphi(b)$
        \item $\varphi(a \cdot b) = \varphi(a) \cdot \varphi(b)$
        \item $\varphi(1) = 1$
    \end{itemize}
\end{defn}
\begin{ex}
    $\varphi:\ZZ \rightarrow \ZZ$ $u \mapsto 0$ is \underline{not} a morphism of
    rings with 1. (it is a morphism of general rings).
\end{ex}
\begin{fact}\label{fact:!morph}
    For any ring $R$ there is a unique morphism $\varphi:\ZZ \rightarrow R$. Given
    $z \in \ZZ$, we write $z_{R}$, or simply $z$ for its image under $\varphi$.
\end{fact}
\begin{ex}
    $5 \in \ZZ$, $5_{\QQ} \in \QQ$ usual number $5$. $5_{\Zn 2} = 1_{\Zn 2}$
\end{ex}
\begin{defn}
    Let $R$ be a ring. A subset $I \subset R$ is called an \textbf{ideal} if
    \begin{itemize}
        \item $I$ is a subgroup of $(R, +, 0)$
        \item $a \cdot f \in I$ for all $a \in R, f \in I$.
    \end{itemize}
\end{defn}
\begin{defn}
    Let $R$ be a ring. A subset $S \subset R$ is called a \textbf{subring} if
    \begin{itemize}
        \item $S$ is a subgroup of $(R, +, 0)$
        \item $a \cdot b \in S$ for all $a, b \in S$.
        \item $1 \in S$.
    \end{itemize}
\end{defn}
\begin{rmk}\hspace{0.5cm}
    \begin{itemize}
        \item The only subset that is both a subring and an ideal is $R$ itself.
        (reason: if $1 \in I$, then $a \cdot 1 \in I$ for all $a \in R$, meaning $I = R$)
        \item $I = \{0\}, I = R$ are always ideals.
        \item In rings without unity, the 2 notions align closer: ideal becomes a special
        case of subring as $1 \in S$ condition is dropped.
    \end{itemize}
\end{rmk}
\begin{ex} \hspace{0.5cm}
    \begin{itemize}
        \item Every subgroup of $(\ZZ, +, 0)$ is an ideal of $\ZZ$.
        \item If $F$ is a field, then $\{0\}, R$ are the only ideals
        \item Let $R = \mathcal{C}_C(\RR), S \in R$ subset.
        $$
        I = \{f \in \mathcal{C}_C(\RR) \mid f \mid_{S} = 0 \}
        $$
        is an ideal
    \end{itemize}
\end{ex}
\begin{defn}
    An ideal $I \in R$ is called \textbf{principal} if $I = \{a \cdot r \mid r \in R\}$
    for some $a \in R$. Then $a$ is called a \textbf{generator}.
\end{defn}
\begin{defn}
    Let $a_1, a_2, \dots a_n \in R$. An \textbf{ideal generated by} $a_1, \dots a_n$ is
    $$
    (a_1, \dots a_n) = \{a_1r_1 + \dots + a_nr_n \mid r_i \in R\}
    $$
\end{defn}
\begin{fact}
    Given ideals $I, J \subset R$ we have
    \begin{itemize}
        \item $I \cap J$ is an ideal
        \item $I + J = \{a + b \mid a \in I, b \in J \}$ is an ideal
        \item $I \cdot J = \left\{\sum\limits_{i = 1}^{n}a_ib_i \mid a_i \in I, b_i \in J \right\}$ is an ideal
    \end{itemize}
\end{fact}
\section{January 6, 2017} %Pranav
\begin{fact}
    Let $\varphi: R \rightarrow S$ be a morphism. Then
    $$
    \ker(\varphi) = \{x \in R \mid \varphi(x) = 0\}
    $$
    is an ideal.
\end{fact}
\begin{proof} (A Pranav Exclusive)
    We first show that the kernel is a subgroup of $(R, +, 0)$. Well, we first show that
    $0 \in \ker(\varphi)$. Well,
    $$
    \varphi(0) = \varphi(0 + 0) = \varphi(0) + \varphi(0)
    $$
    so, we have that $\varphi(0) = 0$ and thus $0 \in \ker(\phi)$.
    Next, we show that inverses are in the kernel as well. \\
    If we have that $\varphi(a) = 0$, then we have
    $$0 = \varphi(0) = \varphi(a + (-a)) = \varphi(a) + \varphi(-a) = \varphi(-a)$$
    Now, we complete this step by proving closure. Assume $a,b \in \ker(\varphi)$. Then,
    $$\phi(a + b) = \phi(a) + \phi(b) = 0 + 0 = 0$$
    Thus, we have that the kernel is a subgroup. Now, we verify the second condition.
    Fix $a \in R$ and $f \in \ker(\varphi)$. We have that
    $$
    \phi(a \cdot f) = \phi(a) \cdot \phi(f) = \phi(a) \cdot 0 = 0
    $$
    Thus, we have that $a \cdot f \in \ker(\varphi)$, meaning that $\ker(\varphi)$
    is an ideal.
\end{proof}
\noindent Question: Is every ideal the kernel of morphism?
\begin{prop}
    Let $R$ be a ring, $I \subset R$ an ideal. Let $R / I$ be the quotient of
    abelian groups and $p: R \rightarrow R / I$ the canonical projection. Then there is a
    unique product map
    $$
    \cdot: R/I \times R/I \rightarrow R/I
    $$
    making $R / I$ into a ring such that $p$ is a morphism.
\end{prop}
\begin{proof}
    For $p$ to be a morphism of rings, we need
    \begin{itemize}
        \item $p(1_R) = 1_{R/I}$
        \item The following diagram to commute
        \[ \begin{tikzcd}[%
        ,every arrow/.append style={maps to}
        ,every label/.append style={font=\normalsize}
        ,row sep=1.5cm
        ,column sep=1.5cm]
        R \times R \arrow{r}{\cdot_R} \arrow[swap]{d}{p \times p} & R \arrow{d}{p} \\%
        R/I \times R/I \arrow{r}{\cdot_{R/I}}& R/I
        \end{tikzcd}
        \]
    \end{itemize}
    Uniqueness of $\cdot_{R/I}$ follows from surjectivity of $p \times p$ (each element
    in $R / I \times R/I$ must go precisely to the result of the composition of $p$ and
    $\cdot_R$) \\
    For existence, define $1_{R/I} = p(1_R)$ and $(a + I) \cdot (b + I) \defeq (a \cdot b) + I$.
    We have to show this is well-defined (i.e it is independent of choice of $a, b$). \\
    Well, choose $a^\prime, b^\prime$ such that $a^\prime  + I = a + I, b^\prime + I = b + I$.
    Thus, $a^\prime = a + i$, $b^\prime = b + j$ for some $i, j \in I$. Then
    $$
    (a^\prime + I)(b^\prime + I) = (a^\prime \cdot b^\prime) + I = ((a + i)\cdot(b + j)) + I =
    (a\cdot b + a \cdot j + b \cdot i + i \cdot j) + I = a\cdot b + I
    $$
    as we note that $a \cdot j, b \cdot i$, and $i \cdot j$ are all in $I$ as $I$
    is an ideal. \\
    We have that all of the ring axioms for $R / I$ are inherited from the ring
    structure on $R$.
\end{proof}
\begin{rmk}
    $\ker(p) = I$
\end{rmk}
\begin{thm}\label{thm:homomorphism} \textbf{(Homomorphism Theorem):}
    Let $\phi: R \rightarrow S$ be a morphism of rings, $I \subset \ker(\varphi)$ be
    an ideal of $R$. There is a unique morphism $\overline{\varphi}: R/I \rightarrow S$
    such that $\overline{\varphi} \circ p = \varphi$ i.e.
    \[
\begin{tikzcd}[,every arrow/.append style={maps to}
,every label/.append style={font=\normalsize},column sep=1.5em]
 & R/I \arrow{dr}{\overline{\varphi}} \\
R \arrow{ur}{p} \arrow{rr}{\varphi} && S
\end{tikzcd}
\]
commutes. Moreover, $\overline{\varphi}$ is injective $\iff$ $\ker(\varphi) = I$
\end{thm}
\begin{proof}
    All statements follow from looking at the abelian group $(R, +, 0)$ and its
    subgroup $I$, except multiplicativity of $\overline{\varphi}$. \\
    (A Pranav Exclusive) Some justification:
    the uniqueness of this morphism follows because the projection map is surjective, meaning that in
    order for the composition to be commutative, we must have that each element in $R / I$ maps exactly
    to where its associated element maps under $\varphi$. Now, the existence. We simply need to check that
    the map $\overline{\varphi}$ that sends $a + I$ to $\varphi(a)$ is well defined and is a morphism. We note that
    the additive morphism properties are inherited from the fact that $\varphi$ is a morphism itself. So, we check
    the well-definedness of $\overline{\varphi}$. Pick 2 representatives of $a + I$, call them $a + I$ and
    $a^\prime + I$. We have that $a^\prime = a + i$ for $i \in I$. Then, we have that
    $$
    \overline{\varphi}(a^\prime + I) = \overline{\varphi}(a + i + I) = \overline{\varphi}(a + I) + \overline{\varphi}(i + I) =
    \overline{\varphi}(a + I) + \overline{\varphi}(I) = \overline{\varphi}(a + I) + 0
    $$
    as we have that $\varphi(i) = 0$ for all $i \in I$ (since $I \subset \ker(\varphi)$). We finally verify the injective
    biconditional. Assume $\overline{\varphi}$ is injective. We already have that $I \subset \ker(\varphi)$.
    Now, since $\overline{\varphi}$ is injective, its kernel is trivial, and is thus the identity of $R / I$, namely $I$ itself.
    For any $g \in \ker(\varphi)$ we note that $g + I$ must belong to the kernel of $\overline{\varphi}$, meaning that
    $g + I = I$ and thus $g \in I$. This gives us double containment and thus equality. \\
    Now, assume that $\ker(\varphi) = I$. We consider $\ker(\overline{\varphi})$. This is exactly the collection
    $\{a + I \mid a \in \ker(\varphi) \}$. Thus, this is $\{a + I \mid a \in I\}$ and thus we have that
    $\ker(\overline{\varphi}) = I$. Since the kernel of $\overline{\varphi}$ is trivial, we have that
    $\overline{\varphi}$ is injective. \\
    Checking Multiplicativity: Let $A,B \in R/I$. Choose $a,b \in R$ such that $p(a) = A, p(b) = B$. Then
    $$
    \overline{\varphi}(A \cdot B) = \overline{\varphi}(p(a)\cdot p(b)) =
    \overline{\varphi}(p(ab)) = \varphi(ab) = \varphi(a)\varphi(b) =
    \overline{\varphi}(p(a))\overline{\varphi}(p(b)) = \overline{\varphi}(A)\overline{\varphi}(B)
    $$
\end{proof}
\begin{defn}
    Let $R$ be a ring.
    \begin{itemize}
        \item Let $a,b \in R$. We say that \textbf{$a$ divides $b$} (denoted $a \mid b$)
        if there is $c \in R$ such that $ac = b$.
        \item We say $0 \neq a \in R$ is a \textbf{zero divisor} if there is
        $0 \neq b \in R$ such that $ab = 0$.
        \item We call $R$ a \textbf{domain} (or \textbf{integral domain}) if it has
        no zero divisors.
    \end{itemize}
\end{defn}
\begin{fact}
    $a \mid b \iff (b) \subset (a) \iff b \in (a)$
\end{fact}
\begin{proof} (A Pranav Exclusive)
    We first show the first forward implication. Assume that $a \mid b$. Then, there is
    $c \in R$ such that $ac = b$. Now, fix $g \in (b)$. It is of the form $br$ for some
    $r \in R$. Thus, we have that $g = (ac)r = a(cr)$. Since $cr \in R$, we have that $g \in (a)$. \\
    Next, we show the second forward implication. Assume that $(b) \subset (a)$. Well,
    $b \in (b) \subset (a)$. \\
    Finally, we show that $b \in (a)$ implies the original condition. Well, if $b \in (a)$, then
    $b = ar$ for $r \in R$. This is exactly what it means for $a \mid b$! Thus, we have
    shown equality of the above statements.
\end{proof}
\begin{fact} (Cancellation Law) If $a \neq 0 \in R$ is not a zero divisor, then
    for $x,y \in R$
    $$
    ax = ay \Rightarrow x = y
    $$
\end{fact}
\begin{proof}
    $ax = ay \iff a(x - y) = 0$. $a \neq 0$ implies that $x - y = 0$ as $a$ is not
    a zero divisor.
\end{proof}
\begin{defn}
    An ideal $I \subsetneq R$ is called
    \begin{itemize}
        \item \textbf{prime} if $a \cdot b \in I$ implies $a \in I$ or $b \in I$ for all $a,b \in R$.
        \item \textbf{maximal} if $I$ and $R$ are the only ideals containing $I$.
    \end{itemize}
\end{defn}
\begin{ex}
    In $R = \ZZ$, the ideals are of the form $n\ZZ$. $n\ZZ$ is prime $\iff$ $n$ is prime or $n = 0$.
\end{ex}
\begin{proof}
    (A Pranav Exlusive). We start with the forward direction. We proceed by contrapositive.
    Assume that $n \neq 0$ and that $n$ is not prime. Then, $n$ is composite (we exclude $n = 1$ as
    we must have a properly contained ideal by definition). Thus, we have that $n = ab$ for some $1 < a,b < n$.
    Note that we have $ab = n \in n\ZZ$, but we have that both $a$ and $b$ are less than $n$, and thus there is no $z \in \ZZ$
    such that $nz = a$ or $nz = b$. This means that $n\ZZ$ is not prime, as we have found $a,b$ such that
    $ab \in n\ZZ$ but neither $a$ nor $b$ are in $n\ZZ$. \\
    Now, the reverse direction. First, we show the condition for $n$ prime. Assume that we have $a,b \in \ZZ$ such that $ab \in n\ZZ$.
    This means that we have $ab = nq$ for some $q \in \ZZ$. In particular, this means that $n$ divides
    the product $ab$. However, we note that as $n$ is prime, we have that $n$ must divide $a$
    or $b$ by Euclid's lemma. Thus, we have that either $a = nr$ or $b = nr$ (or both), which implies
    that $a \in n\ZZ$ or $b \in n\ZZ$. Next, for $n = 0$. Well, if $ab \in 0\ZZ$, then $ab = 0$. This in $\ZZ$ implies
    that either $a$ or $b$  is $0$ and is also in $n\ZZ$. This completes the reverse direction.
\end{proof}
\begin{thm}\label{thm:ideals}
    Let $R$ be a ring.
    \begin{enumerate}[i)]
        \item $R$ is a domain $\iff \{0\}$ is prime.
        \item $R / I$ is a domain $\iff I \subset R$ is a prime ideal.
        \item Let $\varphi: R \rightarrow S$ be a morphism, $S$ a domain. Then
        $\ker(\varphi)$ is prime. The converse is true if $\varphi$ is surjective.
        \item $R$ is a field $\iff \{0\}$ is maximal.
        \item $R / I$ is a field $\iff I \subset R$ is a maximal ideal.
        \item Every field is a domain.
        \item Every maximal ideal is prime.
    \end{enumerate}
\end{thm}
\begin{proof}
    We first claim that iii) implies ii) which in turn implies i). First, for iii) implies
    ii), we note that letting $S$ be $R / I$ (which means $\varphi$ is the projection
    map $p$ (which is definitely surjective)) gives us ii). (We have that $\ker(p) = I$).\\
    ii) implies i) simply by letting $I$ be the zero ideal. \\
    Now, we prove statement iii). \\
    Let $a, b \in R$ such that $a \cdot b \in \ker(\varphi)$. Then $0 = \varphi(a \cdot b) = \varphi(a)\varphi(b)$.
    Since we have that $S$ is a domain, then we have no zero divisors, meaning that either
    $\varphi(a) = 0$ or $\varphi(b) = 0$. This in turn implies that either $a \in \ker(\varphi)$ or $b \in \ker(\varphi)$,
    so we have show that $\ker(\varphi)$ is a prime ideal. Now, the converse assuming surjectivity.
    We want to show that $S$ has no zero divisors. Well, fix $A,B \in S$ such that $A \cdot B = 0$.
    Since $\varphi$ is surjective, we have $a, b \in R$ such that $\varphi(a) = A$ and $\varphi(b) = B$.
    Then, we have $0 = \varphi(a)\varphi(b) = \varphi(ab)$, meaning that $ab$ is in $\ker(\varphi)$.
    Because we assume that $\ker(\varphi)$ is prime, this in turn implies that either
    $a$ or $b$ is in $\ker(\varphi)$ meaning that either $\varphi(a) = 0$ or $\varphi(b) = 0$.
    This means that either $A$ or $B$ is 0, and thus $S$ is a domain, as desired. \\
    Next, note that v) implies iv). This comes from letting $I$ be the zero ideal. \\
    The proof of v) comes from the bijection
    $$
    \{\textrm{ideals in $R$ containing $I$}\} \leftrightarrow \{\textrm{ideals in $R/I$}\}
    $$
    This is a homework problem. \\
    Now, we show vi). Assume that $F$ is a field. Pick $a,b \in F$ such that $a \cdot b = 0$
    with $a \neq 0$. We will show that $b$ must be 0, thereby showing that $F$ is a domain.
    Well, since $a \neq 0$, and $F \backslash \{0\}$ is a group, we have that $a^{-1}$ exists.
    Thus, we have that $ab = 0$ implies that $a^{-1}ab = 0$ and thus $b = 0$, as desired. \\
    vii) follows from the facts vi), v) and ii). We have that
    \begin{center}
        $I$ is a maximal ideal $\overset{\textrm{v}}{\iff}$ $R/I$ is a field $\overset{\textrm{vi}}{\Rightarrow}$ $R / I$ is a domain
        $\overset{\textrm{ii}}{\iff}$ I is prime.
    \end{center}
\end{proof}
\section{January 9, 2017}
\begin{defn}
    Let $R$ b a domain. The canonical morphism $\ZZ \rightarrow R$ of Fact \ref{fact:!morph} has
    a prime ideal as its kernel. By Thm \ref{thm:ideals}, this is of the form
    $p\ZZ$ with $p$ prime of $p = 0$. We call $p$ the \textbf{characteristic} of $R$.
\end{defn}
\begin{ex}
\[
\begin{tabular}{ll}
    char$(\ZZ) = 0$ & char$(\Zn 3) = 3$ \\
    char$(\QQ) = 0$ & char$(\Zn 6)$ doesn't exist! $\Zn 6 $ is not a domain.
\end{tabular}
\]
\end{ex}
\begin{unnumlemma}
\textbf{(Zorn's Lemma)} (from Artin). An \textbf{inductive} (every totally ordered
subset has an upper bound) partially ordered set $S$ has at least one maximal element.
\end{unnumlemma}
\begin{thm}
    Let $R$ be a ring. Every proper ideal is contained in a max ideal.
\end{thm}
\begin{proof}
    Let $I \subset R$ be a proper ideal. Let $\mathcal{M}$ be the set of all proper
    ideals of $R$ that contain $I$, with partial order given by inclusion. \\
    Let $\mathcal{C} \subset \mathcal{M}$ be a totally ordered subset.
    \begin{claim}
        $J_0 = \left(\bigcup\limits_{J \in \mathcal{C}}{J}\right) \in \mathcal{M}$
    \end{claim}
    \begin{proof} (of claim). We want to show that $J_0$ is a proper ideal containing
        $I$. First, we show it is an ideal by showing closure of the subgroup and the
        ideal multiplicative closure. Let $f_1, f_2 \in J_0$ and $a \in R$. Now, this means there
        is $J_1, J_2 \in \mathcal{C}$ such that $f_1 \in J_1$ and $f_2 \in J_2$. However,
        since $\mathcal{C}$ is totally ordered, we have that the larger of $J_1$ and $J_2$
        contains both $f_1$ and $f_2$, meaning that we have the existence of $J \in \mathcal{C}$
        such that $f_1,f_2 \in J$. Since $J$ is an ideal, we have that $f_1 + f_2 \in J$ and that
        $a \cdot f_1 \in J$. This thus implies that since $J \in \mathcal{C}$, we have that
        $a \cdot f_1$ and $f_1 + f_2$ are both in $J_0$. Thus $J_0$ is an ideal. Since $I \in \mathcal{C}$,
        we also have that $I \subset J_0$. Finally, $J_0$ is not $R$, because otherwise $1 \in J_0$,
        which would mean that $1 \in J$ for some $J \in \mathcal{C}$. This would then imply that
        that $J = R$, which is not possible as $J$ itself is a proper ideal. Thus, we have that $J_0 \in \mathcal{M}$.
    \end{proof}
    Thus, for every totally ordered subset of $\mathcal{M}$, we have the existence of an
    upper bound (namely $J_0$). This gives us, by Zorn's Lemma, that $\mathcal{M}$ has
    a maximal element. This maximal element is exactly what we wished to show existed.
\end{proof}
\begin{defn}
Let $R,S$ be rings. Their product is the set $R \times S$ with component-wise
operations
\begin{itemize}
    \item $(r,s) + (r^\prime, s^\prime) = (r + r^\prime, s + s^\prime)$
    \item $(r,s) \cdot  (r^\prime, s^\prime) = (r \cdot r^\prime, s \cdot s^\prime)$
    \item $1_{R \times S} = (1_R, 1_S), 0_{R \times S} = (0_R, 0_S)$
\end{itemize}
\end{defn}
\begin{rmk}
    Given morphisms $\varphi_1:R \rightarrow S_1, \varphi_2:R \rightarrow S_2$, we
    get a unique morphism $\varphi_{1} \times \varphi_2: R \rightarrow S_1 \times S_2$.
\end{rmk}
\begin{rmk}
    Given $I,J \subset R$ ideals we have
    $$
    I \cdot J \subset I \cap J \subset I, J \subset I + J
    $$
\end{rmk}
\begin{defn}
    Two ideals $I,J \subset R$ are \textbf{coprime} if $I + J = R$.
\end{defn}
\begin{thm}
    \textbf{(Chinese Remainder Theorem)} Let $R$ be a ring, $I_1, \dots I_n \subset R$
    be pairwise coprime ideals. Then the natural morphism
    $$
    p: R \rightarrow R / I_1 \times R / I_2 \times \dots \times R / I_n
    $$
    factors through the quotient $R / (I_1 \cap I_2 \cap \dots \cap I_n)$ and induces
    an isomorphism of rings
    $$
    \overline{p}: R / (I_1 \cap I_2 \cap \dots \cap I_n) \rightarrow R / I_1 \times R / I_2 \times \dots \times R / I_n
    $$
    Moreover, $I_1 \cdot I_2 \cdots I_n = I_1 \cap I_2 \cap \dots I_n$
\end{thm}
\begin{proof}
    As $p$ is the natural morphism to a product of rings, we let $p = p_1 \times p_2 \dots \times p_n$,
    where each $p_i$ is the projection morphism from $R$ to $R / I_i$. Now, we can say that
    $\ker(p) = \{r \in R \mid 0 = p_1(r), 0 = p_2(r), \dots 0 = p_n(r) \}$. Well, since each $p_i$ by definition has
    kernel exactly $I_i$, this is the same as saying that
    $\ker(p) = \{r \in R \mid r \in I_1 \cap I_2 \cap \dots \cap I_n \}$. \\
    By the homomorphism theorem (\ref{thm:homomorphism}), we have that $p$ factors through
    $R / I_1 \cap I_2 \cap \dots I_n$ and also induces an injective ring morphism
    $\overline{p}: R/ I_1 \cap \dots \cap I_n \rightarrow R/I_1 \times \dots R/I_n$.
    \begin{claim}
        $\overline{p}$ is also surjective, and hence a isomorphism.
    \end{claim}
    \begin{proof} (of claim)
        We note that since each of the ideals are coprime, we have that $I_1 + I_k = R$.
        Now, we also note that $R \cdot R = R$. Thus, we can express
        $$
        R = (I_1  + I_2) \cdot (I_1 + I_3) \cdots (I_1 + I_n)
        $$
        expanding the product, we note that by the earlier remark that any term
        containing an $I_1$ (which is almost all of them) will be contained in
        $I_1$. The only term that is outside arises from selecting the second term
        in every single term of the product, so we can write that the above expression
        is
        $$
        \subset I_1 + (I_2 \cdot I_3 \cdots I_n)
        $$
        Now, since $R \subset I_1 + (I_2 \cdot I_3 \cdots I_n)$, we can take
        $v_1 \in I_1$ and $u_1 \in I_2 \cdots I_n$ such that $u_1 + v_1 = 1$.
        Now, since $u_1 \in I_2 \cdots I_n$, $u_1 \in I_j$ for $j \neq 1$. Thus,
        we can say that $u_1$ maps to $0_{R/I_j}$ under the projection map, as it is
        in the kernel. \\
        Similarly, since $u_1 = 1 - v_1$, with $v_1 \in I_1$, we have that $u_1 \in 1 + I$,
        meaning that $u_1$ maps to $1_{R/I_1}$ under the projection map. \\
        So, we have (abusing notation) that $u_1 = 1$ in $R/I_1$ and $u_1 = 0$ in $R/I_j$ for $j \neq 1$
        (really, as we showed above, it belongs to the associated cosets). \\
        Now, we can repeat this construction with any $I_i$ instead of $I_1$.
        Thus, we get for each such construction a $v_i \in I_i$ and $u_i \in I_1 \cdot I_2 \cdots \widehat{I_i} \cdots I_n$
        With this construction, we now have the existence of the $u_i$ that belong
        to the $1$ coset in exactly $R/I_i$ and the $0$ coset in all remaining $R/I_j$.
        With this, we can prove surjectivity.
        Fix any $(x_1, \dots x_n) \in R/I_1 \times \dots R/I_n$. We have that there exists
        an associated $r_1, \dots r_n \in R$ such that $p_1(r_1) = x_1, \dots p_n(r_n) = x_n$.
        Now, if we consider the element $r \in R$ that equals $u_1r_1 + u_2r_2 \dots u_nr_n$,
        note that $p(r) = (p_1(r), p_2(r) \dots p_n(r))$. However, since the $u_i$ map to $1$ under
        $p_i$ and to $0$ otherwise, this maps precisely to $(x_1, \dots x_n)$. Thus, we have
 that $p(r)$ maps to the desired element in the product, meaning that the
        associated coset will map to the desired element under $\overline{p}$. This proves
        surjectivity.
    \end{proof}
    Thus, we have that $\overline{p}$ is an isomorphism. Now, we show the second part of
    part of the statement. \\
    Well, we know by definition that $I_1 \cdot I_2 \cdots I_n \subset I_1 \cap \dots \cap I_n$. So,
    we simply need to show the other containment, which we do by induction on $n$. \\
    $n = 1$: $I_1 \subset I_1$. \\
    $n = 2$: Take $u_1 \in I_1$ and $u_2 \in I_2$ such that $1 = u_1 + u_2$ (this exists as $I_1 + I_2 = R$.)
    Now, for any $u \in I_1 \cap I_2$, we have
    $$
    u = u \cdot 1 = u \cdot (u_1 + u_2) = u \cdot u_1 + u \cdot u_2
    $$
    Since $u \in I_1$ and $u \in I_2$, we have $u \cdot u_1 \in I_2 \cdot I_1$ and
    $u \cdot u_2 \in I_1 \cdot I_2$. Thus, we have the sum in $I_1 \cdot I_2$. This gives us
    $I_1 \cap I_2 \in I_1 \cdot I_2$. \\
    Now, for general $n$. By the inductive hypothesis, we have that
    $I_1 \cap I_2 \dots I_n \subset (I_1 \cdots I_{n-1}) \cap I_n$. From the claim above,
    we know that $R = (I_1 \cdot I_{n-1}) + I_n$. This implies thus that the ideals
    $(I_1 \cdots I_{n-1})$ and $I_n$ are coprime. Thus, applying the $n = 2$ case on these
    2 ideals, we have that $(I_1 \cdots I_{n-1}) \cap I_n \subset (I_1 \cdots I_{n-1}) \cdot I_n$,
    thereby proving the desired result. 
\end{proof}

\section{January 11, 2017}
\rmk .

\begin{itemize}
        \item Any field is a domain.
        \item Any subring of a domain is a domain.
        \item Any subring of a field is a domain.
    \end{itemize}

Is the opposite true?
\begin{thm}
Let $R$ be a domain.

1) There exists a pair $(i,K)$ with $K$ a field, $i:R\rightarrow K$ an injective morphism such that if $(j,L)$ is another such pair, there exists a morphism $l:K\rightarrow L$ such that $j=l\circ i,$ which is to say that the following diagram commutes.

\begin{tikzcd}[,every arrow/.append style={maps to}
,every label/.append style={font=\normalsize},column sep=1.5em]
 & R \arrow{dr}{j}\arrow{dl}{i} \\
K \arrow{rr}{l} && L
\end{tikzcd}

2) If $(i',K')$ is another pair as in 1) there exists a unique isomorphism $\phi:K\rightarrow K'$ such that 

\begin{tikzcd}[,every arrow/.append style={maps to}
,every label/.append style={font=\normalsize},column sep=1.5em]
 & R \arrow{dr}{i'}\arrow{dl}{i} \\
K \arrow{rr}{\phi} && K'
\end{tikzcd} commutes.

\end{thm}

\rmk.

\begin{itemize}
	\item $(i,K)$ is an example of a ``universal object"
	\item $(j,L)$ is called a test object
	\item $K$ is produced from $R,$ just like the rationals are produced from the integers.
\end{itemize}

\begin{proof}

2) Given two universal objects $(i,K),(i',K'),$ apply 1) with $(i,K)$ as the universal object, and $(i',K')$ as a test object to get $l:K\rightarrow K'$. Do it the other way to get $l':K'\longrightarrow K.$

\begin{claim}

$l\circ l'=\text{id}_{K'},l'\circ l=\text{id}_K$

\begin{proof}

Note that both $l'\circ l$ and $\text{id}_k$ make the diagrams

\begin{tikzcd}[,every arrow/.append style={maps to}
,every label/.append style={font=\normalsize},column sep=1.5em]
 & R \arrow{dr}{i}\arrow{dl}{i} \\
K \arrow{rr}{l'\circ l} && L
\end{tikzcd}

\begin{tikzcd}[,every arrow/.append style={maps to}
,every label/.append style={font=\normalsize},column sep=1.5em]
 & R \arrow{dr}{i}\arrow{dl}{i} \\
K \arrow{rr}{\text{id}_K} && L
\end{tikzcd} commute.

When $(i,k)$ is both a universal object and a test object, we get $l\circ l=\text{id}_K.$ \end{proof}
\end{claim}


1) Consider the set $P=R\times R\setminus\{0\}$. Introduce the relation $(n,d)\sim (n',d')\iff nd'=n'd$.

\begin{claim}

$\sim$ is an equivalence relation. 

\begin{proof} 

Reflexive: $(n,d)\sim(n,d)\iff nd=nd.$

Symmetric $(n,d)\sim(n',d')\iff nd'=n'd\iff n'd=nd'\iff (n',d')\sim (n,d).$

Transitive: Assume $(n_1,d_1)\sim(n_2,d_2)\sim(n_3,d_3)$. We want $(n_1,d_1)\sim(n_3,d_3).$
We have $n_1d_2=n_2d_1,n_2d_3=n_3d_2$ and want $n_1d_3=n_3d_1.$
We see that $n_1d_3 n_2d_2=n_1d_3n_2d_3=n_2d_1n_3d_2=n_3d_1n_2d_2.$
Since $R$ is a domain, $n_2d_2$ is not a zero-divisor. If $n_2d_2\ne 0$, then by Fact 6, Jan 6, we get $n_1d_3=n_3d_1$. If $n_2d_2=0$, then ($d_2\ne 0$ and not a 0-divisor) $n_2=0$. For the same reason, $n_1=n_3=0.$ Again, $n_1d_3=n_3d_1$. Either way, we are done.
\end{proof}
\end{claim}

Put $K=P/\sim.$ Write $[n,d]$ for the image of $(n,d)\in P$ in $K$. Define

$$[n,d]\cdot[n',d']=[nn',dd']$$

$$[n,d]+[n',d']=[nd'+n'd,dd']$$

$$0=[0,1],1=[1,1]$$

$$i:R\rightarrow K, i(r)=[r,1].$$

We leave as homework the verifications that $+,\cdot$ are well defined, that $K$ is a field, and that $i$ is a morphism. Injectivity is obvious. Given $(j,L)$, define $l:K\longrightarrow L$ by 

$$l([n,d])=l(i(n))\cdot l(i(d)^{-1})=j(n)j(d)^{-1}.$$

Homework: $l$ is well defined and a ring morphism.

\end{proof}

\begin{defn}

A pair $(i,K)$ is called a (the) field of fractions (fraction field) of $R$.

\end{defn}

\begin{defn}

1) Let $R$ be a ring. A polynomial in $T$ over $R$ is a formal expression $a_nT^n+a_{n-1}T^{n-1}+\ldots+a_0,a_i\in R.$

2) Given $P(T)=a_nT^n+\ldots+a_0,Q(T)=b_nT^n+\ldots+b_0$ define

$$(P+Q)(T)=(a_n+b_n)T^n+\ldots+(a_0+b_0)$$

$$(P\cdot Q)(T)=(c_mT^m+c_{m-1}T^{m-1}+\ldots+c_0$$ where 

$$c_k=\sum_{i+j=k}a_i\cdot b_j.$$

3) Given $r\in R$ we have the constant polynomial
$r:(a_nT^n+\ldots+a_0,a_0=r,a_i=0\text{ for } i>0).$ In particular, we have $0,1$ as constant polynomials.

4) Let $R[T]$ be the set of al polynomial in $T$ over $R$. 

\begin{fact} $(R[T],+,\cdot,0,1)$ is a ring. Moreover $R\rightarrow R[T]$, $r\rightarrow$ constant polynomial $r$ is an injective morphism. The proof is left as an exercise to the reader.
\end{fact}

\end{defn}

\begin{defn}

Given $0\ne P\in R[T],$ define deg$(P)=\min\{n|a_m=0\forall m>n\}$, deg$(0)=-\infty$.
\end{defn}
\begin{fact}
1) deg$(P+Q)\leq\max(\text{deg}(P),\text{deg}(Q))$ with equality if deg$(P)\ne$ deg$(Q)$.
2) deg$(P\cdot Q)\leq$ deg$(P)+$deg$(Q)$ with equality if the leading coefficient of $P$ (or $Q$) is not a 0 divisor.
3) In particular, if $R$ is a domain, so is $R[T].$ The proof is left as an exercise to the reader.
\end{fact}

\section{January 13, 2017} % Andrew

\begin{rmk}
Any $P \in R[T]$ gives a function $R \rightarrow R$ by $r \mapsto P(r) = a_nr^n + ... + a_0$.  However, $P$ is not necessarily determined by this function.  For example, let $R = \ZZ / p\ZZ$ where $p$ is a prime and $P(T) = T^p - T$.  Since $x^p = x$ for all $x \in R$, $P$ and $0$ give the same function.  However, $P \neq 0$.
\end{rmk}

\begin{ex}

\item $P = T^2 + 3T - 2$, $Q = -T^2 + 3T - 7$ gives $P+Q = 6T - 9$ ($\text{deg}(P+Q) < \max(\text{deg}(P),\text{deg}(Q))$)

\item $R = \ZZ/4\ZZ$, $P = 2T^2+1$, $Q = 2T^3+3T$ gives $PQ = 3T$ ($\text{deg}(PQ) < \text{deg}(P)+\text{deg}(Q)$)

\end{ex}

\begin{fact}
Let $\phi : R \rightarrow S$ be a morphism and let $s \in S$.  There exists a unique morphism $\phi_s : R[T] \rightarrow S$ such that $\phi_s(r) = \phi(r)$ for all $r \in R$ and $\phi_s(T) = s$.
\end{fact}

\begin{proof}
If $\phi_s$ is any such morphism then $\phi_s(a_nT^n + ... + a_0)$ must equal $\phi(a_n)s^n + ... + \phi(a_0)$.  This proves uniqueness and existence (upon checking that this is a morphism).
\end{proof}

\begin{ex} \hspace{0.5cm}

\begin{itemize}

\item If $\phi = id : R \rightarrow R$ then we get evaluation morphism $R[T] \rightarrow R$ given by $P \mapsto P(s)$.

\item Let $I \subseteq R$ be an ideal and let $\phi : R \rightarrow R/I \xhookrightarrow{} R/I[T]$ and let $s = T$.  We get ``reduction mod $I$" morphism $R[T]\rightarrow R/I[T]$.

\end{itemize}

\end{ex}

\begin{rmk} (def. 1.5)
$a \in R$ is \textbf{nilpotent} if $a^n = 0$ for some $n \in \NN$
\end{rmk}

\begin{prop}
Let $P = a_nT^n + ... + a_0 \in R[T]$.  We have $P \in R[T]^\times$ iff $a_0 \in R^\times$ and $a_1,...,a_n$ are nilpotent.
\end{prop}

\begin{proof} \hspace{0.5cm}

Assume that $R$ is a domain.  We have that $P$ is a unit iff there exists $Q \in R[T]$ such that $PQ = 1$.  By 1-11 Fact 6, $0 = deg(1) = deg(PQ) = deg(P) + deg(Q)$ ($R$ is a domain so the leading coefficient of $P$ (alternatively $Q$) is not a zero divisor).  Thus $deg(P),deg(Q) = 0$.  Thus $a_1,...,a_n = 0$ are nilpotent and $a_0 \in R^\times$.

Let $R$ be a general ring.  Let $\mathcal{P} \subseteq R$ be a prime ideal.  Since $P$ is a unit in $R[T]$, the image of $P$ in $R / \mathcal{P}[T]$ is a unit.  Since $R/\mathcal{P}$ is a domain by 1-6 thm. 8, by the above argument $a_1,...,a_n = 0_{R/\mathcal{P}}$ and thus $a_1,...,a_n \in \mathcal{P}$.  Since this holds for all $\mathcal{P}$, by HW we have that $a_1,...,a_n$ are nilpotent.

\end{proof}

\begin{lemma}
Let $P \in R[T]$ and $r \in R$.  We have $P(r) = 0$ iff $(T-r) \mid P$.
\end{lemma}

\begin{proof}

The backward direction is clear.  Apply fact 1 with $S = R[T]$, $\phi : R \xhookrightarrow{} R[T]$, $s = T+r$ to get a morphism $R[T] \rightarrow R[T]$.  This is an isomorphism with inverse given by the same construction with $s = T-r$.  Under this isomorphism, $P \mapsto Q$ with $Q(0) = 0$.  Thus $Q(T) = b_nT^n + ... + b_1T$ so $T \mid Q$.  Taking the preimage under the above isomorphism, we have $(T-r) \mid P$.

Gitlin's thoughts:  The constructed isomorphism can be thought of as the map $R[T] \rightarrow R[T]$ which ``replaces every $T$ with $T+r$."  Thus $Q(x) = P(x+r)$ for all $x \in R$.  In particular, $Q(0) = P(r)$ which is $0$.  The inverse map is the map $R[T] \rightarrow R[T]$ which ``replaces every $T$ with $T-r$."  In particular, the preimage of $Q(T) = b_nT^n + ... + b_1T$ is $b_n(T-r)^n + ... + b_1(T-r)$.

\end{proof}

\begin{prop}
Let $P,D \in R[T]$.  Assume that $D \neq 0$ and that the leading coefficient of $D$ is a unit.  There exist unique $Q,Z \in R[T]$ with $deg(Z) < deg(D)$ such that $P = QD+Z$.
\end{prop}

\begin{proof}\hspace{0.5cm}

Choose $Q$ so that $deg(Z)$ is minimal where $Z = P-QD$.  We claim $deg(Z) < deg(D)$.  Suppose not.  Let $D = d_nT^n + ... + d_0$ and $Z = z_mT^m + ... + z_0$ with $m \geq n$.  Note that $P-(Q+z_md_n^{-1}T^{m-n})D = Z - (z_md_n^{-1}T^{m-n})D$ has degree less than $deg(Z)$, contradicting the minimality of $Z$.  This shows existence.

Gitlin's thoughts:  The set of ``candidates" is the set of elements of $R[T]$ that have the form $P-*D$ where $*$ varies over $R[T]$.  Clearly $P-(Q+z_md_n^{-1}T^{m-n})D$ is a candidate.  Furthermore, the leading term $z_mT^m$ of $Z$ cancels with the leading term $z_md_n^{-1}T^{m-n} \cdot d_nT^n = z_mT^m$ of $(z_md_n^{-1}T^{m-n})D$ in the subtraction $Z - (z_md_n^{-1}T^{m-n})D$ so the degree of $Z$ is at least one more than the degree of $Z - (z_md_n^{-1}T^{m-n})D$.

Let $Q',Z'$ be another such pair.  We have $QD+Z = P = Q'D+Z'$ so $(Q-Q')D = Z'-Z$.  Thus (1-11 fact 6) $deg(D) > \max(deg(Z'),deg(Z)) \geq deg(Z'-Z) = deg((Q-Q')D) = deg(Q-Q') + deg(D)$ (the leading coefficient of $D$ is a unit and thus not a divisor of zero).  This means $deg(Q-Q') = - \infty$ so $Q-Q' = 0$ so $Q= Q'$.  Thus $Z = P-QD = P-Q'D = Z'$.  This shows uniqueness.

Gitlin's thoughts:  My uniqueness proof likely differs from the one given in class.  Sorry Tasho.  I couldn't follow your inequalities.

\end{proof}

\section{Jan. 18, 2017}
\begin{lemma}
(This is Lemma 3 from Jan. 13) Given $P \in R[T], r \in R$, then $P(r) = 0 \Leftrightarrow (T - r) \mid P$.
\end{lemma}

\begin{prop}
(This is Proposition 4 from Jan. 13) Given $P, D \in R[T]$ such that $D$ has a unit as a leading coefficient, there exists unique $Q, Z \in R[T]$ with $deg(Z) < deg(D)$ such that $P = DQ + Z$.
\end{prop}

\begin{proof}
This proof rehashes the one given in class by Tasho to rectify any mistakes or lack of clarity in the one given on Jan. 13 (no offense Gitlin). Following the proof given on Jan. 13 (before the proof of uniqueness), we have the following string of inequalities:
\[deg(D) > deg(Z) \geq deg(Z - Z') = deg(D) + deg(Q - Q').\]
This implies that $deg(Q - Q') = -\infty = deg(Z - Z')$, so $Q - Q' = Z - Z' = 0$, i.e. $Q = Q'$ and $Z = Z'$, completing the proof. 
\end{proof}

The following is a proof Tasho gave of Lemma 3 from Jan. 13 which utilizes Proposition 4 from Jan. 13.
\begin{proof}
As per Proposition 4, write $P(T) = (T - r) \cdot Q(T) + Z(T)$ with $deg(Z) < deg(T - r) = 1 \Rightarrow Z \in R$. We have then $P(r) = (r - r) \cdot Q(r) + Z(r) = Z$, but since $P(r) = 0$ we have now $Z = 0 \Leftrightarrow (T - r) \mid P$. 
\end{proof}

\begin{defn}
Let $P \in R[T], r \in R$. Define \textbf{the multiplicity of $r$ in $P$} by $\max\{n \in \NN \colon (T - r)^n \mid P\}$.
\end{defn}

\begin{cor}
Let $R$ be a domain. Then $0 \neq P \in R[T]$ has at most $deg(P)$-many zeroes, counted with multiplicity.
\end{cor}
\begin{proof}
We induct on $deg(P)$. As a base case take $deg(P) = 0$: $P \in R$ has no zeroes, as required. For an inductive step assume that the statement holds for all polynomials of degree less than $deg(P)$. If $P$ has no zeroes, we are done. Otherwise let $r \in R$ be a zero of $P$. By Lemma 3 from Jan. 13, $P = (T - r) \cdot Q(T)$ for some $Q \in R[T]$ with $deg(Q) < deg(P)$. By inductive hypothesis $Q$ has at most $deg(Q) = deg(P) - 1$ zeroes. Since $R$ is a domain, a zero of $P$ is either $r$ or a zero of $Q$, so $P$ has at most $deg(Q) + 1 = deg(P)$ zeroes, as required. This completes the inductive step, and thus the proof. 
\end{proof}
The following example answers the question ``does this fail when $R$ is not a domain?":
\begin{ex}
If $R = \ZZ / 6\ZZ$ and we define $P(T) = 3T$, then $\#\{ \text{zeroes of } P \} = \#\{0, 2\} > 1 = deg(P)$.
\end{ex}

\begin{defn}
A field $k$ is called \textbf{algebraically closed} provided that every nonconstant $P \in k[T]$ has a zero.
\end{defn}

\begin{prop}
Let $k$ be an algebraically closed field, $P \in k[T]$. Then 
\[P(T) = c \cdot (T - r_1)^{m_1} \cdots (T - r_n)^{m_n}\]
for $c, r_1, ..., r_n \in k, m_1, ..., m_n \in \NN$.
\end{prop}

\begin{proof}
This is proved on Homework 2. (Future readers: feel free to input the proof here when completed.)
\end{proof}

\begin{defn}
Define $\mathbf{R[T_1, ..., T_n]}$ recursively as $(R[T_1, ..., T_{n-1}])[T_n]$. More concretely: a \textbf{monomial} in $T_1, ..., T_n$ is an expression $T_1^{m_1} \cdots T_n^{m_n}$, for $m_i \in \NN$. A polynomial in $T_1, ..., T_n$ with coefficients in $R$ is an $R$-linear combination of monomials; the set of all polynomials in $T_1, ..., T_n$ is $R[T_1, ..., T_n]$.
\end{defn}

Recall: $P \in R[T] \leftrightarrow$ an eventually-zero sequence $(a_0, a_1, ...) \leftrightarrow$ a map $\NN \rightarrow R$ with finite support. So $P \in R[T_1, ..., T_n] \leftrightarrow$ a map $\NN^n \rightarrow R$ with finite support, so we may think of a polynomial $P \in R[T_1, ..., T_n]$ as
\[P(T) = \sum_{c_i \in \NN} a_{c_1, ..., c_n} \cdot T_1^{c_1} \cdots T_n^{c_n}.\]

\begin{fact}
Given a ring morphism $\phi \colon R \rightarrow S, s_1, ..., s_n \in S$, there exists a unique ring morphism \[\phi_{s_1, ..., s_n} \colon R[T_1, ..., T_n] \rightarrow S\] such that $\phi_{s_1, ..., s_n}(r) = \phi(r)$ for $r \in R$, and $\phi_{s_1, ..., s_n}(T_i) = s_i$.
\end{fact}
\begin{proof}
We induct on $n$. As a base case let $s \in S$. Fact 5.1, Jan. 13 above guarantees the existence of a unique morphism $\phi_s \colon R[T] \rightarrow S$ with the given properties. For an inductive case assume the statement holds for $n - 1$. By inductive hypothesis we have a unique morphism $\phi_{s_1, ..., s_{n-1}} \colon R[T] \rightarrow S$ satisfying the given properties. Applying the base case to $\phi_{s_1, ..., s_{n-1}}$ with $s = s_n$ gives the desired (unique) morphism, completing the inductive step and thus the proof. 
\end{proof}

\begin{defn} \hspace{0.5cm}
\begin{itemize}
\item The image of $\phi_{s_1, ..., s_n}$ is called the \textbf{$R$-subalgebra of $S$ generated by $s_1, ..., s_n$}
\item $s_1, ..., s_n$ are called \textbf{algebraically independent} provided that $\phi_{s_1, ..., s_n}$ is injective. 
\end{itemize}
\end{defn}

To gain some intuition about algebraic independence, consider the algebraic relation
\[r_1s_1^{m_{1,1}} \cdots s_n^{m_{1, n}} + \cdots + r_ns_1^{m_{n, 1}} \cdots s_n^{m_{n, n}} = 0.\]
Taking preimages under $\phi_{s_1, ..., s_n}$ we have
\begin{align*}
\phi_{s_1, ..., s_n}^{-1}(r_1s_1^{m_{1,1}} \cdots + s_n^{m_{1, n}} + \cdots + r_ns_1^{m_{n, 1}} \cdots s_n^{m_{n, n}}) &\in \phi_{s_1, ..., s_n}^{-1}(0) = \ker(\phi_{s_1, ..., s_n}) \\
\Rightarrow \phi^{-1}(r_1)T_1^{m_{1,1}} \cdots T_n^{m_{1,n}} + \cdots + \phi^{-1}(r_n)T_1^{m_{n,1}} \cdots T_n^{m_{n,n}} &\in \ker(\phi_{s_1, ..., s_n}) 
\end{align*}
thus if there exists such a relation that is nontrivial, there is a nonzero element in $\ker(\phi_{s_1, ..., s_n})$, so it is not injective. (Note from Ben: this seems to assume that the coefficients $r_i$ are in the image of $\phi$, which is not generally true. Will bring up with Tasho and edit as necessary.)

\begin{defn}
Let $R$ be a ring, $0 \neq a \in R$ not a unit. \hspace{0.5cm}
\begin{itemize}
\item $a$ is called \textbf{irreducible} provided that $a = bc \Rightarrow b \in R^{\times}$ or $c \in R^{\times}$.
\item $a$ is called \textbf{prime} provided that $a = bc \Rightarrow a \mid b$ or $a \mid c$.
\end{itemize}
\end{defn}

\begin{fact}
$R$ a domain $\Rightarrow$ every prime element of $R$ is irreducible.
\end{fact}

\begin{proof}
Let $p \in R$ be prime and write $p = ab$. Then without loss of generality we have $p \mid a \Rightarrow a = pc$ for some $c \in R$, so we may rewrite $p = pcb$, so $p(1-bc) = 0$. Since $R$ is a domain and $p \neq 0$, we have $1 = bc$, so $c = b^{-1} \Leftrightarrow b \in R^{\times}$.
\end{proof}

\begin{rmk}
\hspace{0.5cm}
\begin{itemize}
\item If $R$ is not a domain, prime elements need not be irreducible (see Homework 2).
\item Even if $R$ is a domain, irreducible elements need not be prime (see Homework 2).
\end{itemize}
\end{rmk}

\begin{defn}
Let $R$ be a domain. \hspace{0.5cm}
\begin{enumerate}
\item $R$ \textbf{has factorization} provided that every $0 \neq r \in R$ can be written $\epsilon \cdot u_1 \cdots u_k$ with $\epsilon \in R^{\times}$ and $u_i \in R$ irreducible.
\item Write $a \sim b \Leftrightarrow a = \epsilon \cdot b, \epsilon \in R^{\times}$.
\item $R$ has \textbf{unique factorization} provided that $R$ has factorization and if $\epsilon \cdot u_1 \cdots u_k = \mu \cdot v_1 \cdots v_m$ then $k = m$ and there exists a permutation $\sigma \in S_m$ such that $u_i \sim v_{\sigma(i)}$.
\item Such an $R$ is called a \textbf{unique factorization domain}, or UFD.
\end{enumerate}
\end{defn}
\end{document}
