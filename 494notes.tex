\documentclass{amsart}
\usepackage{amsfonts}
\usepackage{amsmath}
\usepackage{amsthm}
\usepackage{amssymb}
\usepackage{mathtools}
\usepackage{enumerate}
\usepackage{graphicx}
\usepackage{dsfont}
\usepackage{bbm}
\usepackage{tikz}
\usepackage{tikz-cd}
\usepackage{xifthen}
\usepackage{cancel}
\usepackage{hyperref}

\setlength{\textwidth}{\paperwidth}
\addtolength{\textwidth}{-2in}
\calclayout

\newcommand{\NN}{\mathbb{N}}
\newcommand{\RR}{\mathbb{R}}
\newcommand{\HH}{\mathbb{H}}
\newcommand{\QQ}{\mathbb{Q}}
\newcommand{\ZZ}{\mathbb{Z}}
\newcommand{\Zn}[1]{\mathbb{Z} / #1 \mathbb{Z}}
\newcommand{\CC}{\mathbb{C}}
\newcommand{\FF}{\mathbb{F}}
\newcommand{\PP}{\mathbb{P}}
\newcommand{\dd}[2][ ]{\frac{\partial #1}{\partial #2}} % Partial derivative
\newcommand{\dsq}[3][ ]{\frac{\partial^2 #1}            % Second partial
{\ifthenelse { \equal {#2} {#3} }{\partial #2^2}{\partial #2 \partial #3}}}
\newcommand{\defarr}                        %definition iff arrow
{\overset{\textrm{def}}{\Longleftrightarrow}}
\newcommand{\defeq}                         %definition equality sign
{\overset{\textrm{def}}{=}}
\newcommand{\argeq}[1]                      %definition equality sign
{\overset{\textrm{#1}}{=}}
\newcommand{\Mor}{\text{Mor}}

\DeclareMathOperator{\lcm}{lcm}
\DeclareMathOperator{\orb}{orb}
\DeclareMathOperator{\im}{im}
\DeclareMathOperator{\supp}{supp}
\DeclareMathOperator{\stab}{Stab}
\DeclareMathOperator{\sgn}{sign}
\DeclareMathOperator{\spn}{span}
\DeclareMathOperator{\tr}{tr}

\newtheorem{thm}{Theorem}[section]
\newtheorem{lemma}[thm]{Lemma}
\newtheorem*{unnumlemma}{Lemma}
\newtheorem{fact}[thm]{Fact}
\newtheorem{prop}[thm]{Proposition}
\newtheorem*{claim}{Claim}
\newtheorem{cor}[thm]{Corollary}

\theoremstyle{definition}
\newtheorem{defn}[thm]{Definition}
\theoremstyle{remark}
\newtheorem*{rmk}{Remark}
\newtheorem*{ex}{Example}
%==============================================================================
% MATH 494 Collaborative Notes
% This is a collaborative notesheet that consists of notes from each day in Math
% 494, transcribed into Tex format for convenience and security from the
% Notebook Thief. General format:
% Contribute by adding a new section, titled "Month Day, Year"
% This is done in the format \section{Date}
% Tex notes using the numbering system used by Tasho himself. I've set up the
% counters on the theorems, definitions, and "facts" to run in tandem with the
% section that the content is in.
% Make sure to properly wrap all proofs, facts, and definitions using the above
% environments.
% Enjoy!
% Current Collaborators:
% 1. Pranav
% 2. Nick
% 3. Andrew
% 4. Ben
%==============================================================================
\begin{document}
\title{Math 494: Honors Algebra II}
\maketitle
\tableofcontents

\section{January 4, 2017} %Pranav
\noindent \textbf{Rings}
\begin{defn} \label{Defn 1, Jan 4} \hspace{0.5cm}
    \begin{enumerate}[a)]
    \item A \textbf{ring} is a tuple $(R, +, \cdot, 0)$ where:
    \begin{itemize}
        \item $R$ is a set
        \item $0 \in R$
        \item $+,\cdot: R \times R \rightarrow R$, $\quad$  $(a,b) \mapsto a + b, a \cdot b$
    \end{itemize}
    subject to:
    \begin{itemize}
        \item $(R, +, 0)$ is an abelian group
        \item $(a \cdot b) \cdot c = a \cdot (b \cdot c)$
        \item $(a + b) \cdot c = a \cdot c + b \cdot c$
        \item $a \cdot (b + c) = a \cdot b + a \cdot c$
    \end{itemize}
    \item A \textbf{ring with unity} is a tuple $(R, +, \cdot, 0, 1)$, where
    $(R,+,\cdot,0)$ is a ring, and $1 \in R$ is subject to $1 \cdot a = a \cdot 1 = a$
    for all $a \in R$.
    \item A ring $(R, +, \cdot, 0)$ is called \textbf{commutative} if $ab = ba$ for all
    $a, b \in R$.
    \item A \textbf{field} is a commutative ring with unity $(R,+,\cdot,0,1)$ such
    that $(R \backslash \{0\}, \cdot, 1)$ is a group.
    \end{enumerate}
\end{defn}
\begin{rmk} \hspace{0.5cm}
    \begin{itemize}
        \item We don't really need to include 0,1 in notation: they are unique
        if they exist
        \item There is a notion of a \textbf{skew field}: ring with unity
        $(R,+,\cdot,0,1)$ such that $(R \backslash \{0\}, \cdot , 1)$ is a group.
        (This drops the commutative condition from the definition of a field).
        \item In French: \textit{corps} is a skew field, and \textit{corps commutatif} is a field.
    \end{itemize}
\end{rmk}
\begin{fact}\label{Fact 2, Jan 4}
 Let $R$ be a ring. For all $a \in R$, $0 \cdot a = 0$.
\end{fact}
\begin{proof}
    $(0 \cdot a) = (0 + 0) \cdot a = 0 \cdot a + 0 \cdot a \Rightarrow 0 = 0 \cdot a$
\end{proof}
\begin{ex} \hspace{0.5cm}
    \begin{itemize}
        \item $\ZZ$ is a ring, commutative, with unity
        \item $\QQ, \RR, \CC$ are fields
        \item $\HH = \{a + bi + cj + dk \mid a,b,c,d \in \RR \}$ where $i^2 = j^2 = k^2 = ijk = -1$ are
        called the \textbf{Hamiltonian Quaternions} and are a skew-field
        \item $\mathcal{C}_{C}(\RR) = $ functions on $\RR$ with compact
        support \\
        ($\supp(f) = \overline{\{x \in \RR \mid f(x) \neq 0\}}$) is a
        commutative ring without unity
        \item $R = \{\star\}, 0 = 1 = \star$ is the \textbf{zero ring}.
    \end{itemize}
\end{ex}
\begin{fact} \label{Fact 3, Jan 4}
    If $(R,+,\cdot,0,1)$ is a ring with unity and $0 = 1$, then $R$ is the zero ring.
\end{fact}
\begin{proof}
    Take $a \in R$. Then $a = a \cdot 1 = a \cdot 0 = 0$ by Fact \ref{Fact 2, Jan 4}.
\end{proof}
\noindent \underline{Convention}: Unless otherwise noted, ring will refer to
a commutative ring with 1.
\begin{defn} \label{Defn 4, Jan 4}
    Let $R$ be a ring. Its \textbf{group of units} is
    $$
    R^\times = \{a \in R \mid \exists \, b \in R: ab = 1\}
    $$
\end{defn}
\begin{fact} \label{Fact 5, Jan 4} \hspace{0.5cm}
    \begin{itemize}
        \item For $a \in R^\times$, there is a unique $b \in R$ such that $ab = 1$.
        Write $b = a^{-1}$.
        \item For $a,b \in R^\times$, $a \cdot b \in R^\times$.
    \end{itemize}
\end{fact}
\begin{proof} \hspace{0.5cm}
    \begin{itemize}
        \item Given $b, b^\prime$, we have $b = b \cdot 1 = b(ab^\prime) = (ba)b^\prime = 1 \cdot b^\prime
         = b^\prime$.
         \item $(a \cdot b) \cdot (b^{-1} \cdot a^{-1}) = 1$
    \end{itemize}
\end{proof}
\begin{ex}
    $\RR^\times = \RR \backslash \{0\}$, $\ZZ^\times = \{1, -1\}$
\end{ex}
\begin{defn} \label{Defn 6, Jan 4}
    Let $R, S$ be rings. A \textbf{morphism} $\phi:R \rightarrow S$ is a map of
    sets $\varphi:R \rightarrow S$ satisfying
    \begin{itemize}
        \item $\varphi(a + b) = \varphi(a) + \varphi(b)$
        \item $\varphi(a \cdot b) = \varphi(a) \cdot \varphi(b)$
        \item $\varphi(1) = 1$
    \end{itemize}
\end{defn}
\begin{ex}
    $\varphi:\ZZ \rightarrow \ZZ$ $u \mapsto 0$ is \underline{not} a morphism of
    rings with 1. (it is a morphism of general rings).
\end{ex}
\begin{fact}\label{Fact 7, Jan 4}
    For any ring $R$ there is a unique morphism $\varphi:\ZZ \rightarrow R$. Given
    $z \in \ZZ$, we write $z_{R}$, or simply $z$ for its image under $\varphi$.
\end{fact}
\begin{ex}
    $5 \in \ZZ$, $5_{\QQ} \in \QQ$ usual number $5$. $5_{\Zn 2} = 1_{\Zn 2}$
\end{ex}
\begin{defn} \label{Defn 8, Jan 4}
    Let $R$ be a ring. A subset $I \subset R$ is called an \textbf{ideal} if
    \begin{itemize}
        \item $I$ is a subgroup of $(R, +, 0)$
        \item $a \cdot f \in I$ for all $a \in R, f \in I$.
    \end{itemize}
\end{defn}
\begin{defn} \label{Defn 9, Jan 4}
    Let $R$ be a ring. A subset $S \subset R$ is called a \textbf{subring} if
    \begin{itemize}
        \item $S$ is a subgroup of $(R, +, 0)$
        \item $a \cdot b \in S$ for all $a, b \in S$.
        \item $1 \in S$.
    \end{itemize}
\end{defn}
\begin{rmk}\hspace{0.5cm}
    \begin{itemize}
        \item The only subset that is both a subring and an ideal is $R$ itself.
        (reason: if $1 \in I$, then $a \cdot 1 \in I$ for all $a \in R$, meaning $I = R$)
        \item $I = \{0\}, I = R$ are always ideals.
        \item In rings without unity, the 2 notions align closer: ideal becomes a special
        case of subring as $1 \in S$ condition is dropped.
    \end{itemize}
\end{rmk}
\begin{ex} \hspace{0.5cm}
    \begin{itemize}
        \item Every subgroup of $(\ZZ, +, 0)$ is an ideal of $\ZZ$.
        \item If $F$ is a field, then $\{0\}, R$ are the only ideals
        \item Let $R = \mathcal{C}_C(\RR), S \in R$ subset.
        $$
        I = \{f \in \mathcal{C}_C(\RR) \mid f \mid_{S} = 0 \}
        $$
        is an ideal
    \end{itemize}
\end{ex}
\begin{defn} \label{Defn 10, Jan 4}
    An ideal $I \in R$ is called \textbf{principal} if $I = \{a \cdot r \mid r \in R\}$
    for some $a \in R$. Then $a$ is called a \textbf{generator}.
\end{defn}
\begin{defn} \label{Defn 11, Jan 4}
    Let $a_1, a_2, \dots a_n \in R$. An \textbf{ideal generated by} $a_1, \dots a_n$ is
    $$
    (a_1, \dots a_n) = \{a_1r_1 + \dots + a_nr_n \mid r_i \in R\}
    $$
\end{defn}
\begin{fact} \label{Fact 12, Jan 4}
    Given ideals $I, J \subset R$ we have
    \begin{itemize}
        \item $I \cap J$ is an ideal
        \item $I + J = \{a + b \mid a \in I, b \in J \}$ is an ideal
        \item $I \cdot J = \left\{\sum\limits_{i = 1}^{n}a_ib_i \mid a_i \in I, b_i \in J \right\}$ is an ideal
    \end{itemize}
\end{fact}

\section{January 6, 2017} %Pranav
\begin{fact}
    Let $\varphi: R \rightarrow S$ be a morphism. Then
    $$
    \ker(\varphi) = \{x \in R \mid \varphi(x) = 0\}
    $$
    is an ideal.
\end{fact}
\begin{proof} (A Pranav Exclusive)
    We first show that the kernel is a subgroup of $(R, +, 0)$. Well, we first show that
    $0 \in \ker(\varphi)$. Well,
    $$
    \varphi(0) = \varphi(0 + 0) = \varphi(0) + \varphi(0)
    $$
    so, we have that $\varphi(0) = 0$ and thus $0 \in \ker(\phi)$.
    Next, we show that inverses are in the kernel as well. \\
    If we have that $\varphi(a) = 0$, then we have
    $$0 = \varphi(0) = \varphi(a + (-a)) = \varphi(a) + \varphi(-a) = \varphi(-a)$$
    Now, we complete this step by proving closure. Assume $a,b \in \ker(\varphi)$. Then,
    $$\phi(a + b) = \phi(a) + \phi(b) = 0 + 0 = 0$$
    Thus, we have that the kernel is a subgroup. Now, we verify the second condition.
    Fix $a \in R$ and $f \in \ker(\varphi)$. We have that
    $$
    \phi(a \cdot f) = \phi(a) \cdot \phi(f) = \phi(a) \cdot 0 = 0
    $$
    Thus, we have that $a \cdot f \in \ker(\varphi)$, meaning that $\ker(\varphi)$
    is an ideal.
\end{proof}
\noindent Question: Is every ideal the kernel of morphism?
\begin{prop}
    Let $R$ be a ring, $I \subset R$ an ideal. Let $R / I$ be the quotient of
    abelian groups and $p: R \rightarrow R / I$ the canonical projection. Then there is a
    unique product map
    $$
    \cdot: R/I \times R/I \rightarrow R/I
    $$
    making $R / I$ into a ring such that $p$ is a morphism.
\end{prop}
\begin{proof}
    For $p$ to be a morphism of rings, we need
    \begin{itemize}
        \item $p(1_R) = 1_{R/I}$
        \item The following diagram to commute
        \[ \begin{tikzcd}[%
        ,every arrow/.append style={maps to}
        ,every label/.append style={font=\normalsize}
        ,row sep=1.5cm
        ,column sep=1.5cm]
        R \times R \arrow{r}{\cdot_R} \arrow[swap]{d}{p \times p} & R \arrow{d}{p} \\%
        R/I \times R/I \arrow{r}{\cdot_{R/I}}& R/I
        \end{tikzcd}
        \]
    \end{itemize}
    Uniqueness of $\cdot_{R/I}$ follows from surjectivity of $p \times p$ (each element
    in $R / I \times R/I$ must go precisely to the result of the composition of $p$ and
    $\cdot_R$) \\
    For existence, define $1_{R/I} = p(1_R)$ and $(a + I) \cdot (b + I) \defeq (a \cdot b) + I$.
    We have to show this is well-defined (i.e it is independent of choice of $a, b$). \\
    Well, choose $a^\prime, b^\prime$ such that $a^\prime  + I = a + I, b^\prime + I = b + I$.
    Thus, $a^\prime = a + i$, $b^\prime = b + j$ for some $i, j \in I$. Then
    $$
    (a^\prime + I)(b^\prime + I) = (a^\prime \cdot b^\prime) + I = ((a + i)\cdot(b + j)) + I =
    (a\cdot b + a \cdot j + b \cdot i + i \cdot j) + I = a\cdot b + I
    $$
    as we note that $a \cdot j, b \cdot i$, and $i \cdot j$ are all in $I$ as $I$
    is an ideal. \\
    We have that all of the ring axioms for $R / I$ are inherited from the ring
    structure on $R$.
\end{proof}
\begin{rmk}
    $\ker(p) = I$
\end{rmk}
\begin{thm}\label{thm:homomorphism} \textbf{(Homomorphism Theorem):}
    Let $\phi: R \rightarrow S$ be a morphism of rings, $I \subset \ker(\varphi)$ be
    an ideal of $R$. There is a unique morphism $\overline{\varphi}: R/I \rightarrow S$
    such that $\overline{\varphi} \circ p = \varphi$ i.e.
    \[
\begin{tikzcd}[,every arrow/.append style={maps to}
,every label/.append style={font=\normalsize},column sep=1.5em]
 & R/I \arrow{dr}{\overline{\varphi}} \\
R \arrow{ur}{p} \arrow{rr}{\varphi} && S
\end{tikzcd}
\]
commutes. Moreover, $\overline{\varphi}$ is injective $\iff$ $\ker(\varphi) = I$
\end{thm}
\begin{proof}
    All statements follow from looking at the abelian group $(R, +, 0)$ and its
    subgroup $I$, except multiplicativity of $\overline{\varphi}$. \\
    (A Pranav Exclusive) Some justification:
    the uniqueness of this morphism follows because the projection map is surjective, meaning that in
    order for the composition to be commutative, we must have that each element in $R / I$ maps exactly
    to where its associated element maps under $\varphi$. Now, the existence. We simply need to check that
    the map $\overline{\varphi}$ that sends $a + I$ to $\varphi(a)$ is well defined and is a morphism. We note that
    the additive morphism properties are inherited from the fact that $\varphi$ is a morphism itself. So, we check
    the well-definedness of $\overline{\varphi}$. Pick 2 representatives of $a + I$, call them $a + I$ and
    $a^\prime + I$. We have that $a^\prime = a + i$ for $i \in I$. Then, we have that
    $$
    \overline{\varphi}(a^\prime + I) = \overline{\varphi}(a + i + I) = \overline{\varphi}(a + I) + \overline{\varphi}(i + I) =
    \overline{\varphi}(a + I) + \overline{\varphi}(I) = \overline{\varphi}(a + I) + 0
    $$
    as we have that $\varphi(i) = 0$ for all $i \in I$ (since $I \subset \ker(\varphi)$). We finally verify the injective
    biconditional. Assume $\overline{\varphi}$ is injective. We already have that $I \subset \ker(\varphi)$.
    Now, since $\overline{\varphi}$ is injective, its kernel is trivial, and is thus the identity of $R / I$, namely $I$ itself.
    For any $g \in \ker(\varphi)$ we note that $g + I$ must belong to the kernel of $\overline{\varphi}$, meaning that
    $g + I = I$ and thus $g \in I$. This gives us double containment and thus equality. \\
    Now, assume that $\ker(\varphi) = I$. We consider $\ker(\overline{\varphi})$. This is exactly the collection
    $\{a + I \mid a \in \ker(\varphi) \}$. Thus, this is $\{a + I \mid a \in I\}$ and thus we have that
    $\ker(\overline{\varphi}) = I$. Since the kernel of $\overline{\varphi}$ is trivial, we have that
    $\overline{\varphi}$ is injective. \\
    Checking Multiplicativity: Let $A,B \in R/I$. Choose $a,b \in R$ such that $p(a) = A, p(b) = B$. Then
    $$
    \overline{\varphi}(A \cdot B) = \overline{\varphi}(p(a)\cdot p(b)) =
    \overline{\varphi}(p(ab)) = \varphi(ab) = \varphi(a)\varphi(b) =
    \overline{\varphi}(p(a))\overline{\varphi}(p(b)) = \overline{\varphi}(A)\overline{\varphi}(B)
    $$
\end{proof}
\begin{defn}
    Let $R$ be a ring.
    \begin{itemize}
        \item Let $a,b \in R$. We say that \textbf{$a$ divides $b$} (denoted $a \mid b$)
        if there is $c \in R$ such that $ac = b$.
        \item We say $0 \neq a \in R$ is a \textbf{zero divisor} if there is
        $0 \neq b \in R$ such that $ab = 0$.
        \item We call $R$ a \textbf{domain} (or \textbf{integral domain}) if it has
        no zero divisors.
    \end{itemize}
\end{defn}
\begin{fact}
    $a \mid b \iff (b) \subset (a) \iff b \in (a)$
\end{fact}
\begin{proof} (A Pranav Exclusive)
    We first show the first forward implication. Assume that $a \mid b$. Then, there is
    $c \in R$ such that $ac = b$. Now, fix $g \in (b)$. It is of the form $br$ for some
    $r \in R$. Thus, we have that $g = (ac)r = a(cr)$. Since $cr \in R$, we have that $g \in (a)$. \\
    Next, we show the second forward implication. Assume that $(b) \subset (a)$. Well,
    $b \in (b) \subset (a)$. \\
    Finally, we show that $b \in (a)$ implies the original condition. Well, if $b \in (a)$, then
    $b = ar$ for $r \in R$. This is exactly what it means for $a \mid b$! Thus, we have
    shown equality of the above statements.
\end{proof}
\begin{fact} (Cancellation Law) If $a \neq 0 \in R$ is not a zero divisor, then
    for $x,y \in R$
    $$
    ax = ay \Rightarrow x = y
    $$
\end{fact}
\begin{proof}
    $ax = ay \iff a(x - y) = 0$. $a \neq 0$ implies that $x - y = 0$ as $a$ is not
    a zero divisor.
\end{proof}
\begin{defn}
    An ideal $I \subsetneq R$ is called
    \begin{itemize}
        \item \textbf{prime} if $a \cdot b \in I$ implies $a \in I$ or $b \in I$ for all $a,b \in R$.
        \item \textbf{maximal} if $I$ and $R$ are the only ideals containing $I$.
    \end{itemize}
\end{defn}
\begin{ex}
    In $R = \ZZ$, the ideals are of the form $n\ZZ$. $n\ZZ$ is prime $\iff$ $n$ is prime or $n = 0$.
\end{ex}
\begin{proof}
    (A Pranav Exlusive). We start with the forward direction. We proceed by contrapositive.
    Assume that $n \neq 0$ and that $n$ is not prime. Then, $n$ is composite (we exclude $n = 1$ as
    we must have a properly contained ideal by definition). Thus, we have that $n = ab$ for some $1 < a,b < n$.
    Note that we have $ab = n \in n\ZZ$, but we have that both $a$ and $b$ are less than $n$, and thus there is no $z \in \ZZ$
    such that $nz = a$ or $nz = b$. This means that $n\ZZ$ is not prime, as we have found $a,b$ such that
    $ab \in n\ZZ$ but neither $a$ nor $b$ are in $n\ZZ$. \\
    Now, the reverse direction. First, we show the condition for $n$ prime. Assume that we have $a,b \in \ZZ$ such that $ab \in n\ZZ$.
    This means that we have $ab = nq$ for some $q \in \ZZ$. In particular, this means that $n$ divides
    the product $ab$. However, we note that as $n$ is prime, we have that $n$ must divide $a$
    or $b$ by Euclid's lemma. Thus, we have that either $a = nr$ or $b = nr$ (or both), which implies
    that $a \in n\ZZ$ or $b \in n\ZZ$. Next, for $n = 0$. Well, if $ab \in 0\ZZ$, then $ab = 0$. This in $\ZZ$ implies
    that either $a$ or $b$  is $0$ and is also in $n\ZZ$. This completes the reverse direction.
\end{proof}
\begin{thm}\label{thm:ideals}
    Let $R$ be a ring.
    \begin{enumerate}[i)]
        \item $R$ is a domain $\iff \{0\}$ is prime.
        \item $R / I$ is a domain $\iff I \subset R$ is a prime ideal.
        \item Let $\varphi: R \rightarrow S$ be a morphism, $S$ a domain. Then
        $\ker(\varphi)$ is prime. The converse is true if $\varphi$ is surjective.
        \item $R$ is a field $\iff \{0\}$ is maximal.
        \item $R / I$ is a field $\iff I \subset R$ is a maximal ideal.
        \item Every field is a domain.
        \item Every maximal ideal is prime.
    \end{enumerate}
\end{thm}
\begin{proof}
    We first claim that iii) implies ii) which in turn implies i). First, for iii) implies
    ii), we note that letting $S$ be $R / I$ (which means $\varphi$ is the projection
    map $p$ (which is definitely surjective)) gives us ii). (We have that $\ker(p) = I$).\\
    ii) implies i) simply by letting $I$ be the zero ideal. \\
    Now, we prove statement iii). \\
    Let $a, b \in R$ such that $a \cdot b \in \ker(\varphi)$. Then $0 = \varphi(a \cdot b) = \varphi(a)\varphi(b)$.
    Since we have that $S$ is a domain, then we have no zero divisors, meaning that either
    $\varphi(a) = 0$ or $\varphi(b) = 0$. This in turn implies that either $a \in \ker(\varphi)$ or $b \in \ker(\varphi)$,
    so we have show that $\ker(\varphi)$ is a prime ideal. Now, the converse assuming surjectivity.
    We want to show that $S$ has no zero divisors. Well, fix $A,B \in S$ such that $A \cdot B = 0$.
    Since $\varphi$ is surjective, we have $a, b \in R$ such that $\varphi(a) = A$ and $\varphi(b) = B$.
    Then, we have $0 = \varphi(a)\varphi(b) = \varphi(ab)$, meaning that $ab$ is in $\ker(\varphi)$.
    Because we assume that $\ker(\varphi)$ is prime, this in turn implies that either
    $a$ or $b$ is in $\ker(\varphi)$ meaning that either $\varphi(a) = 0$ or $\varphi(b) = 0$.
    This means that either $A$ or $B$ is 0, and thus $S$ is a domain, as desired. \\
    Next, note that v) implies iv). This comes from letting $I$ be the zero ideal. \\
    The proof of v) comes from the bijection
    $$
    \{\textrm{ideals in $R$ containing $I$}\} \leftrightarrow \{\textrm{ideals in $R/I$}\}
    $$
    This is a homework problem. \\
    Now, we show vi). Assume that $F$ is a field. Pick $a,b \in F$ such that $a \cdot b = 0$
    with $a \neq 0$. We will show that $b$ must be 0, thereby showing that $F$ is a domain.
    Well, since $a \neq 0$, and $F \backslash \{0\}$ is a group, we have that $a^{-1}$ exists.
    Thus, we have that $ab = 0$ implies that $a^{-1}ab = 0$ and thus $b = 0$, as desired. \\
    vii) follows from the facts vi), v) and ii). We have that
    \begin{center}
        $I$ is a maximal ideal $\overset{\textrm{v}}{\iff}$ $R/I$ is a field $\overset{\textrm{vi}}{\Rightarrow}$ $R / I$ is a domain
        $\overset{\textrm{ii}}{\iff}$ I is prime.
    \end{center}
\end{proof}

\section{January 9, 2017}
\begin{defn}
    Let $R$ b a domain. The canonical morphism $\ZZ \rightarrow R$ of Fact \ref{fact:!morph} has
    a prime ideal as its kernel. By Thm \ref{thm:ideals}, this is of the form
    $p\ZZ$ with $p$ prime of $p = 0$. We call $p$ the \textbf{characteristic} of $R$.
\end{defn}
\begin{ex}
\[
\begin{tabular}{ll}
    char$(\ZZ) = 0$ & char$(\Zn 3) = 3$ \\
    char$(\QQ) = 0$ & char$(\Zn 6)$ doesn't exist! $\Zn 6 $ is not a domain.
\end{tabular}
\]
\end{ex}
\begin{unnumlemma}
\textbf{(Zorn's Lemma)} (from Artin). An \textbf{inductive} (every totally ordered
subset has an upper bound) partially ordered set $S$ has at least one maximal element.
\end{unnumlemma}
\begin{thm}\label{thm:maxext}
    Let $R$ be a ring. Every proper ideal is contained in a max ideal.
\end{thm}
\begin{proof}
    Let $I \subset R$ be a proper ideal. Let $\mathcal{M}$ be the set of all proper
    ideals of $R$ that contain $I$, with partial order given by inclusion. \\
    Let $\mathcal{C} \subset \mathcal{M}$ be a totally ordered subset.
    \begin{claim}
        $J_0 = \left(\bigcup\limits_{J \in \mathcal{C}}{J}\right) \in \mathcal{M}$
    \end{claim}
    \begin{proof} (of claim). We want to show that $J_0$ is a proper ideal containing
        $I$. First, we show it is an ideal by showing closure of the subgroup and the
        ideal multiplicative closure. Let $f_1, f_2 \in J_0$ and $a \in R$. Now, this means there
        is $J_1, J_2 \in \mathcal{C}$ such that $f_1 \in J_1$ and $f_2 \in J_2$. However,
        since $\mathcal{C}$ is totally ordered, we have that the larger of $J_1$ and $J_2$
        contains both $f_1$ and $f_2$, meaning that we have the existence of $J \in \mathcal{C}$
        such that $f_1,f_2 \in J$. Since $J$ is an ideal, we have that $f_1 + f_2 \in J$ and that
        $a \cdot f_1 \in J$. This thus implies that since $J \in \mathcal{C}$, we have that
        $a \cdot f_1$ and $f_1 + f_2$ are both in $J_0$. Thus $J_0$ is an ideal. Since $I \in \mathcal{C}$,
        we also have that $I \subset J_0$. Finally, $J_0$ is not $R$, because otherwise $1 \in J_0$,
        which would mean that $1 \in J$ for some $J \in \mathcal{C}$. This would then imply that
        that $J = R$, which is not possible as $J$ itself is a proper ideal. Thus, we have that $J_0 \in \mathcal{M}$.
    \end{proof}
    Thus, for every totally ordered subset of $\mathcal{M}$, we have the existence of an
    upper bound (namely $J_0$). This gives us, by Zorn's Lemma, that $\mathcal{M}$ has
    a maximal element. This maximal element is exactly what we wished to show existed.
\end{proof}
\begin{defn}
Let $R,S$ be rings. Their product is the set $R \times S$ with component-wise
operations
\begin{itemize}
    \item $(r,s) + (r^\prime, s^\prime) = (r + r^\prime, s + s^\prime)$
    \item $(r,s) \cdot  (r^\prime, s^\prime) = (r \cdot r^\prime, s \cdot s^\prime)$
    \item $1_{R \times S} = (1_R, 1_S), 0_{R \times S} = (0_R, 0_S)$
\end{itemize}
\end{defn}
\begin{rmk}
    Given morphisms $\varphi_1:R \rightarrow S_1, \varphi_2:R \rightarrow S_2$, we
    get a unique morphism $\varphi_{1} \times \varphi_2: R \rightarrow S_1 \times S_2$.
\end{rmk}
\begin{rmk}
    Given $I,J \subset R$ ideals we have
    $$
    I \cdot J \subset I \cap J \subset I, J \subset I + J
    $$
\end{rmk}
\begin{defn}
    Two ideals $I,J \subset R$ are \textbf{coprime} if $I + J = R$.
\end{defn}
\begin{thm}
    \textbf{(Chinese Remainder Theorem)} Let $R$ be a ring, $I_1, \dots I_n \subset R$
    be pairwise coprime ideals. Then the natural morphism
    $$
    p: R \rightarrow R / I_1 \times R / I_2 \times \dots \times R / I_n
    $$
    factors through the quotient $R / (I_1 \cap I_2 \cap \dots \cap I_n)$ and induces
    an isomorphism of rings
    $$
    \overline{p}: R / (I_1 \cap I_2 \cap \dots \cap I_n) \rightarrow R / I_1 \times R / I_2 \times \dots \times R / I_n
    $$
    Moreover, $I_1 \cdot I_2 \cdots I_n = I_1 \cap I_2 \cap \dots I_n$
\end{thm}
\begin{proof}
    As $p$ is the natural morphism to a product of rings, we let $p = p_1 \times p_2 \dots \times p_n$,
    where each $p_i$ is the projection morphism from $R$ to $R / I_i$. Now, we can say that
    $\ker(p) = \{r \in R \mid 0 = p_1(r), 0 = p_2(r), \dots 0 = p_n(r) \}$. Well, since each $p_i$ by definition has
    kernel exactly $I_i$, this is the same as saying that
    $\ker(p) = \{r \in R \mid r \in I_1 \cap I_2 \cap \dots \cap I_n \}$. \\
    By the homomorphism theorem (\ref{thm:homomorphism}), we have that $p$ factors through
    $R / I_1 \cap I_2 \cap \dots I_n$ and also induces an injective ring morphism
    $\overline{p}: R/ I_1 \cap \dots \cap I_n \rightarrow R/I_1 \times \dots R/I_n$.
    \begin{claim}
        $\overline{p}$ is also surjective, and hence a isomorphism.
    \end{claim}
    \begin{proof} (of claim)
        We note that since each of the ideals are coprime, we have that $I_1 + I_k = R$.
        Now, we also note that $R \cdot R = R$. Thus, we can express
        $$
        R = (I_1  + I_2) \cdot (I_1 + I_3) \cdots (I_1 + I_n)
        $$
        expanding the product, we note that by the earlier remark that any term
        containing an $I_1$ (which is almost all of them) will be contained in
        $I_1$. The only term that is outside arises from selecting the second term
        in every single term of the product, so we can write that the above expression
        is
        $$
        \subset I_1 + (I_2 \cdot I_3 \cdots I_n)
        $$
        Now, since $R \subset I_1 + (I_2 \cdot I_3 \cdots I_n)$, we can take
        $v_1 \in I_1$ and $u_1 \in I_2 \cdots I_n$ such that $u_1 + v_1 = 1$.
        Now, since $u_1 \in I_2 \cdots I_n$, $u_1 \in I_j$ for $j \neq 1$. Thus,
        we can say that $u_1$ maps to $0_{R/I_j}$ under the projection map, as it is
        in the kernel. \\
        Similarly, since $u_1 = 1 - v_1$, with $v_1 \in I_1$, we have that $u_1 \in 1 + I$,
        meaning that $u_1$ maps to $1_{R/I_1}$ under the projection map. \\
        So, we have (abusing notation) that $u_1 = 1$ in $R/I_1$ and $u_1 = 0$ in $R/I_j$ for $j \neq 1$
        (really, as we showed above, it belongs to the associated cosets). \\
        Now, we can repeat this construction with any $I_i$ instead of $I_1$.
        Thus, we get for each such construction a $v_i \in I_i$ and $u_i \in I_1 \cdot I_2 \cdots \widehat{I_i} \cdots I_n$
        With this construction, we now have the existence of the $u_i$ that belong
        to the $1$ coset in exactly $R/I_i$ and the $0$ coset in all remaining $R/I_j$.
        With this, we can prove surjectivity.
        Fix any $(x_1, \dots x_n) \in R/I_1 \times \dots R/I_n$. We have that there exists
        an associated $r_1, \dots r_n \in R$ such that $p_1(r_1) = x_1, \dots p_n(r_n) = x_n$.
        Now, if we consider the element $r \in R$ that equals $u_1r_1 + u_2r_2 \dots u_nr_n$,
        note that $p(r) = (p_1(r), p_2(r) \dots p_n(r))$. However, since the $u_i$ map to $1$ under
        $p_i$ and to $0$ otherwise, this maps precisely to $(x_1, \dots x_n)$. Thus, we have
 that $p(r)$ maps to the desired element in the product, meaning that the
        associated coset will map to the desired element under $\overline{p}$. This proves
        surjectivity.
    \end{proof}
    Thus, we have that $\overline{p}$ is an isomorphism. Now, we show the second part of
    part of the statement. \\
    Well, we know by definition that $I_1 \cdot I_2 \cdots I_n \subset I_1 \cap \dots \cap I_n$. So,
    we simply need to show the other containment, which we do by induction on $n$. \\
    $n = 1$: $I_1 \subset I_1$. \\
    $n = 2$: Take $u_1 \in I_1$ and $u_2 \in I_2$ such that $1 = u_1 + u_2$ (this exists as $I_1 + I_2 = R$.)
    Now, for any $u \in I_1 \cap I_2$, we have
    $$
    u = u \cdot 1 = u \cdot (u_1 + u_2) = u \cdot u_1 + u \cdot u_2
    $$
    Since $u \in I_1$ and $u \in I_2$, we have $u \cdot u_1 \in I_2 \cdot I_1$ and
    $u \cdot u_2 \in I_1 \cdot I_2$. Thus, we have the sum in $I_1 \cdot I_2$. This gives us
    $I_1 \cap I_2 \in I_1 \cdot I_2$. \\
    Now, for general $n$. By the inductive hypothesis, we have that
    $I_1 \cap I_2 \dots I_n \subset (I_1 \cdots I_{n-1}) \cap I_n$. From the claim above,
    we know that $R = (I_1 \cdot I_{n-1}) + I_n$. This implies thus that the ideals
    $(I_1 \cdots I_{n-1})$ and $I_n$ are coprime. Thus, applying the $n = 2$ case on these
    2 ideals, we have that $(I_1 \cdots I_{n-1}) \cap I_n \subset (I_1 \cdots I_{n-1}) \cdot I_n$,
    thereby proving the desired result.
\end{proof}

\section{January 11, 2017}
\begin{rmk} \hspace{0.5cm}
\begin{itemize}
        \item Any field is a domain.
        \item Any subring of a domain is a domain.
        \item Any subring of a field is a domain.
    \end{itemize}
Is the opposite true?
\end{rmk}
\begin{thm} \label{Thm 1, Jan 11}
Let $R$ be a domain.

1) There exists a pair $(i,K)$ with $K$ a field, $i:R\rightarrow K$ an injective morphism such that if $(j,L)$ is another such pair, there exists a morphism $l:K\rightarrow L$ such that $j=l\circ i,$ which is to say that the following diagram commutes.

\begin{tikzcd}[,every arrow/.append style={maps to}
,every label/.append style={font=\normalsize},column sep=1.5em]
 & R \arrow{dr}{j}\arrow{dl}{i} \\
K \arrow{rr}{l} && L
\end{tikzcd}

2) If $(i',K')$ is another pair as in 1) there exists a unique isomorphism $\phi:K\rightarrow K'$ such that

\begin{tikzcd}[,every arrow/.append style={maps to}
,every label/.append style={font=\normalsize},column sep=1.5em]
 & R \arrow{dr}{i'}\arrow{dl}{i} \\
K \arrow{rr}{\phi} && K'
\end{tikzcd} commutes.

\end{thm}

\begin{rmk}

\begin{itemize}
	\item $(i,K)$ is an example of a ``universal object''
	\item $(j,L)$ is called a test object
	\item $K$ is produced from $R,$ just like the rationals are produced from the integers.
\end{itemize}

\end{rmk}
\begin{proof}

2) Given two universal objects $(i,K),(i',K'),$ apply 1) with $(i,K)$ as the universal object, and $(i',K')$ as a test object to get $l:K\rightarrow K'$. Do it the other way to get $l':K'\longrightarrow K.$

\begin{claim}

$l\circ l'=\text{id}_{K'},l'\circ l=\text{id}_K$

\begin{proof}

Note that both $l'\circ l$ and $\text{id}_k$ make the diagrams

\begin{tikzcd}[,every arrow/.append style={maps to}
,every label/.append style={font=\normalsize},column sep=1.5em]
 & R \arrow{dr}{i}\arrow{dl}{i} \\
K \arrow{rr}{l'\circ l} && L
\end{tikzcd}

\begin{tikzcd}[,every arrow/.append style={maps to}
,every label/.append style={font=\normalsize},column sep=1.5em]
 & R \arrow{dr}{i}\arrow{dl}{i} \\
K \arrow{rr}{\text{id}_K} && L
\end{tikzcd} commute.

When $(i,k)$ is both a universal object and a test object, we get $l\circ l=\text{id}_K.$ \end{proof}
\end{claim}


1) Consider the set $P=R\times R\setminus\{0\}$. Introduce the relation $(n,d)\sim (n',d')\iff nd'=n'd$.

\begin{claim}

$\sim$ is an equivalence relation.

\begin{proof}

Reflexive: $(n,d)\sim(n,d)\iff nd=nd.$

Symmetric $(n,d)\sim(n',d')\iff nd'=n'd\iff n'd=nd'\iff (n',d')\sim (n,d).$

Transitive: Assume $(n_1,d_1)\sim(n_2,d_2)\sim(n_3,d_3)$. We want $(n_1,d_1)\sim(n_3,d_3).$
We have $n_1d_2=n_2d_1,n_2d_3=n_3d_2$ and want $n_1d_3=n_3d_1.$
We see that $n_1d_3 n_2d_2=n_1d_3n_2d_3=n_2d_1n_3d_2=n_3d_1n_2d_2.$
Since $R$ is a domain, $n_2d_2$ is not a zero-divisor. If $n_2d_2\ne 0$, then by \ref{Fact 6, Jan 6}, we get $n_1d_3=n_3d_1$. If $n_2d_2=0$, then ($d_2\ne 0$ and not a 0-divisor) $n_2=0$. For the same reason, $n_1=n_3=0.$ Again, $n_1d_3=n_3d_1$. Either way, we are done.
\end{proof}
\end{claim}

Put $K=P/\sim.$ Write $[n,d]$ for the image of $(n,d)\in P$ in $K$. Define

$$[n,d]\cdot[n',d']=[nn',dd']$$

$$[n,d]+[n',d']=[nd'+n'd,dd']$$

$$0=[0,1],1=[1,1]$$

$$i:R\rightarrow K, i(r)=[r,1].$$

We leave as homework the verifications that $+,\cdot$ are well defined, that $K$ is a field, and that $i$ is a morphism. Injectivity is obvious. Given $(j,L)$, define $l:K\longrightarrow L$ by

$$l([n,d])=l(i(n))\cdot l(i(d)^{-1})=j(n)j(d)^{-1}.$$

Homework: $l$ is well defined and a ring morphism.

\end{proof}

\begin{defn}\label{Defn 2, Jan 11}

A pair $(i,K)$ is called a (the) field of fractions (fraction field) of $R$.

\end{defn}

\begin{defn} \label{Defn 3, Jan 11}

1) Let $R$ be a ring. A \textbf{polynomial in $T$ over $R$} is a formal expression $a_nT^n+a_{n-1}T^{n-1}+\ldots+a_0,a_i\in R.$

2) Given $P(T)=a_nT^n+\ldots+a_0,Q(T)=b_nT^n+\ldots+b_0$ define

$$(P+Q)(T)=(a_n+b_n)T^n+\ldots+(a_0+b_0)$$

$$(P\cdot Q)(T)=(c_mT^m+c_{m-1}T^{m-1}+\ldots+c_0$$ where

$$c_k=\sum_{i+j=k}a_i\cdot b_j.$$

3) Given $r\in R$ we have the constant polynomial
$r:(a_nT^n+\ldots+a_0,a_0=r,a_i=0\text{ for } i>0).$ In particular, we have $0,1$ as constant polynomials.

4) Let $R[T]$ be the set of al polynomial in $T$ over $R$.

\begin{fact} \label{Fact 4, Jan 11}
    $(R[T],+,\cdot,0,1)$ is a ring. Moreover $R\rightarrow R[T]$, $r\rightarrow$ constant polynomial $r$ is an injective morphism. The proof is left as an exercise to the reader.
\end{fact}

\end{defn}

\begin{defn} \label{Defn 5, Jan 11}

Given $0\ne P\in R[T],$ define deg$(P)=\min\{n|a_m=0\forall m>n\}$, deg$(0)=-\infty$.
\end{defn}
\begin{fact}\label{Fact 6, Jan 11}
1) deg$(P+Q)\leq\max(\text{deg}(P),\text{deg}(Q))$ with equality if deg$(P)\ne$ deg$(Q)$.
2) deg$(P\cdot Q)\leq$ deg$(P)+$deg$(Q)$ with equality if the leading coefficient of $P$ (or $Q$) is not a 0 divisor.
3) In particular, if $R$ is a domain, so is $R[T].$ The proof is left as an exercise to the reader.
\end{fact}

\section{January 13, 2017} % Andrew

\begin{rmk}
Any $P \in R[T]$ gives a function $R \rightarrow R$ by $r \mapsto P(r) = a_nr^n + ... + a_0$.  However, $P$ is not necessarily determined by this function.  For example, let $R = \ZZ / p\ZZ$ where $p$ is a prime and $P(T) = T^p - T$.  Since $x^p = x$ for all $x \in R$, $P$ and $0$ give the same function.  However, $P \neq 0$.
\end{rmk}

\begin{ex}

\item $P = T^2 + 3T - 2$, $Q = -T^2 + 3T - 7$ gives $P+Q = 6T - 9$ ($\text{deg}(P+Q) < \max(\text{deg}(P),\text{deg}(Q))$)

\item $R = \ZZ/4\ZZ$, $P = 2T^2+1$, $Q = 2T^3+3T$ gives $PQ = 3T$ ($\text{deg}(PQ) < \text{deg}(P)+\text{deg}(Q)$)

\end{ex}

\begin{fact}
Let $\phi : R \rightarrow S$ be a morphism and let $s \in S$.  There exists a unique morphism $\phi_s : R[T] \rightarrow S$ such that $\phi_s(r) = \phi(r)$ for all $r \in R$ and $\phi_s(T) = s$.
\end{fact}

\begin{proof}
If $\phi_s$ is any such morphism then $\phi_s(a_nT^n + ... + a_0)$ must equal $\phi(a_n)s^n + ... + \phi(a_0)$.  This proves uniqueness and existence (upon checking that this is a morphism).
\end{proof}

\begin{ex} \hspace{0.5cm}

\begin{itemize}

\item If $\phi = id : R \rightarrow R$ then we get evaluation morphism $R[T] \rightarrow R$ given by $P \mapsto P(s)$.

\item Let $I \subseteq R$ be an ideal and let $\phi : R \rightarrow R/I \xhookrightarrow{} R/I[T]$ and let $s = T$.  We get ``reduction mod $I$" morphism $R[T]\rightarrow R/I[T]$.

\end{itemize}

\end{ex}

\begin{rmk} (def. 1.5)
$a \in R$ is \textbf{nilpotent} if $a^n = 0$ for some $n \in \NN$
\end{rmk}

\begin{prop}
Let $P = a_nT^n + ... + a_0 \in R[T]$.  We have $P \in R[T]^\times$ iff $a_0 \in R^\times$ and $a_1,...,a_n$ are nilpotent.
\end{prop}

\begin{proof} \hspace{0.5cm}

Assume that $R$ is a domain.  We have that $P$ is a unit iff there exists $Q \in R[T]$ such that $PQ = 1$.  By 1-11 Fact 6, $0 = deg(1) = deg(PQ) = deg(P) + deg(Q)$ ($R$ is a domain so the leading coefficient of $P$ (alternatively $Q$) is not a zero divisor).  Thus $deg(P),deg(Q) = 0$.  Thus $a_1,...,a_n = 0$ are nilpotent and $a_0 \in R^\times$.

Let $R$ be a general ring.  Let $\mathcal{P} \subseteq R$ be a prime ideal.  Since $P$ is a unit in $R[T]$, the image of $P$ in $R / \mathcal{P}[T]$ is a unit.  Since $R/\mathcal{P}$ is a domain by 1-6 thm. 8, by the above argument $a_1,...,a_n = 0_{R/\mathcal{P}}$ and thus $a_1,...,a_n \in \mathcal{P}$.  Since this holds for all $\mathcal{P}$, by HW we have that $a_1,...,a_n$ are nilpotent.

\end{proof}

\begin{lemma}\label{lemma:polyroot}
Let $P \in R[T]$ and $r \in R$.  We have $P(r) = 0$ iff $(T-r) \mid P$.
\end{lemma}

\begin{proof}

The backward direction is clear.  Apply fact 1 with $S = R[T]$, $\phi : R \xhookrightarrow{} R[T]$, $s = T+r$ to get a morphism $R[T] \rightarrow R[T]$.  This is an isomorphism with inverse given by the same construction with $s = T-r$.  Under this isomorphism, $P \mapsto Q$ with $Q(0) = 0$.  Thus $Q(T) = b_nT^n + ... + b_1T$ so $T \mid Q$.  Taking the preimage under the above isomorphism, we have $(T-r) \mid P$.

Gitlin's thoughts:  The constructed isomorphism can be thought of as the map $R[T] \rightarrow R[T]$ which ``replaces every $T$ with $T+r$."  Thus $Q(x) = P(x+r)$ for all $x \in R$.  In particular, $Q(0) = P(r)$ which is $0$.  The inverse map is the map $R[T] \rightarrow R[T]$ which ``replaces every $T$ with $T-r$."  In particular, the preimage of $Q(T) = b_nT^n + ... + b_1T$ is $b_n(T-r)^n + ... + b_1(T-r)$.

\end{proof}

\begin{prop}
Let $P,D \in R[T]$.  Assume that $D \neq 0$ and that the leading coefficient of $D$ is a unit.  There exist unique $Q,Z \in R[T]$ with $deg(Z) < deg(D)$ such that $P = QD+Z$.
\end{prop}

\begin{proof}\hspace{0.5cm}

Choose $Q$ so that $deg(Z)$ is minimal where $Z = P-QD$.  We claim $deg(Z) < deg(D)$.  Suppose not.  Let $D = d_nT^n + ... + d_0$ and $Z = z_mT^m + ... + z_0$ with $m \geq n$.  Note that $P-(Q+z_md_n^{-1}T^{m-n})D = Z - (z_md_n^{-1}T^{m-n})D$ has degree less than $deg(Z)$, contradicting the minimality of $Z$.  This shows existence.

Gitlin's thoughts:  The set of ``candidates" is the set of elements of $R[T]$ that have the form $P-*D$ where $*$ varies over $R[T]$.  Clearly $P-(Q+z_md_n^{-1}T^{m-n})D$ is a candidate.  Furthermore, the leading term $z_mT^m$ of $Z$ cancels with the leading term $z_md_n^{-1}T^{m-n} \cdot d_nT^n = z_mT^m$ of $(z_md_n^{-1}T^{m-n})D$ in the subtraction $Z - (z_md_n^{-1}T^{m-n})D$ so the degree of $Z$ is at least one more than the degree of $Z - (z_md_n^{-1}T^{m-n})D$.

Let $Q',Z'$ be another such pair.  We have $QD+Z = P = Q'D+Z'$ so $(Q-Q')D = Z'-Z$.  Thus (1-11 fact 6) $deg(D) > \max(deg(Z'),deg(Z)) \geq deg(Z'-Z) = deg((Q-Q')D) = deg(Q-Q') + deg(D)$ (the leading coefficient of $D$ is a unit and thus not a divisor of zero).  This means $deg(Q-Q') = - \infty$ so $Q-Q' = 0$ so $Q= Q'$.  Thus $Z = P-QD = P-Q'D = Z'$.  This shows uniqueness.

Gitlin's thoughts:  My uniqueness proof likely differs from the one given in class.  Sorry Tasho.  I couldn't follow your inequalities.

\end{proof}

\section{January 18, 2017}
\begin{lemma}
(This is Lemma 3 from Jan. 13) Given $P \in R[T], r \in R$, then $P(r) = 0 \Leftrightarrow (T - r) \mid P$.
\end{lemma}

\begin{prop}
(This is Proposition 4 from Jan. 13) Given $P, D \in R[T]$ such that $D$ has a unit as a leading coefficient, there exists unique $Q, Z \in R[T]$ with $deg(Z) < deg(D)$ such that $P = DQ + Z$.
\end{prop}

\begin{proof}
This proof rehashes the one given in class by Tasho to rectify any mistakes or lack of clarity in the one given on Jan. 13 (no offense Gitlin). Following the proof given on Jan. 13 (before the proof of uniqueness), we have the following string of inequalities:
\[deg(D) > deg(Z) \geq deg(Z - Z') = deg(D) + deg(Q - Q').\]
This implies that $deg(Q - Q') = -\infty = deg(Z - Z')$, so $Q - Q' = Z - Z' = 0$, i.e. $Q = Q'$ and $Z = Z'$, completing the proof.
\end{proof}

The following is a proof Tasho gave of Lemma 3 from Jan. 13 which utilizes Proposition 4 from Jan. 13.
\begin{proof}
As per Proposition 4, write $P(T) = (T - r) \cdot Q(T) + Z(T)$ with $deg(Z) < deg(T - r) = 1 \Rightarrow Z \in R$. We have then $P(r) = (r - r) \cdot Q(r) + Z(r) = Z$, but since $P(r) = 0$ we have now $Z = 0 \Leftrightarrow (T - r) \mid P$.
\end{proof}

\begin{defn}
Let $P \in R[T], r \in R$. Define \textbf{the multiplicity of $r$ in $P$} by $\max\{n \in \NN \colon (T - r)^n \mid P\}$.
\end{defn}

\begin{cor}
Let $R$ be a domain. Then $0 \neq P \in R[T]$ has at most $deg(P)$-many zeroes, counted with multiplicity.
\end{cor}
\begin{proof}
We induct on $deg(P)$. As a base case take $deg(P) = 0$: $P \in R$ has no zeroes, as required. For an inductive step assume that the statement holds for all polynomials of degree less than $deg(P)$. If $P$ has no zeroes, we are done. Otherwise let $r \in R$ be a zero of $P$. By Lemma 3 from Jan. 13, $P = (T - r) \cdot Q(T)$ for some $Q \in R[T]$ with $deg(Q) < deg(P)$. By inductive hypothesis $Q$ has at most $deg(Q) = deg(P) - 1$ zeroes. Since $R$ is a domain, a zero of $P$ is either $r$ or a zero of $Q$, so $P$ has at most $deg(Q) + 1 = deg(P)$ zeroes, as required. This completes the inductive step, and thus the proof.
\end{proof}
The following example answers the question ``does this fail when $R$ is not a domain?":
\begin{ex}
If $R = \ZZ / 6\ZZ$ and we define $P(T) = 3T$, then $\#\{ \text{zeroes of } P \} = \#\{0, 2\} > 1 = deg(P)$.
\end{ex}

\begin{defn}
A field $k$ is called \textbf{algebraically closed} provided that every nonconstant $P \in k[T]$ has a zero.
\end{defn}

\begin{prop}
Let $k$ be an algebraically closed field, $P \in k[T]$. Then
\[P(T) = c \cdot (T - r_1)^{m_1} \cdots (T - r_n)^{m_n}\]
for $c, r_1, ..., r_n \in k, m_1, ..., m_n \in \NN$.
\end{prop}

\begin{proof}
This is proved on Homework 2. (Future readers: feel free to input the proof here when completed.)
\end{proof}

\begin{defn}
Define $\mathbf{R[T_1, ..., T_n]}$ recursively as $(R[T_1, ..., T_{n-1}])[T_n]$. More concretely: a \textbf{monomial} in $T_1, ..., T_n$ is an expression $T_1^{m_1} \cdots T_n^{m_n}$, for $m_i \in \NN$. A polynomial in $T_1, ..., T_n$ with coefficients in $R$ is an $R$-linear combination of monomials; the set of all polynomials in $T_1, ..., T_n$ is $R[T_1, ..., T_n]$.
\end{defn}

Recall: $P \in R[T] \leftrightarrow$ an eventually-zero sequence $(a_0, a_1, ...) \leftrightarrow$ a map $\NN \rightarrow R$ with finite support. So $P \in R[T_1, ..., T_n] \leftrightarrow$ a map $\NN^n \rightarrow R$ with finite support, so we may think of a polynomial $P \in R[T_1, ..., T_n]$ as
\[P(T) = \sum_{c_i \in \NN} a_{c_1, ..., c_n} \cdot T_1^{c_1} \cdots T_n^{c_n}.\]

\begin{fact}
Given a ring morphism $\phi \colon R \rightarrow S, s_1, ..., s_n \in S$, there exists a unique ring morphism \[\phi_{s_1, ..., s_n} \colon R[T_1, ..., T_n] \rightarrow S\] such that $\phi_{s_1, ..., s_n}(r) = \phi(r)$ for $r \in R$, and $\phi_{s_1, ..., s_n}(T_i) = s_i$.
\end{fact}
\begin{proof}
We induct on $n$. As a base case let $s \in S$. Fact 5.1, Jan. 13 above guarantees the existence of a unique morphism $\phi_s \colon R[T] \rightarrow S$ with the given properties. For an inductive case assume the statement holds for $n - 1$. By inductive hypothesis we have a unique morphism $\phi_{s_1, ..., s_{n-1}} \colon R[T] \rightarrow S$ satisfying the given properties. Applying the base case to $\phi_{s_1, ..., s_{n-1}}$ with $s = s_n$ gives the desired (unique) morphism, completing the inductive step and thus the proof.
\end{proof}

\begin{defn} \hspace{0.5cm}
\begin{itemize}
\item The image of $\phi_{s_1, ..., s_n}$ is called the \textbf{$R$-subalgebra of $S$ generated by $s_1, ..., s_n$}
\item $s_1, ..., s_n$ are called \textbf{algebraically independent} provided that $\phi_{s_1, ..., s_n}$ is injective.
\end{itemize}
\end{defn}

To gain some intuition about algebraic independence, consider the algebraic relation
\[r_1s_1^{m_{1,1}} \cdots s_n^{m_{1, n}} + \cdots + r_ns_1^{m_{n, 1}} \cdots s_n^{m_{n, n}} = 0.\]
Taking preimages under $\phi_{s_1, ..., s_n}$ we have
\begin{align*}
\phi_{s_1, ..., s_n}^{-1}(r_1s_1^{m_{1,1}} \cdots + s_n^{m_{1, n}} + \cdots + r_ns_1^{m_{n, 1}} \cdots s_n^{m_{n, n}}) &\in \phi_{s_1, ..., s_n}^{-1}(0) = \ker(\phi_{s_1, ..., s_n}) \\
\Rightarrow \phi^{-1}(r_1)T_1^{m_{1,1}} \cdots T_n^{m_{1,n}} + \cdots + \phi^{-1}(r_n)T_1^{m_{n,1}} \cdots T_n^{m_{n,n}} &\in \ker(\phi_{s_1, ..., s_n})
\end{align*}
thus if there exists such a relation that is nontrivial, there is a nonzero element in $\ker(\phi_{s_1, ..., s_n})$, so it is not injective. (Note from Ben: this seems to assume that the coefficients $r_i$ are in the image of $\phi$, which is not generally true. Will bring up with Tasho and edit as necessary.)

\begin{defn}
Let $R$ be a ring, $0 \neq a \in R$ not a unit. \hspace{0.5cm}
\begin{itemize}
\item $a$ is called \textbf{irreducible} provided that $a = bc \Rightarrow b \in R^{\times}$ or $c \in R^{\times}$.
\item $a$ is called \textbf{prime} provided that $a = bc \Rightarrow a \mid b$ or $a \mid c$.
\end{itemize}
\end{defn}

\begin{fact}
$R$ a domain $\Rightarrow$ every prime element of $R$ is irreducible.
\end{fact}

\begin{proof}
Let $p \in R$ be prime and write $p = ab$. Then without loss of generality we have $p \mid a \Rightarrow a = pc$ for some $c \in R$, so we may rewrite $p = pcb$, so $p(1-bc) = 0$. Since $R$ is a domain and $p \neq 0$, we have $1 = bc$, so $c = b^{-1} \Leftrightarrow b \in R^{\times}$.
\end{proof}

\begin{rmk}
\hspace{0.5cm}
\begin{itemize}
\item If $R$ is not a domain, prime elements need not be irreducible (see Homework 2).
\item Even if $R$ is a domain, irreducible elements need not be prime (see Homework 2).
\end{itemize}
\end{rmk}

\begin{defn}
Let $R$ be a domain. \hspace{0.5cm}
\begin{enumerate}
\item $R$ \textbf{has factorization} provided that every $0 \neq r \in R$ can be written $\epsilon \cdot u_1 \cdots u_k$ with $\epsilon \in R^{\times}$ and $u_i \in R$ irreducible.
\item Write $a \sim b \Leftrightarrow a = \epsilon \cdot b, \epsilon \in R^{\times}$.
\item $R$ has \textbf{unique factorization} provided that $R$ has factorization and if $\epsilon \cdot u_1 \cdots u_k = \mu \cdot v_1 \cdots v_m$ then $k = m$ and there exists a permutation $\sigma \in S_m$ such that $u_i \sim v_{\sigma(i)}$.
\item Such an $R$ is called a \textbf{unique factorization domain}, or UFD.
\end{enumerate}
\end{defn}

\section{January 20, 2017}
\begin{prop} \label{Prop 1, Jan 20}
    Let $R$ be a domain with factorization. Then $R$ has unique factorization if and
    only if every irreducible element is prime.
\end{prop}
\begin{proof}
    Say $R$ has unique factorization. Let $p$ be irreducible. We will show that
    $p$ is prime. Let $a,b \in R$ be such that $p \mid ab$. Using factorization,
    we write that $a = \zeta u_1u_2 \cdots u_k$, $b = \mu v_1\cdots v_l$, and that
    $\frac{ab}{p} = \alpha w_1w_2 \cdots w_m$. \\
    From the fact that $\frac{ab}{p}\cdot p = ab$, we have that $\alpha p w_1w_2\cdots w_m = \zeta\mu u_1\cdots u_kv_1 \cdots v_l$.
    By uniqueness of factorization, we have that these expressions are permutations
    upto multiplication by units, so we have that $p \sim u_i$ or $p \sim v_j$. In
    the first case, we have that $p \mid a$ and in the second, we have that $p \mid b$. \\
    Assume now that every irreducible is prime. We will show uniqueness by induction
    on the length of the factorization. \\
    Length = 0: Then, $a$ is a unit, so $a$ is not divisible by any irreducible. \\
    Assume now that the statement holds for a length $k$ factorization.
    Assume now that we have
    $$
    a = \zeta u_1 \cdots u_{k+1} = \mu v_1 \cdots v_l
    $$
    Now $u_{k+1}$ divides $v_1 \cdots v_l$. But since $u_{k+1}$ is prime, $u_{k+1} \mid v_j$
    for some $1 \leq j \leq l$. Reordering the factors, we have that
    $u_{k+1} \mid v_l$. Since $v_l$ is irreducible, this thus implies that if $u_{k+1}x = v_l$,
    $x$ must be a unit, and thus $v_l \sim u_{k+1}$. Furthermore, we can write that
    $\zeta u_1 \cdots u_k = \mu \left(\frac{v_l}{u_{k+1}}\right) v_1 \cdots v_{l-1}$. Applying
    the inductive hypothesis, we have $l -1 = k$ and the the first $k$ terms
    are simply a permutation of each other, while the last is simply a unit scaling
    away from the other. Thus, these are the same factorization.
\end{proof}
\begin{defn} \label{Defn 2, Jan 20}
    A ring is called \textbf{principal}, if every ideal is principal. If,
    in addition, the ring is a domain, it is a \textbf{Prinicipal Ideal Domain}, or
    \textbf{PID}.
\end{defn}
\begin{ex}
    $\ZZ$ is a PID. $\Zn n$ are principal, any field is a PID. Quotients and products
    of principal rings are principals.
\end{ex}
\begin{thm} \label{Thm 3, Jan 20}
    Every PID is a UFD
\end{thm}
\begin{proof}
    Let $R$ be a PID. We first show the existence of factorization. Say that
    $a \in R$, $a \neq 0$, $a \in R^\times$, and has no factorization. Then,
    $a$ is not irreducible, (as otherwise, it itself is a factorization). So,
    we can right that $a = a_0 = a_1b_1$. However, note that one of these must
    not have a factorization, as otherwise the product of their factorizations
    is a factorization of $R$. So, WoLOG assume that $a_1$ does not have a factorization.
    Again, $a_1$ is not irreducible, and we write $a_1 = a_2b_2$. Recursively
    applying this argument, we have that $a_0 = a_1a_2a_3 \dots$ an infinite sequence with
    $a_i \neq 0$, $a_i \notin R^\times$ and $a_i \not\sim a_{i+1}$, and $a_{i+1} \mid a_{i}$
    In terms of the ideals, we have that
    $$
    (a_0) \subsetneq (a_1) \subsetneq \dots
    $$
    Now, we consider $I = \bigcup\limits_{i=0}^{\infty}{(a_i)}$. Since we have a
    nested sequence of ideals, we have that $I$ itself is an ideal. We have that $R$
    is a principal ideal, which means that $I = (c)$ for some $c \in R$. This means
    though that $c \in I$, which means that $c \in (a_k)$ for some $k \in \NN$.
    However, this means that for any $n \geq k$, we have that $(c) \subset (a_k) \subset I = (c)$,
    meaning that each $(a_n) = (c)$. However, we assumed proper inclusion above,
    and this is thus a contradiction. Thus, we have factorization. \\
    Next, we verify uniqueness. We can show this by showing that every irreducible is
    prime. Let $u$ be an irreducible. We want to show it is prime, so assume that $u \mid ab$.
    Assume that $u \nmid a$. We will show that $u \mid b$. Since we have that $u \nmid a$,
    we have that $u \notin (a)$, so we have that $(a) \subsetneq (a, u)$. Since
    $R$ is principal, we have that $(u, a) = (d)$. Since $u \in (d)$ we have that $d \mid u$.
    However, since $u$ is irreducible, $d$ must be a unit, or $d \sim u$. We know because
    $(u) \subsetneq (u,a)$, we have that $d \sim u$ is impossible. Thus, assume that
    $d$ is a unit. This means though that $(u,a)  = R$. This means that we have $1 \in R$,
    so $1 = \alpha u + \beta a$ (this is what it means to be in $(u,a)$). Scaling
    this equality by $b$, we have that $b = b\alpha u + \beta (ab)$. Now, since $u \mid ab$
    and $u \mid u$, $u \mid b$, as desired.
\end{proof}
\begin{defn} \label{Defn 4, Jan 20}
    A \textbf{Euclidean Domain} is a pair $(R, H)$, where $R$ is a domain, and
    $H: R \setminus \{0\} \rightarrow \NN$ is a function such that
    \begin{enumerate}
        \item $H(ab) \geq H(a)$
        \item Given $X,d \in R$ with $d \neq 0$, there are $q, r \in R$ such that
        \begin{enumerate}
            \item $X = qd + r$
            \item either $r = 0$ or $H(r) < H(d)$
        \end{enumerate}
    \end{enumerate}
\end{defn}
\begin{ex}
    If $F$ is a field, then $(F[T], deg)$ is a Euclidean domain.
\end{ex}
\begin{prop} \label{Prop 5, Jan 20}
    Every Euclidean domain is a PID.
\end{prop}
\begin{proof}
    Let $I \subset R$ be an ideal. WoLOG, $I \neq \{0\}$. Choose $d \in I$ such
    that $H(d)$ is minimal. To show that $I = (d)$, let $a \in I$ and take $q,r \in R$
    such that $a = qd + r$. Then, $r = a - qd \in I$. If $r \neq 0$, we have that $H(r) < H(d)$,
    contradicting the minimality of $d$. Thus $r = 0$, and $a = qd$. Thus,
    every ideal is of the form $(d)$, showing it is principal.
\end{proof}
\begin{cor} \label{Cor 6, Jan 20}
    If $F$ is a field, $F[T]$ is a PID, and hence a UFD.
\end{cor}
\begin{defn} \label{Defn 7, Jan 20}
    Let $R$ be a UFD. We call $0 \neq P \in R[T]$ \textbf{primitive} if $a \in R$
    with $a \mid P$ means that $a \in R^\times$.
\end{defn}
\begin{rmk}
    This is equivalent to no irreducible $p \in R$ divides all the coefficients of
    $P$. Intuitively, we can say the ``gcd'' of the coefficients of $P$ is 1.
\end{rmk}
\begin{lemma} \label{Lemma 8, Jan 20}
    \textbf{(Gauss)} If $P,Q$ are primitive, so is $PQ$.
\end{lemma}
\begin{proof}
    Let $P \neq 0$, $Q \neq 0$ such that $PQ$ is not primitive. Let $p \in R$ prime
    such that $p \mid PQ$. Thus, $PQ$ becomes $0$ under the
    $R[T] \rightarrow (R/(p))[T]$. \\
    But $R / (p)$ is a domain by \ref{Thm 8, Jan 6}, and thus, by \ref{Fact 6, Jan 11}, we have
    that $(R/(p))[T]$ is also a domain. Thus, we have that either $P$ or $Q$ is
    0 in $(R / (p))[T]$, meaning it is also divisible $p$, as desired.
\end{proof}

\section{January 23, 2017}
\noindent Recall: $P\in R[T]$ is primitive if $a|P\Longrightarrow a\in R^{\times}$.
Lemma \ref{Lemma 8, Jan 20} (Gauss): $P,Q$ primitive $\Longrightarrow P\cdot Q$ primitive.

\noindent $R$ is a UFD, $F$ is it's fraction field
\begin{lemma} \label{Lemma 1, Jan 23}
    Let $P,Q\in R[T], P$ primitive. If $Q=a\cdot P, a\in F$, then $a\in R$.
\end{lemma}
\begin{proof}
    Write $a={n\over d}$ with $n,d\in R$. Decompose $n=\epsilon n_1,\ldots, n_k,d=\mu v_1,\ldots,v_k$ into irreducibles. We may assume $n_i\sim v_j$ for $i,j$. Then $\mu v_1,\ldots, v_kQ=\epsilon   n_1\ldots n_k P$. So $v_1|n_1\ldots n_ka_i$ for all $i$, where $P(T)=a_nT^n+\ldots+a_0$. Since $v_1$ is prime, and $n_i$ are irreducible, and $v_1\nsim u_1,\ldots, u_k$, so $v_1|a_i$ for all $i$. This contradicts primitivity of $P$.
\end{proof}
\begin{thm} \label{Thm 2, Jan 23}
    The ring $R[T]$ is a UFD and its irreducible elements are
    (1) $p\in R$ irreducible
    (2) $P\in R[T]$ primitive, and irreducible in $F[T]$.
\end{thm}
\begin{proof}
    Step 1: The above elements are irreducible.
    Given $p\in R$ irreducible, write $p=PQ,$ with $P,Q\in R[T]$. Since $R$ is a domain, we have $0=\text{deg}(P)=\text{deg}(P)+\text{deg}(Q)$, so $P,Q\in R$. But then either $P\in R^{\times}$ or $Q\in R^{\times}.$
    Let $P\in R[T]$ be primitive, irreducible in $F[T]$. Write $P=QS$ with $Q,S\in R[T].$ Since $P$ is irreducible in $F[T],$ either $Q$ or $S$ lies in $F[T]^{\times}=F^{\times}$ by (\ref{Prop 2, Jan 13}). Say WOLOG $S=R[T]\cap F^{\times}=R\setminus[0].$ Then $S^{-1}P=Q$. By \ref{Lemma 1, Jan 23}, $S^{-1}\in R.$ So $S\in R^{\times}.$
    Step 2: Every element of $R[T]$ has a decompose with factors as in 1), 2).
    Take $P\in R[T].$ Decompose $P$ as an element of $F[T]$. $P=c\cdot \tilde{Q}_1,\ldots,\tilde{Q}_n, c\in F^{\times}, \tilde{Q}_i\in F[T]$ irreducible. By \ref{Lemma 3, Jan 23}, write $\tilde{Q}_I=c_i\cdot Q_i$ with $c_i\in F^{\times}, Q_i\in R[T]$ primitive. Thus $P=c\cdot c_1,\ldots c_n\cdot Q_1,\ldots Q_n$. By Gauss Lemma (\ref{Lemma 8, Jan 20}), $Q_1,\ldots Q_n$ is primitive, so by \ref{Lemma 1, Jan 23}, $a\in R$. Factor $a=\epsilon n_1,\ldots n_k$ in $R$.

    Remark: This shows in particular, that (1)+(2) are all the irreducible elements in $R[T]$.
    Uniqueness of factorization
    Let $\epsilon n_1,\ldots, n_k P_1,\ldots, P_n= \mu v_1\ldots v_l Q_1-Q_m$ with $u_i,v_j$ as in (1) and $P_i,Q_j$ as in (2). Uniqueness in $F[T]$ tells us $n>m$, and after reordering, $P_i=c_iQ_i$ with $c_i\in F^{\times}$. Applying \ref{Lemma 1, Jan 23} to $P_i=cP_iQ_i$ and $c_i^{-1}P_i=Q_i$ to see $c_i\in R^{\times}$. Thus $P_i\sim Q_i$ is $R[T]$. Then $\epsilon n_1,\ldots, n_k=\mu {Q_1\ldots Q_n\over P_1\ldots P_n}\cdot v_1\ldots _l$ and uniqueness in $R$ gives us $k=l$ and $n_i\sim v_i$ after reordering.
\end{proof}

\begin{lemma}\label{Lemma 3, Jan 23}
    Let $0\ne P\in F[T].$ There exists $c\in F^{\times}$ such that $cP\in R[T]$ primitive.
\end{lemma}
\begin{proof}
There is $a\in R$ such that $aP\in R[T].$ Let $d$ be a gcd of all coefficients of $aP$. Then $d|ap$ and $ad^{-1}P\in R[T]$ primitive.
\end{proof}
\begin{lemma} \label{Lemma 4, Jan 23}
Let $P,Q\in R[T],P$ primitive. Then $P/Q$ in $R[T]\iff P/Q\in F[T]$.
\end{lemma}
\begin{proof}
$\Longrightarrow$: trivial.
$\Longleftarrow$: Let $Q=P\tilde{S}$ with $\tilde{S}\in F[T]$. By \ref{Lemma 3, Jan 23}, $\tilde{S}=aS,a\in F^{\times}$. $S\in R[T]$ primitive. So $Q=aPS$. By Gauss lemma (\ref{Lemma 8, Jan 20}), $PS$ is primitive. By \ref{Lemma 1, Jan 23}, $a\in R$, then $\tilde{S}=aS\in R[T].$
\end{proof}

\section{January 25, 2017}

\begin{prop} \label{Prop 1, Jan 25}
(Eisenstein Criterion) Let $R$ be a UFD, $F$ its fraction field (as before). Let $P(T)=a_nT^n+\ldots+a_0\in R[T]$. If $p\in R$ prime such that $p$ does not divide $a_n$, then $p|a_{n-1},\ldots,a_0$ and $p^2$ does not divide $p_0$ then $P$ is irreducible in $F[T]$.
\end{prop}
\begin{proof}
Suppose not. By Thm 2 last time we have that $P$ is not irreducible in $R[T]\Longrightarrow P=Q\cdot S$ where $Q,S\in R[T].$ Let $\bar{P},\bar{Q},\bar{S}$ be images in $(R/(P))[T]$. Write $Q(T)=b_mT^m+\ldots+b_0, S(T)=c_jT^j+\ldots+c_0$. Then $\bar{P}(T)(=\bar{a_n}T^n)=Q(T)(=\bar{b_k}T^k)\cdot \bar{S}(T)(=\bar{c_j}T^j)$ with $k+j=n$ since $R/(P)$ is a domain since $P$ is prime). Then $p|b_0,p|c_0\Longrightarrow p^2|a_0=b_0c_0$ which is a contradiction.
\end{proof}
\noindent \textbf{Cyclotomic Polynomials:}
\begin{defn} \label{Defn 2, Jan 25}
For $n\in \NN$, the \textbf{\textit{n}th cyclotomic polynomial} is $\Phi_n(T)=\Pi_{gcd(k,n)=1,k=1}^n(T-e^{2\pi ik/n})$.
\end{defn}
\begin{prop} \label{Prop 3, Jan 25}
$\Phi$ is monic and has integer coefficients.
\end{prop}
\begin{proof}
Induction on $n\in \NN$.

$n=1: T-1,n=2: T+1$

$k<n\Longrightarrow k=n:T^n-1=\Pi_{k=1}^n(T-e^{2\pi ik/n})=$
$\Pi_{d|n}\Phi_d(T)$. By induction, $\Phi_d\in \ZZ[T]$ and monic for $d|n, d\ne n$. In particular, these $\Phi_d$ are primitive, so Gauss lemma (\ref{Lemma 8, Jan 20}) implies that $\Pi_{d|n,d\ne n}\Phi_d=:P$ is primitive. $T^n-1=\Phi_n(T)\cdot P(T)$ in $\CC[T]$. So $P|T^n-1$ in $\CC[T]\Longrightarrow P|T^n-1$ in $\QQ[T]$. By \ref{Lemma 4, Jan 23}, $P|T^n-1$ in $\ZZ[T]$.
\end{proof}
\begin{prop} \label{Prop 4, Jan 25}
If $p\in \NN$ prime, then $\Phi_p$ is irreducible.\footnote{We see in Galois theory that each $\Phi_n$ is in fact irreducible, but this is harder. See \ref{Prop 3, Feb 24}.}
\end{prop}
\begin{proof}
$T^p-1=\Phi_1\cdot \Phi_p=(T-1)\cdot\Phi_p$ and $T^p-1=(T-1)(T^{p-1}+T^{p-2}+\ldots+1)\Longrightarrow \Phi_p=T^{p-1}+\ldots+1.$ Reduce mod $p$: $(T-1)\bar{\Phi_p}(T)=T^p-1=(T-1)^p$ implies that $\bar{\Phi_p}(T)=(T-1)^{p-1}$. Consider the isomorphism $\ZZ[T]\longrightarrow \ZZ[T]$ defined by $T\longrightarrow T+1$. Let $Q$ be image of $\Phi_p$. Then $\Phi_p$ irreducible if and only if $Q$ is irreducible. But $\bar{Q}(T)=\bar{\Phi}_p(T+1)=T^{p-1}$. Thus if $Q(T)=a_{p-1}T^{p-1}+\ldots+a_0$, $p$ does not divide $a_{p-1},p|a_{p-2},\ldots,a_0$. But $a_0=Q(0)=\Phi_p(1)=p$. Apply Eisenstein (\ref{Prop 1, Jan 25}).
\end{proof}
Adjoining Elements:

Let $R$ be a ring. We want to add a new element $s$, subject to some relation $a_ns^n+\ldots+a_n=0$ for $a_i\in R$. More precisely, we want a ring $S$, a morphism $R\longrightarrow S$, an element $s\in S$ satisfying the relation, such that $(T,t)$ another such pair then there exists a unique morphism $S\longrightarrow T,s\longrightarrow t$.

Set $P\in R[T], P(T)=a_nT^n+\ldots+a_0, S=R[T]/(P),$ $s=$ image of $T$ in $S$. Note that $P(S)\in S$, if we think of $P\in (R[T])[T]$ (with ``constant'' coefficients), $P(T)=P.$
\begin{defn}\label{Defn 5, Jan 25}
Let $R$ be ring.

(i) An \textbf{\textit{R}-algebra} is a pair $(S,\phi), S$ ring, $\phi:R\longrightarrow S$ morphism ($\phi$ usually suppressed, so $\phi(r)s=``rs"$).

(ii) A \textbf{morphism of \textit{R}-algebras} $(S,\phi)\longrightarrow (T,\psi)$ is a ring morphism $f:S\longrightarrow T$ such that $f(\phi(r)s)=f(rs)=rf(s)=\psi(r)f(s)$.
\end{defn}
\begin{rmk}
$\psi$ need not be injective.
\end{rmk}
\begin{ex}
$R[T]$ is an $R$-algebra, $R/I$ is an $R$-algebra, any ring is a $\ZZ$-algebra in a canonical way.
\end{ex}

\documentclass{amsart}
\usepackage{amsfonts}
\usepackage{amsmath}
\usepackage{amsthm}
\usepackage{amssymb}
\usepackage{mathtools}
\usepackage{enumerate}
\usepackage{graphicx}
\usepackage{dsfont}
\usepackage{bbm}
\usepackage{tikz}
\usepackage{tikz-cd}
\usepackage{xifthen}
\usepackage{cancel}
\usepackage{hyperref}

\setlength{\textwidth}{\paperwidth}
\addtolength{\textwidth}{-2in}
\calclayout

\newcommand{\NN}{\mathbb{N}}
\newcommand{\RR}{\mathbb{R}}
\newcommand{\HH}{\mathbb{H}}
\newcommand{\QQ}{\mathbb{Q}}
\newcommand{\ZZ}{\mathbb{Z}}
\newcommand{\Zn}[1]{\mathbb{Z} / #1 \mathbb{Z}}
\newcommand{\CC}{\mathbb{C}}
\newcommand{\FF}{\mathbb{F}}
\newcommand{\PP}{\mathbb{P}}
\newcommand{\dd}[2][ ]{\frac{\partial #1}{\partial #2}} % Partial derivative
\newcommand{\dsq}[3][ ]{\frac{\partial^2 #1}            % Second partial
{\ifthenelse { \equal {#2} {#3} }{\partial #2^2}{\partial #2 \partial #3}}}
\newcommand{\defarr}                        %definition iff arrow
{\overset{\textrm{def}}{\Longleftrightarrow}}
\newcommand{\defeq}                         %definition equality sign
{\overset{\textrm{def}}{=}}
\newcommand{\argeq}[1]                      %definition equality sign
{\overset{\textrm{#1}}{=}}

\DeclareMathOperator{\lcm}{lcm}
\DeclareMathOperator{\orb}{orb}
\DeclareMathOperator{\im}{im}
\DeclareMathOperator{\supp}{supp}
\DeclareMathOperator{\stab}{Stab}
\DeclareMathOperator{\sgn}{sign}
\DeclareMathOperator{\spn}{span}
\DeclareMathOperator{\tr}{tr}

\newtheorem{thm}{Theorem}[section]
\newtheorem{lemma}[thm]{Lemma}
\newtheorem*{unnumlemma}{Lemma}
\newtheorem{fact}[thm]{Fact}
\newtheorem{prop}[thm]{Propostion}
\newtheorem*{claim}{Claim}
\newtheorem{cor}{Corollary}

\theoremstyle{definition}
\newtheorem{defn}[thm]{Definition}
\theoremstyle{remark}
\newtheorem*{rmk}{Remark}
\newtheorem*{ex}{Example}
%==============================================================================
% MATH 494 Collaborative Notes
% This is a collaborative notesheet that consists of notes from each day in Math
% 494, transcribed into Tex format for convenience and security from the
% Notebook Thief. General format:
% Contribute by adding a new section, titled "Month Day, Year"
% This is done in the format \section{Date}
% Tex notes using the numbering system used by Tasho himself. I've set up the
% counters on the theorems, definitions, and "facts" to run in tandem with the
% section that the content is in.
% Make sure to properly wrap all proofs, facts, and definitions using the above
% environments.
% Enjoy!
% Current Collaborators:
% 1. Pranav
% 2. Nick
% 3. Andrew
% 4. Ben
%==============================================================================
\begin{document}
\title{Math 494: Honors Algebra II}
\maketitle
\tableofcontents

\section{Jan. 27, 2017}

\begin{defn}
\hspace{0.5cm}
\begin{enumerate}
\item[(i)] A \textit{morphism of fields} is a morphism of rings whose source and target are fields
\item[(ii)] A \textit{subfield} is a subring of a field that is a field
\item[(iii)] An \textit{extension} of a field $k$ is a field $L$ containing $k$, denoted $L/k$.
\end{enumerate}
\end{defn}
 
\begin{rmk}
If $L/k$ is an extension, then $L$ is a $k$-algebra, and in particular a $k$-vector space. An extension $L/k$ is called \textit{finite} if $\dim_k(L) < \infty$. In this case define the \textit{degree} of $L/k$, denoted $[L : k] \stackrel{\text{def}}{=} \dim_k(L)$.\footnote{Debacker noted that the use of this notation, identical to the index of a subgroup, is because field extensions are in fact group-theoretic. When Galois groups are added to the notes, this will be discussed below.}
 \end{rmk}
 
\begin{fact}
 If $k_1, k_2$ are subgields of $L$, $k_1 \cap k_2$ is a subfield.
\end{fact}

\begin{defn}
\hspace{0.5cm}
\begin{enumerate}
\item[(i)] Every field has a smallest field, called the \textit{prime subfield}, equal to $\bigcap_{k \subset L} k^{\text{field}}$, the intersection of all subfields of $L$. \\
\textit{Example:} The prime subfield of $\RR$ is $\QQ$. To see this, observe: any subfield of $\RR$ must contain 1 and 0, thus sums of the form $1 + \cdots + 1$, so it must contain $\NN$. Throwing in additive and then multiplicative inverses gives $\QQ$.
\item[(ii)] Given an extension $L/k$ and any subset $S \subset L$, let $k(S)$ be the smallest subfield of $L$ containing $k$ and $S$. 
\end{enumerate}
\end{defn}

\begin{prop}
Let $L$ be a ring and $k \subset L$ a subfield. If $L$ is a domain and $\dim_k(L) < \infty$ then $L$ is a field.
\end{prop}
\begin{proof}
Let $0 \neq a \in L$. The map $a \cdot \colon L \rightarrow L$ that sends $x$ to $ax$ is $k$-linear, and since $L$ is a domain, injective: if $ax = 0$ then since $a \neq 0$, $x = 0$. By rank-nullity, the dimension of the image of $a \cdot$ must be equal to that of the target, i.e. $a \cdot$ is surjective. Thus there exists $G \in L$ so that $aG = 1$, implying that, as desired, $L$ is a field.
\end{proof}

\begin{defn}
Let $L/k$ be an extension, $L \supset E, F \supset k$ intermediate fields, each finite over $k$. The \textit{composite of E and F} is 
\[E \cdot F = \left\{ \sum_{i = 1}^n e_i \cdot f_i \colon n \in \NN, e_i \in E, f_i \in F \right\}.\]
\end{defn}

\begin{prop}
$EF$ is a field extension of $k$. It is finite and $[EF : k] \leq [E : k] \cdot [F : k]$.
\end{prop}
\begin{proof}
It is clear that $EF$ is a subring of $L$ containing $k$. Let $\{e_1, ..., e_n\}$ and $\{f_1, ..., f_m\}$ be, respectively, bases for $E/k$ and $F/k$. Then the set $\{e_i \cdot f_j \colon i \in \{1, ..., n\}, j \in \{1, ..., m\}\}$ spans the $k$-vector space $EF$. Thus by elementary results, $\dim_k(EF) \leq \#\{e_i \cdot f_j\} = [E : k] \cdot [F : k]$. In particular, $[EF : k] < \infty$. Applying \textbf{Proposition 1.4} above, we have that $EF$ is a field.
\end{proof}

\begin{prop}
Let $L/k$ be a finite extension, $V$ a finite dimensional $L$-vector space. Then $V$ is finite dimensional as a $k$-vector space, with $\dim_k(V) = \dim_L(V) \cdot [L : k]$.
\end{prop}
\begin{proof}
Let $\{v_1, ..., v_n\}$ be a basis for $V$ over $L$, $\{\ell_1, ..., \ell_m\}$ be a basis for $L$ over $k$. We will show $\{\ell_i \cdot v_j \colon i \in \{1, ..., n\}, j \in \{1, ..., j\}\}$ is a basis for $V$ over $k$:
\begin{enumerate}
\item[(i)] Spanning: given $x \in V$ we may write $x \in \lambda_1v_1 + \cdots + \lambda_nv_n$ for $\lambda_i \in L$, with each $\lambda_i = \mu_{i,1}\ell_1 + \cdots + \mu_{i,m}\ell_m$. Then $x = \sum_i \sum_j \mu_{i, j} \ell_j v_i$ with the $\mu_{i,j} \in k$.
\item[(ii)] Linear independence: Write $\sum \mu_{i,j} \ell_j v_i = 0$. Then $0 = \sum_i \left( \sum_j\mu_{i,j} \ell_j \right) v_i$. By linear independence of the $v_i$ we have $\sum_j\mu_{i,j} \ell_j = 0$, by linear independence of the $\ell_j$ we have $\mu_{i,j} = 0$ for all $i,j$.
\end{enumerate}
\end{proof}

\begin{cor}
If $M/L/K$ is a tower of finite ixtensions, then $[M : k] = [M : L] \cdot [L : k]$.
\end{cor}

\begin{defn}
Let $L/k$ be an extension, and $a \in L$. If there is $0 \neq p \in k[T]$ such that $p(a) = 0$, then $a$ is \textit{algebraic over $k$}. Otherwise, $a$ is \textit{transcendental over $k$}.
\end{defn}

As per the above definitions, we would like to study the following construction: let $\text{ev} \colon k[T] \rightarrow L$, $p \mapsto p(a)$ be the evaluation morphism. If $a$ is transcendental, ev is injective. Further, since $L$ is a field, we have that ev factors uniquely through the fraction field $k(T)$\footnote{This notation was established on homework: $k(T)$ is the set of rational functions with coefficients in $k$.} and induces an isomorphism $k(T) \stackrel{\sim}{\rightarrow} k(a)$. If $a$ is algebraic, let $0 \neq I \subset k[T]$ be the kernel of ev. Since $k[T]$ is a Euclidean domain, hence a PID (see a proposition from Jan. 20), so there exists a unique monic polynomial $p \in k[T]$ such that $(p) = I$.

\begin{defn}
$p$ is called the \textit{minimal polynomial} of $a$ over $k$ $a/k$, written $Ma/k$. $deg(a/k) := \deg(Ma/k)$ is its \textit{degree}.
\end{defn}

\begin{rmk}
$Ma/k$ is also the minimal polynomial of the endomorphism (linear operator that is not necessarily an isomorphism) $a \cdot \colon L \rightarrow L$.
\end{rmk}
\end{document}
\section{January 30, 2017}
\begin{prop} \label{Prop 1, Jan 30}
    Let $L / K$ be a field extension, $a \in L$ algebraic over $K$. Then,
    \begin{itemize}
        \item $M_{a,K} \in k[T]$ is irreducible
        \item $K[T] / (M_{a/K})$ is isomorphic to $K(a) \subset L$
        \item The images of $1, T, \dots T^{n-1}$ in $K[T] / (M_{a/K})$ are a
        basis of this $K$-vector space.
        \item $[K(a) : K] = \deg(a/K) = n$
    \end{itemize}
\end{prop}
\begin{proof}
    By definition of $M_{a/K}$ the evaluation $ev: k[T] \rightarrow L$ gives an
    injection from $K[T] / (M_{a/K}) \rightarrow K(a)$. Since $K(a)$ is a domain,
    so is $K[T] / (M_{a/K})$. By \ref{Thm 8, Jan 6}, this means $(M_{a/K})$ is prime.
    this means that $M_{a/K}$ is prime, so $M_{a/K}$ is irreducible by \ref{Fact 8, Jan 18} \\
    Now we show the images of $1, \dots T^{n-1}$ in $K[T] / (M_{a/K})$ generate the space.
    (Pranav Exclusive) It is enough to show that they generate $T^n$, as for any higher
    dimensional coset, by division algorithm we can reduce to a coset of $(M_{a/K})$ with
    degree less than $M(a/K)$, which is then definitionally generated by $1, \dots T^{n-1}$.
    If $M_{a,K}(T) = T^n + a_{n-1}T^{n-1} + \dots a_0$, Then $T^n = -a_{n-1}T^{n-1} \dots -a_0$
    They are also linearly independent. If not, let $b_{n-1}T^{n-1} + \dots b_0 = 0$
    be a non-trivial relation. Then $Q(T) = b_{n-1}T^{n-1} + \dots b_0 \in K[T]$ lies
    in $(M_{a,k})$ as it is in the kernel of the quotient map. We thus have that
    $M_{a,k} \mid Q$. This contradicts that $\deg(Q) < \deg(M_{a,k})$ \\
    Now we know that $K[T] / (M_{a,K})$ is a domain, containing $K$ and is finite
    dimensional over $K$, so by \ref{Prop 4, Jan 27}, $K[T] / (M_{a,k})$ is a field.
    This implies that $K[T] / (M_{a/K}) \rightarrow K(a)$ is surjective as the image
    in $K(a)$ is now a field that contains $K$ and $a$ and because $K(a)$ is the smallest
    such field.
\end{proof}

\begin{prop}\label{Prop 2, Jan 30}
    Let $L / K$ be an extension, $a \in L$. The following are equivalent.
    \begin{enumerate}[i)]
        \item $a$ is algebraic over $K$
        \item $[K(a) : K] < \infty$
        \item There is an intermediate field $L/E/K$, finite over $K$, containing $a$
    \end{enumerate}
\end{prop}
\begin{proof}
    i) $\Rightarrow$ ii): This comes from the above proposition (\ref{Prop 1, Jan 30}), final condition. \\
    ii) $\Rightarrow$ iii): Just let $E$ be $K(a)$. \\
    iii) $\Rightarrow$ i): Since $E / K$ is finite, the set of powers $1,a,a^2 \dots$ cannot
    be linearly independent over $K$. Thus, there are $n \in \NN$ and $b_0, b_{1}, \dots b_{n} \in K$
    such that $b_na^n + \dots b_0 = 0$ Put $P(T) = b_nT^n + \dots b_0 \in K[T]$. Then $P(a) = 0$,
    so $a$ is algebraic over $K$, as desired.
\end{proof}

\begin{cor} \label{Cor 3, Jan 30}
    Let $L / K$ be an extension, $a,b \in L$ algebraic over $K$. Then
    \begin{itemize}
        \item $a + b$, $a \cdot b$ are algebraic over $K$. $\deg(a + b), (a \cdot b) \leq \deg(a)
        \cdot \deg(b)$
        \item $a \neq 0$ $a^{-1}$ is algebraic over $K$ and $\deg(a^{-1}) = \deg(a)$
    \end{itemize}
\end{cor}

\begin{proof}
    By above proposition (\ref{Prop 2, Jan 30}), $K(a)$ and $K(b)$ are finite over $K$. \\
    Now, $a + b, a \cdot b \in K(a)\cdot K(b)$ and $[K(a) \cot K(b) : K] \leq [K(a) : K]
    \cdot [K(b): K]$. \\
    By \ref{Prop 6, Jan 27} $a + b$, and $a \cdot b$ are algebraic over $K$. Note that $a^{-1} \in K(a)$
    so $K(a^{-1}) = K(a)$
\end{proof}
\begin{defn} \label{Defn 5, Jan 30}
    An extension $L / K$ is called \textbf{algebraic}, if all $a \in L$ are algebraic
    over $K$.
\end{defn}
\begin{cor} \label{Cor 6, Jan 30}
    If $L / K$ is finite, then it is algebraic, and $\deg(a,K) \mid [L:K]$, for
    any $a \in L$.
\end{cor}
\begin{proof}
    \ref{Prop 2, Jan 30} and \ref{Cor 8, Jan 27}
\end{proof}

\begin{rmk}
    $a \in L$ is algebraic over $K \iff K(a) / K$ is algebraic.
\end{rmk}
\begin{rmk}
    There exist infinite algebraic extensions.
\end{rmk}

\section{February 1, 2017}
\begin{prop} \label{Prop 1, Feb 1}
    Let $M / L / K$ be a tower of extensions.
    \begin{enumerate}[1)]
        \item If $M / K$ is algebraic, then $M / L$ is algebraic.
        \item If $M / L$ and $L / K$ are algebraic, so is $M/K$.
    \end{enumerate}
\end{prop}

\begin{proof}
    Of 1). Let $a \in M$. There exist $P \neq 0 \in K[T]$ such that $P(a) = 0$. But
    $P$ can be viewed as an element of $L[T]$. \\
    Of 2). Let $a \in M$. Since $a$ is algebraic over $L$, there is $P \neq 0 \in L[T]$
    such that $P(a) = 0$. $P(T) = a_nT^n + \dots a_0, a_i \in L$. Each $a_i$ is
    algebraic over $K$, thus $[K(a_1, \dots a_n) : K] < \infty$ by \ref{Prop 2, Jan 30}. \\
    (This is implicitly inducting on that proposition, which said extending by a single
    element is still finite). \\
    Now, $P \in K(a_1 \dots a_0[T]$, so $a$ is algebraic over the field $K(a_1, \dots a_n)$.
    Thus, we have that (again by \ref{Prop 2, Jan 30}) that $[K(a_1, \dots a_n, a) : K(a_1, \dots a_n)] < \infty$
    By \ref{Cor 8, Jan 27}, $[K(a_1, \dots a_n, a) : K] = [K(a_1, \dots a_n, a) : K(a_1, \dots a_n)] \cdot
    [K(a_1, \dots a_n) : K] < \infty$. Thus, by \ref{Prop 2, Jan 30}, $M$ is algebraic.
\end{proof}

\begin{prop}\label{Prop 2, Feb 1}
    Let $P \neq 0 \in K[T]$ There exists a finite extension $L / K$ and $a \in L$
    such that $P(a) = 0$.
\end{prop}
\begin{proof}
    Let $Q$ be an irreducible factor of $P$. Let $L = K[T] / (Q)$, $a =$ image of $T$.
    Then $P(a) = 0$, as by coset operations this is exactly $P + (Q) = (Q) = 0$ in this
    quotient space. Since $Q$ is irreducible, it is prime (as $K[T]$ is a UFD). So,
    $L$ is a domain by \ref{Thm 8, Jan 6}. By \ref{Prop 4, Jan 27}, the set $1, T, \dots T^{n-1}$ is
    a basis for the $K$-vector space.
\end{proof}

\begin{rmk}
    Same is true for $P_1, \dots P_n \in K[T]$ by inducting on $n$.
\end{rmk}

\begin{cor} \label{Cor 3, Feb 1}
    Let $P \in K[T]$. There exists a finite extension $L / K$ $a_1 \dots a_k \in L$
    distinct, $m_1 \dots m_k \in \NN$ nonzero $c \in K$ such that
    $$
    P(T) = c(T - a_1)^{m_1} \cdots (T - a_k)^{m_k}
    $$
\end{cor}
\begin{proof}
    Induct on above proposition (\ref{Prop 2, Feb 1}) and \ref{Lemma 3, Jan 13}.
\end{proof}

\begin{thm} \label{Thm 4, Feb 1}
    Let $K$ be a field. There exists an algebraically closed extension $L / K$.
\end{thm}

\begin{lemma} \label{Lemma 5, Feb 1}
    Let $K$ be a field. There exists an extension $L / K$ such that for every
    $P \in K[T] \setminus K$ there is $a \in L$ with $P(a) = 0$.
\end{lemma}
\begin{proof}
    Let $S \in K[T] \setminus K$. Let $R = K[T_p \mid p \in S]$ (We construct polynomials
    with variables indexed by the set $S$. Thus, each polynomial gets an assoicated variable in this
    larger polynomial ring). Let $I$ be the ideal generated by $\{P(T_P) \mid p \in S \}$. For example,
    if $P = T^2 + 1$, then we consider $T_{T^2 + 1}^2 + 1$ as one generator of this ideal). Now,
    we do this for every polynomial in $K[T] \setminus K$). \\
    \textbf{Claim:} $I \neq R$. 
    \begin{proof}
        Assume that $I = R$. Then there are $r_1, \dots r_n \in R, P_1, \dots P_n \in S$
        such that
        $$
        1 = \sum{r_iP_i(T_{P_i})}
        $$
        By \ref{Prop 2, Feb 1}, there is a finite extension $L^\prime / K$ and $a_{P_1}, \dots
        a_{P_n} \in L^\prime$ such that $P_i(a_{P_i}) = 0$. For $P \in S$ not among the $P_1, \dots P_n$, choose
        $a_P \in L^\prime$ arbitrarily. We get a map $f:S \rightarrow L^\prime$ that sends
        $P \rightarrow a_P$. \\
        There exists a unique morphism $\phi: R \rightarrow L^\prime$ such that $\phi(T_P) = f(p)$ (This is
        from homework on the construction of the polynomials of too many variables). But,
        \begin{align*}
            \varphi(1) &= \varphi\left(\sum{r_iP_i(T_{P_i})}\right) \\
            &= \sum{r_i\varphi(P_i(T_{P_i}))} \\
            &= \sum{r_iP_i(\varphi(T_{P_i}))} \\
            &= \sum{r_iP_i(a_{P_i})} \\
            &= \sum{r_i0} \\
            &= 0 \neq 1
        \end{align*}
    \end{proof}
    Now, by \ref{Thm 2, Jan 9}, there exists a max ideal $I \subset \mathcal{M} \subset R$.
    Put $L = R / \mathcal{M}$ By \ref{Thm 8, Jan 6}, $L$ is a field. For every $P \in S$,
    let $a_P \in L$ be the image of $T_P$. Then $P(a_{P}) = 0$.
\end{proof}

\begin{proof}
    \textit{of Theorem \ref{Thm 4, Feb 1}}. We apply the above Lemma (\ref{Lemma 5, Feb 1}) inductively
    to obtain
    $$
    K = K_0 \subset K_1 \subset \dots
    $$
    Put $L = \bigcup_{n \geq 0}{K_n}$. Then $L$ is a field. To see that $L$ is algebraically
    closed, let $P \in L[T] \setminus L$. Write $P(T) = a_nT^n \dots a_0$.
    There exists $N$ such that $a_n \dots a_0 \in K_N$. Thus, $P \in K_N[T]$. By
    construction, we have that there is $a_P \in K_{N+1}$ such that $P(a_P) = 0$.
\end{proof}

\section{February 3, 2017}

\begin{defn}
Let $K$ be a field. An algebraic closure of $K$ is an algebraic extension $\bar{K}/K$ such that $\bar{K}$ is alg. closed.
\end{defn}
\begin{lemma}
Let $L/K$ be an extension of $K$ that is algebraically closed and $\bar{K}\subset L$ be the subfield of all elements algebraic over $K$. Then $\bar{K}$ is algebraically closed.
\end{lemma}
\begin{proof}
$P\in \bar{K}[T]-\bar{K}.$ Let $a\in L$ such that $P(a)=0$. Then $a$ is algebraic over $\bar{K},\bar{K}(a)$ is algebraic over $\bar{K}$. By Prop 1, Feb 1, $\bar{K}(a)$ is algebraic over $k$, hence $a$ is algebraic over $k$. Thus $a\in\bar{K}$.
\end{proof}
\begin{cor}
Every field has an algebraic closure.
\end{cor}
\begin{proof}
Thm 4, Feb 1 and Lemma 2.
\end{proof}
\begin{prop}
Let $K$ be a field, $G\subset K^{\star}$ is a finite subgroup. Then $G$ is cyclic.
\end{prop}
\begin{proof}
Commutative $G$ is a direct product of its Sylow subgroups. We will show that all these Sylow subgroups are isomorphic to $(\mathbb{Z}/p^n\mathbb{Z},t)$ for some $n$ and $p$. Then by the CRT we conclude that $G$ is cyclic. WOLOG assume $G$ has $p$-power order. That is, $|G|=p^n$. If $G$ is not isomorphic to $\mathbb{Z}/p^n\mathbb{Z}$, then there exists $m<n$ such that every element is killed by $p^m$. Then every element of $G$ is a root of $T^{p^m}-1$. But corollary 1, Jan 28, $T^{p^m}-1$ has at most $p^m$ distinct roots. Oops! This contradicts $|G|=p^n$.
\end{proof}
\noindent \textbf{Finite Fields}
\begin{fact}
If $F$ is a finite field, then its characteristic is $p>0$, and $|F|=p^n$ for some integer $n$.
\end{fact}
\begin{proof}
Since $F$ is finite, the canonical morphisms $\mathbb{Z}\longrightarrow F$ (Fact 0, Jan 4) cannot be injective. Hence the character of $F=p>0$. The image of this morphism is $\mathbb{F}_p$. $F$ is a vector space over $\mathbb{F}_p$, thus $|F|=p^n$ where $n$ is the dimension of this vector space.
\end{proof}
\begin{prop}
If $P$ is a prime, $n\in\mathbb{N}$, then there exists a finite field with $p^n$ elements.
\end{prop}
\begin{proof} Let $\bar{\mathbb{F}_p}$ be an algebraic closure of $\mathbb{F}_p$ (corollary 3). Consider the map $\sigma:\mathbb{F}_p\longrightarrow\bar{\mathbb{F}_p}$ defined by $x\longrightarrow x^{p^n}$. By hwk this is a ring homomorphism. Let $F\subset\bar{\mathbb{F}_p}$ be the subfield of elements fixed by $\sigma$. The elements of $F$ are precisely the roots of $T^{p^n}-T$. In $\bar{\mathbb{F}_p}$, this polynomial has $p^n$ many roots, counted with multiplicity. In order to have multiplicity 1, we need to check that the derivation of $P(T)=T^{p^n}-1$ is nonzero on these roots. $P'(T)=p^nT^{p^n-1}-1=-1\ne 0$ in $\bar{\mathbb{F}_p}$ which implies that all zeros have multiplicity 1, hence $|F|=p^n$.
\end{proof}
\begin{prop}
Let $F_1,F_2$ be finite fields of the same size. Then there exists an isomorphism from $F_1$ to $F_2$.
\end{prop}
\begin{proof}
Any $a\in F_n$ is algebraic over $\mathbb{F}_p$ and if $M_a\in \mathbb{F}_p(T)$, then $\mathbb{F}_p(a)$ is isomorphic to $\mathbb{F}_p[T]/(M_a)$.We have $[\mathbb{F}_p(a):\mathbb{F}_p]=\deg(M_a)$ by prop 1, Jan 27. On the other hand, $|F_1^{\times}|=p^n-1\longrightarrow a^{p^n}=a$ for all $a\in F_1$ which implies that $F_1$ consists of roots of $P(T)=T^{p^n}-T$ and $P(T)$ has precisely $p^n$ roots, $F_1$ consists precisely of the zero's of $P(T)$. In particular $P(T)$ factors into linear factors over $F_1$.
Thus the $M_a$ ($a\in F_i)$ are precisely the irreducible factors of $T^{p^n}-T=P(T)$. Now $F_1=\mathbb{F}_p(a)\iff |F_1|=p^n=|\mathbb{F}_p(a)|\iff n=\deg M_a$ in which case $f_1$ is isomorphic to $\mathbb{F}_p[T]/(M_a).$ By prop 4 we know that such an $a$ exists. Thus $F_1$ is isomorphic to $\mathbb{F}_p[T]/(Q)$ where $Q$ is any irreducible factor of $P(T)$ of deg $n$. Since the right hand side is independent of $F_1$ (just depends on $|F_1|$) we have $F_1$ is isomorphic to $F_2$.
\end{proof}
\begin{cor}
Let $F_1,F_2$ be finite fields. Then we can imbed $F_1$ into $F_2$ if and only if $|F_1|=p^k$ and $|F_2|=p^n$ with $k|n$. By proposition 6 and 7, $F_1$ is isomorphic to $\bar{\mathbb{F}_p}^{\sigma_1}$ and $F_2$ is isomorphic to $\bar{\mathbb{F}_p}^{\sigma_2},$ and $\sigma_1(x)=x^{p^k}$ and $\sigma_2(x)=x^{p^n}$ if $k|n$ then $F_1\subset F_2$. If $F_1$ imbeds into $F_2$ then $F_2$ is a vector space over $F_!$ and hence $|F_2|$ is a power of $|F_1|$ and so $k|n$.
\end{cor}

\section{February 6, 2017}

\begin{rmk}
For $L$, $K$ fields, if we want to make $L$ into an extension of $K$, we need to specify a morphism $K \rightarrow L$. (Note that any such morphism is injective, since it is a morphism of fields.) If there is such a morphism, there is usually more than one. \textbf{Galois theory} studies these morphisms.
\end{rmk}

\begin{defn} \label{Defn 1, Feb 6}
Let $L/K$ be an extension, $P \in K[T] \setminus K$. $L$ is a \textbf{minimal splitting field of $P$} provided that i) $P$ splits into linear factors over $L$, and ii) $L$ is generated by the zeroes of $P$ (i.e. $L = K(a_1, ..., a_n)$ where $\{a \in L \colon P(a) = 0\} = \{a_1, ..., a_n\}$).
\end{defn}

\begin{defn} \label{Defn 2, Feb 6}
Let $L_1/K$, $L_2/K$ be field extensions. A \textbf{morphism of extensions} $L_1 \rightarrow L_2$ is a morphism of $K$-algebras between field extensions of $K$. The set of all such morphisms is notated Mor$_K(L_1, L_2)$.
\end{defn}

\begin{rmk}
If $K$ is a field, $(L_1, \varphi_1)$, $(L_2, \varphi_2)$ are extensions, and $f \colon L_1 \rightarrow L_2$ is a morphism of extensions, i.e.
\begin{center}
\begin{tikzcd}
L_1 \arrow{rr}{f} & & L_2 \\
& K \arrow{ul}{\varphi_1} \arrow{ur}[swap]{\varphi_2}
\end{tikzcd}
\end{center}
then we say that $f$ \textit{extends} $\varphi_2$.
\end{rmk}

\begin{defn} \label{Defn 3, Feb 6}
An extension $L/K$ is \textbf{primitive} provided that $L = K(a)$ for some $a \in L$.
\end{defn}

\begin{prop} \label{Prop 4, Feb 6}
Let $K(a)/K$ be a primitive extension, $M/K$ be an arbitrary extension. Then
\begin{align*}
\text{Mor}_K(K(a), M) &\rightarrow \{x \in M \colon M_{a/K}(x) = 0\} \\
\varphi &\mapsto \varphi(a)
\end{align*}
is a bijection.
\end{prop}

\begin{proof}
To show the map is well-defined, i.e. maps into the intended target, consider $\varphi \in \text{Mor}_K(K(a), M)$. Then $M_{a/K}(\varphi(a)) = \varphi(M_{a/K}(a)) = \varphi(0) = 0$, as desired.

To show injectivity, let $\phi, \psi \in \text{Mor}_K(K(a), M)$ s.t. $\phi(a) = \psi(a)$. Let $\rho \colon K \rightarrow M$ be the morphism defining $M$ as a $K$-extension. Then for each $y \in K(a)$ we have $y = k_na^n + \cdots + k_1a + k_0$ for $k_i \in K$ (if this is not obvious, one might peruse \ref{Prop 1, Jan 30}). Then
\begin{align*}
\phi(y) &= \phi(k_na^n + \cdots + k_1a + k_0) \\
&= \sum_{i = 0}^n \phi(k_i) \phi(a) \\
&= \sum_{i = 0}^n \rho(k_i) \phi(a) \\
&= \sum_{i = 0}^n \rho(k_i) \psi(a) \\
&= \sum_{i = 0}^n \psi(k_i) \psi(a) \\
&= \psi(y),
\end{align*}
by the commutativity of the natural extension diagram (see Remark 1.2 above), so we conclude $\phi \equiv \psi$.

To show surjectivity, let $x \in M$ be s.t. $M_{a/K}(x) = 0$. Consider the evaluation morphism $\phi_x \colon K[T] \rightarrow M$ defined by $T \mapsto x$. (see \ref{Fact 1, Jan 13}) Then $M_{a/K} \in \ker(\phi_x)$, so by the homomorphism theorem (\ref{Thm 3, Jan 6}) $\phi_x$ factors through the quotient $K[T]/(M_{a/K})$, i.e. $\phi_x = K[T] \stackrel{\pi}{\rightarrow} K[T]/(M_{a/K}) \stackrel{:)}{\rightarrow} M$, however we have that $K(a) \simeq K[T]/(M_{a/K})$ by \ref{Prop 1, Jan 30}, so we have the following diagram
\begin{center}
\begin{tikzcd}
K[T] \arrow{rr}{\phi_x} \arrow{dr}{\pi} & & M \\
K(a) \arrow{r}{:(} &  K[T]/(M_{a/K}) \arrow{ru}{:)}
\end{tikzcd}
\end{center}
where $:($ is an isomorphism. Letting $F = \text{:)} \circ \text{:(}$ we have $F(a) = x \in M$ as desired. We conclude that the given map is indeed a bijection.
\end{proof}

\begin{cor} \label{Cor 5, Feb 6}
Let $L/K$ be a finite extension generated by $a_1, ..., a_n \in L$. Let $M/K$ be an extension s.t. each minimal polynomial $M_{a_1/K}, ..., M_{a_n/K}$ admits a root in $M$. Then Mor$_K(L, M) \neq \emptyset$.
\end{cor}

\begin{proof}
We consider the restriction maps
\[\text{Mor}_K(K(a_1, ..., a_n), M) \stackrel{\text{Res}}{\rightarrow} \text{Mor}_K(K(a_1, ..., a_{n-1}), M) \stackrel{\text{Res}}{\rightarrow} \cdots \stackrel{\text{Res}}{\rightarrow} \text{Mor}_K(K, M).\]
Each restriction map is surjective, so there composition is as well. Then since Mor$_K(K, M) \ni (K \hookrightarrow M)$, i.e. the map defining $M$ as a $K$-extension, then we have $\text{Mor}_K(K(a_1, ..., a_n), M) \neq \emptyset$ as desired.
\end{proof}

\begin{rmk}
To understand what's going on here, consider the $n=2$ case. We have the following diagram:
\begin{center}
\begin{tikzcd}
K(a_1, a_2) \arrow[dotted]{rrd}{(a)} \\
K(a_1) \arrow[hookrightarrow]{u} \arrow[dotted]{rr}{(b)} & & M \\
 & K \arrow[hookrightarrow]{lu} \arrow{ru}
\end{tikzcd}
\end{center}
The dashed arrow marked (a) represents all morphisms that make the outer diagram commute; the set of all these morphisms is Mor$_K(K(a_1, a_2), M)$. It restricts to the set of all morphisms making the inner diagram commute, represented by the dashed arrow marked (b); this set is Mor$_K(K(a_1), M)$. The restriction map functions as follows.
\begin{align*}
\text{Mor}_K(K(a_1, a_2), M) &\stackrel{\text{Res}}{\rightarrow} \text{Mor}_K(K(a_1), M) \\
\text{Mor}_{K(a_1)}(K(a_1, a_2), M) &\mapsto \phi.
\end{align*}
That is: think of $\phi$ in the role of (b) in the diagram. Then anything that maps to $\phi$ under Res is a map in the role of (a) that makes the outer diagram commute, but in order to restrict to $\phi$, the upper triangle must commute as well. The set of all such maps, those that make the outer and upper diagrams commute, is Mor$_{K(a_1)}(K(a_1, a_2), M) \subset \text{Mor}_K(K(a_1, a_2), M)$. Since $K(a_1, a_2)$ is a primitive extension of $K(a_1)$, we apply \ref{Prop 4, Feb 6} above to get the result that Mor$_{K(a_1)}(K(a_1, a_2), M) \neq \emptyset$. This process is repeated arbitrarily many times in the proof of \ref{Cor 5, Feb 6}. The following corollary proves a similar result for arbitrary (non-primitive) chains of algebraic extensions.
\end{rmk}

\begin{cor} \label{Cor 6, Feb 6}
Let $L/K$ be an algebraic extension and let $M/K$ be an algebraically closed extension. Then Mor$_K(L,M) \neq \emptyset$.
\end{cor}

\begin{proof}
Define the set $\mathcal{M}$ of pairs $(L', \varphi')$ where $L/L'/K$ and $\varphi' \in \text{Mor}_K(L', M)$. Introduce a partial order on $\mathcal{M}$ by
\[(L', \varphi') \leq (L'', \varphi'') \Leftrightarrow L' \subset L'', \quad \left. \varphi''\right|_{L'} = \varphi'.\]
Then let $C \subset M$ be a chain (i.e. a totally ordered subset). $C$ comprises the following diagram
\begin{center}
\begin{tikzcd}
\vdots \arrow[dotted]{rdd} & \\
L'' \arrow[hookrightarrow]{u} \arrow{rd}{\varphi''} & \\
L' \arrow[hookrightarrow]{u} \arrow{r}{\varphi'} & M
\end{tikzcd}
\end{center}
Then taking the union of all such $L \in C$, we get an upper bound for $C$. $\mathcal{M} \neq \emptyset$ since letting $\mathcal{M} = K$ and $\varphi$ be the identity gives an element. Then by Zorn's lemma there exists a maximal element $(L_m, \varphi_m) \in \mathcal{M}$. We want to show that this is in fact $L$. If not, let $a \in L \setminus L_m$. Then $\varphi_m$ makes $M$ into an $L_m$-algebra, and since $M$ is algebraically closed, $M_{a/Lm}$ admits a zero in $M$. By \ref{Prop 4, Feb 6}, Mor$_{L_m}(L_m(a), L_m) \neq \emptyset$. Taking $\varphi^{\text{contr}} \in \text{Mor}_{L_m}(L_m(a), L_m)$, the existence of $(L_m(a), \varphi^{\text{contr}})$ contradicts the maximality of $L_m$, closing the proof.
\end{proof}

\section{February 8, 2017}

\begin{prop} \label{Prop 1, Feb 8}
    Let $L_1, L_2$ be two minimal splitting fields of $P \in K[T] \setminus K$.
    Then $L_1 / K \cong L_2 / K$
\end{prop}

\begin{proof}
    We note that both $L_2$ and $L_1$ are $K$-algebras. By (Cor 5, Feb 6 (FIXME)),
    there exist $K$-algebra morphisms $\varphi_2: L_2 \rightarrow L_1$ and $\varphi_1: L_1 \rightarrow
    L_2$.
    \[
\begin{tikzcd}[,every arrow/.append style={maps to}
,every label/.append style={font=\normalsize},column sep=1.5em]
L_1  \arrow[rr, dashrightarrow] && L_2 \\
 & \arrow[ul, hook] K \arrow[ur, hook] \\
\end{tikzcd}
\]
We know that both of these spaces are finite dimensional (so $\dim_K(L_1) < \infty$
and $\dim_K(L_2) < \infty$ ). These maps are $K$-linear, and are also morphism
of fields, and thus they are injective. Since the dimensions are finite and we
have injective maps in both directions, we have that $\varphi_1$ and $\varphi_2$ are
in fact isomorphisms.
\end{proof}

\begin{prop} \label{Prop 2, Feb 8}
    Let $L / K$ be a finite extension. The following are equivalent.
    \begin{enumerate}
        \item $L$ is the minimal splitting field of some $P \in K[T] \setminus K$.
        \item For every $a \in L$, $M_{a/K}$ decomposes into linear factors in $L[T]$.
        \item For any extension $M/K$, all $K$-algebra morphisms $L \rightarrow M$
        have the same image.
    \end{enumerate}
\end{prop}
\begin{defn} \label{Defn 3, Feb 8}
    A finite extension $L / K$ is called \textbf{normal} if it satisfies the conditions
    above.
\end{defn}
\begin{ex}
    $K = \QQ$, $L = \QQ[T] / (T^2 + 1)$. \\
    Checking the conditions: If we consider $M = \CC$, $a \in L$ defined as the image of $T$
    (that is, the coset $T + (T^2 + 1)$) we know that the set of $K$ algebra morphisms
    from $K(a)$ (which is isomorphic to $L$) we have a bijection between morphisms
    and roots of $T^2 + 1 \in \CC$. namely $i$ and $-i$. So, the two
    morphisms are defined by sending $a \rightarrow i$ and $a \rightarrow -i$. Both
    of these morphisms have image exactly $\QQ[i]$.
\end{ex}
\begin{ex}
    A nonexample: $K = \QQ$, $L = \QQ[T] / (T^3 - 2)$. \\
    Again by logic above, we have three morphisms defined for all the
    roots of $T^3 - 2$ in $\CC$, which are $\sqrt[3]{2}$ multiplied by the
    3rd roots of unity. However, we note that the morphism that sends
    $a \mapsto \sqrt[3]{2}$ has image $\{a + \sqrt[3]{2}b \mid a, b \in \QQ\} \subset \RR$ \\
    However, the others, as they have complex parts in their roots of unity, do not
    map into $\RR$. Thus, the images do not agree. \\
    Some intuition: $\QQ[T] / (T^3 - 2)$ adds a single root but not all three needed
    to split. This, in contrast with $\QQ(\zeta_3)[T] / (T^3 - 2)$ (where $\zeta_3$ is a non
    real third root of unity) only needs to add $\sqrt[3]{2}$ to split.
\end{ex}

\section{February 10, 2017}

\begin{proof}
	(\textit{of \ref{Prop 2, Feb 8}}) We prove the loop $1 \Rightarrow 3 \Rightarrow 2 \Rightarrow 1$.

	$1 \Rightarrow 3$) It's enough to show this for algebraically closed $M$. For if we've shown the result for $\overline{M}$, the restriction $\im(\Mor_K(L, \overline{M}))$ to $M$ gives the result for general $M$.

	Now take $L/K$ the minimal splitting field for $P$, i.e. finite and satisfying $L = K(a_1, ..., a_k)$ for $a_i \in L$ the zeroes of $P$. Then we have $P(T) = (T - a_1)^{m_1} \cdots (T - a_k)^{m_k}$. Now take $f \in \Mor_K(L, M)$ for $M/K$ an algebraically closed extension. Then
	\[\prod_{i = 1}^k (T - a_i)^{m_i} = P(T) \stackrel{(1)}{=} f(P)(T) = \prod_{i = 1}^k (T - f(a_i))^{m_i}\]
	where (1) is true since $f$ fixes each $k \in K$, so in particular it fixes $P \in K[T]$. Then we have that $M \supset \{f(a_i) \colon a_i \text{ a zero for } P\} = \{a_i \colon \text{zeroes of } P\}$, so $M$ contains the zeroes of $P$.

	Then $f(L)$ is a subfield of $M$ generated by the $f(a_i)$, but these are the zeroes of $P$ in $M$, which are independent of $f$. Then each $f \in \Mor_K(L, M)$ are generated by the same elements, i.e. have the same image.

	$3 \Rightarrow 2$) Let $a \in L$ and choose an algebraic closure $\overline{L}$ of L (exists by \ref{Cor 3, Feb 3}). Then let $a = a_1, ..., a_k \in \overline{L}$ be the zeroes of $M_{a/K}$ in $\overline{L}$. By \ref{Prop 4, Feb 6} we have a one-to-one correspondence $\Mor_K{K(a), \overline{L}} \leftrightarrow \{\text{zeroes of $M_a/K$ in } M\}$. Then we may choose $\psi_i \colon K(a) \rightarrow M$ s.t. $\psi_i(a) = a_i$. Then as per \ref{Cor 5, Feb 6}, choose an extension $\widetilde{\psi}_i \colon L \rightarrow \overline{L}$ of $\psi_i$. Then $\widetilde{\psi}_i \in \Mor_K(L, \overline{L})$ with $a \mapsto a_i$. Then by assumption, the inclusion map $L \hookrightarrow \overline{L}$ has the same image as $\widetilde{\psi}_i$, so $a_i \in L$.

	$2 \Rightarrow 1$) Write $L = K(a_1, ..., a_n)$ for some $a_1, ..., a_n \in L$. Then letting $P = \prod_{i = 1}^n M_{a_i/K} \in K[T]$, we have that $L$ is the minimal splitting field of a polynomial $P \in K[T]$.
\end{proof}

\begin{fact} \label{Fact 1, Feb 10}
	Let $L/K$ be a finite extension. Then there exists a finite extension $N/L$ s.t. $N/K$ is normal.
\end{fact}

\begin{proof}
	Write $L = K(a_1,..., a_n)$ and put $P = \prod M_{a_i/K}$. The splitting field $N$ of $P$ is as needed. $N$ is the \textbf{normal closure} of $L/K$.
\end{proof}

\begin{fact} \label{Fact 2, Feb 10}
	If $M/L/K$ is a tower of finite extensions with $M$ normal over $K$, then $M$ is normal over $L$.
\end{fact}

\begin{proof}
	If $P \in K[T]$ is the polynomial such that $M$ is the splitting field of $K$, we have $P \in L[T]$.
\end{proof}

\begin{prop} \label{Prop 3, Feb 10}
	Let $L/K$, $M/K$ be extensions with $L/K$ finite. Then
	\[|\Mor_K(L,M)| \leq [L : K].\]
\end{prop}

\begin{proof}
	We have $L = K(a_1, ..., a_n)$ for some $a_1, ..., a_n \in L$. We induct on $n$.

	For $n = 1$, by \ref{Prop 4, Feb 6}, we have
	\begin{align*}
	|\Mor_K(L,M)| &= |\{x \in M \colon M_{a/K}(x) = 0\}| \\
	&\leq \deg(M_{a/K}) \quad \text{by \ref{Cor 1, Jan 18},} \\
	&= [L : K] \qquad \quad \text{by \ref{Prop 1, Jan 30}.}
	\end{align*}
	For induction, consider $K(a_1, ..., a_n)/K(a_1, ..., a_{n-1})/K$.  Then
	\begin{align*}
	\Mor_K(K(a_1, ..., a_n), M) &\stackrel{\text{Res}}{\rightarrow} \Mor_K(K(a_1, ..., a_{n-1}), M) \\
	\Mor_{K(a_1, ..., a_{n-1})}(K(a_1, ..., a_n), M) &\mapsto \{\varphi\}.
	\end{align*}
	Then
	\begin{align*}
	|\Mor_{K(a_1, ..., a_{n-1})}(K(a_1, ..., a_n), M)| &\leq [K(a_1, ..., a_n) : K(a_1, ..., a_{n-1})] \text{ by base case} \\
	|\Mor_K(K(a_1, ..., a_{n-1}), M)| &\leq [K(a_1, ..., a_{n-1}), K] \qquad \qquad \quad \text{ by hypothesis,} \\
	\Rightarrow [K(a_1, ..., a_n), M] &\leq [K(a_1, ..., a_n) : K(a_1, ..., a_{n-1})] \cdot [K(a_1, ..., a_{n-1}) : K] \text{ by \ref{Cor 8, Jan 27},} \\
	&= [K(a_1, ..., a_n) : K] \\
	&= [L : K]
	\end{align*}
	with $[K(a_1, ..., a_n), M] = |\Mor_K(K(a_1, ..., a_n), M)|$. This completes the proof.
\end{proof}

\begin{rmk}
	If we have equality in the base case, it descends through the inductive case. When is this true? The answer to this question motivates the following:
\end{rmk}

\begin{defn} \label{Defn 4, Feb 10}
	Let $L/K$ be a finite extension. An element $a \in L$ is called \textbf{separable over $K$} provided that all zeroes of $M_{a/K}$ in its splitting field are simple, i.e. have order 1.
\end{defn}

\begin{fact} \label{Fact 5, Feb 10}
	Let $M/L/K$ be a finite tower. If $a \in M$ is separable over $K$, then it is separable over $L$.
\end{fact}

\begin{proof}
	This is a quick exercise in definitions. We have $M_{a/L} \mid M_{a/K}$, so if $M_{a/K}$ splits simply in its splitting field, so must $M_{a/L}$ in the splitting field reduced to the field generated by its zeroes. Thus $a$ is separable over $L$.
\end{proof}

\end{document}
