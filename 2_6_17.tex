\section{February 6, 2017}

\begin{rmk}
For $L$, $K$ fields, if we want to make $L$ into an extension of $K$, we need to specify a morphism $K \rightarrow L$. (Note that any such morphism is injective, since it is a morphism of fields.) If there is such a morphism, there is usually more than one. \textbf{Galois theory} studies these morphisms.
\end{rmk}

\begin{defn} \label{Defn 1, Feb 6}
Let $L/K$ be an extension, $P \in K[T] \setminus K$. $L$ is a \textbf{minimal splitting field of $P$} provided that i) $P$ splits into linear factors over $L$, and ii) $L$ is generated by the zeroes of $P$ (i.e. $L = K(a_1, ..., a_n)$ where $\{a \in L \colon P(a) = 0\} = \{a_1, ..., a_n\}$).
\end{defn}

\begin{defn} \label{Defn 2, Feb 6}
Let $L_1/K$, $L_2/K$ be field extensions. A \textbf{morphism of extensions} $L_1 \rightarrow L_2$ is a morphism of $K$-algebras between field extensions of $K$. The set of all such morphisms is notated Mor$_K(L_1, L_2)$.
\end{defn}

\begin{rmk}
If $K$ is a field, $(L_1, \varphi_1)$, $(L_2, \varphi_2)$ are extensions, and $f \colon L_1 \rightarrow L_2$ is a morphism of extensions, i.e.
\begin{center}
\begin{tikzcd}
L_1 \arrow{rr}{f} & & L_2 \\
& K \arrow{ul}{\varphi_1} \arrow{ur}[swap]{\varphi_2}
\end{tikzcd}
\end{center}
then we say that $f$ \textit{extends} $\varphi_2$.
\end{rmk}

\begin{defn} \label{Defn 3, Feb 6}
An extension $L/K$ is \textbf{primitive} provided that $L = K(a)$ for some $a \in L$.
\end{defn}

\begin{prop} \label{Prop 4, Feb 6}
Let $K(a)/K$ be a primitive extension, $M/K$ be an arbitrary extension. Then
\begin{align*}
\text{Mor}_K(K(a), M) &\rightarrow \{x \in M \colon M_{a/K}(x) = 0\} \\
\varphi &\mapsto \varphi(a)
\end{align*}
is a bijection.
\end{prop}

\begin{proof}
To show the map is well-defined, i.e. maps into the intended target, consider $\varphi \in \text{Mor}_K(K(a), M)$. Then $M_{a/K}(\varphi(a)) = \varphi(M_{a/K}(a)) = \varphi(0) = 0$, as desired.

To show injectivity, let $\phi, \psi \in \text{Mor}_K(K(a), M)$ s.t. $\phi(a) = \psi(a)$. Let $\rho \colon K \rightarrow M$ be the morphism defining $M$ as a $K$-extension. Then for each $y \in K(a)$ we have $y = k_na^n + \cdots + k_1a + k_0$ for $k_i \in K$ (if this is not obvious, one might peruse \ref{Prop 1, Jan 30}). Then
\begin{align*}
\phi(y) &= \phi(k_na^n + \cdots + k_1a + k_0) \\
&= \sum_{i = 0}^n \phi(k_i) \phi(a) \\
&= \sum_{i = 0}^n \rho(k_i) \phi(a) \\
&= \sum_{i = 0}^n \rho(k_i) \psi(a) \\
&= \sum_{i = 0}^n \psi(k_i) \psi(a) \\
&= \psi(y),
\end{align*}
by the commutativity of the natural extension diagram (see Remark 1.2 above), so we conclude $\phi \equiv \psi$.

To show surjectivity, let $x \in M$ be s.t. $M_{a/K}(x) = 0$. Consider the evaluation morphism $\phi_x \colon K[T] \rightarrow M$ defined by $T \mapsto x$. (see \ref{Fact 1, Jan 13}) Then $M_{a/K} \in \ker(\phi_x)$, so by the homomorphism theorem (\ref{Thm 3, Jan 6}) $\phi_x$ factors through the quotient $K[T]/(M_{a/K})$, i.e. $\phi_x = K[T] \stackrel{\pi}{\rightarrow} K[T]/(M_{a/K}) \stackrel{:)}{\rightarrow} M$, however we have that $K(a) \simeq K[T]/(M_{a/K})$ by \ref{Prop 1, Jan 30}, so we have the following diagram
\begin{center}
\begin{tikzcd}
K[T] \arrow{rr}{\phi_x} \arrow{dr}{\pi} & & M \\
K(a) \arrow{r}{:(} &  K[T]/(M_{a/K}) \arrow{ru}{:)}
\end{tikzcd}
\end{center}
where $:($ is an isomorphism. Letting $F = \text{:)} \circ \text{:(}$ we have $F(a) = x \in M$ as desired. We conclude that the given map is indeed a bijection.
\end{proof}

\begin{cor} \label{Cor 5, Feb 6}
Let $L/K$ be a finite extension generated by $a_1, ..., a_n \in L$. Let $M/K$ be an extension s.t. each minimal polynomial $M_{a_1/K}, ..., M_{a_n/K}$ admits a root in $M$. Then Mor$_K(L, M) \neq \emptyset$.
\end{cor}

\begin{proof}
We consider the restriction maps
\[\text{Mor}_K(K(a_1, ..., a_n), M) \stackrel{\text{Res}}{\rightarrow} \text{Mor}_K(K(a_1, ..., a_{n-1}), M) \stackrel{\text{Res}}{\rightarrow} \cdots \stackrel{\text{Res}}{\rightarrow} \text{Mor}_K(K, M).\]
Each restriction map is surjective, so there composition is as well. Then since Mor$_K(K, M) \ni (K \hookrightarrow M)$, i.e. the map defining $M$ as a $K$-extension, then we have $\text{Mor}_K(K(a_1, ..., a_n), M) \neq \emptyset$ as desired.
\end{proof}

\begin{rmk}
To understand what's going on here, consider the $n=2$ case. We have the following diagram:
\begin{center}
\begin{tikzcd}
K(a_1, a_2) \arrow[dotted]{rrd}{(a)} \\
K(a_1) \arrow[hookrightarrow]{u} \arrow[dotted]{rr}{(b)} & & M \\
 & K \arrow[hookrightarrow]{lu} \arrow{ru}
\end{tikzcd}
\end{center}
The dashed arrow marked (a) represents all morphisms that make the outer diagram commute; the set of all these morphisms is Mor$_K(K(a_1, a_2), M)$. It restricts to the set of all morphisms making the inner diagram commute, represented by the dashed arrow marked (b); this set is Mor$_K(K(a_1), M)$. The restriction map functions as follows.
\begin{align*}
\text{Mor}_K(K(a_1, a_2), M) &\stackrel{\text{Res}}{\rightarrow} \text{Mor}_K(K(a_1), M) \\
\text{Mor}_{K(a_1)}(K(a_1, a_2), M) &\mapsto \phi.
\end{align*}
That is: think of $\phi$ in the role of (b) in the diagram. Then anything that maps to $\phi$ under Res is a map in the role of (a) that makes the outer diagram commute, but in order to restrict to $\phi$, the upper triangle must commute as well. The set of all such maps, those that make the outer and upper diagrams commute, is Mor$_{K(a_1)}(K(a_1, a_2), M) \subset \text{Mor}_K(K(a_1, a_2), M)$. Since $K(a_1, a_2)$ is a primitive extension of $K(a_1)$, we apply \ref{Prop 4, Feb 6} above to get the result that Mor$_{K(a_1)}(K(a_1, a_2), M) \neq \emptyset$. This process is repeated arbitrarily many times in the proof of \ref{Cor 5, Feb 6}. The following corollary proves a similar result for arbitrary (non-primitive) chains of algebraic extensions.
\end{rmk}

\begin{cor} \label{Cor 6, Feb 6}
Let $L/K$ be an algebraic extension and let $M/K$ be an algebraically closed extension. Then Mor$_K(L,M) \neq \emptyset$.
\end{cor}

\begin{proof}
Define the set $\mathcal{M}$ of pairs $(L', \varphi')$ where $L/L'/K$ and $\varphi' \in \text{Mor}_K(L', M)$. Introduce a partial order on $\mathcal{M}$ by
\[(L', \varphi') \leq (L'', \varphi'') \Leftrightarrow L' \subset L'', \quad \left. \varphi''\right|_{L'} = \varphi'.\]
Then let $C \subset M$ be a chain (i.e. a totally ordered subset). $C$ comprises the following diagram
\begin{center}
\begin{tikzcd}
\vdots \arrow[dotted]{rdd} & \\
L'' \arrow[hookrightarrow]{u} \arrow{rd}{\varphi''} & \\
L' \arrow[hookrightarrow]{u} \arrow{r}{\varphi'} & M
\end{tikzcd}
\end{center}
Then taking the union of all such $L \in C$, we get an upper bound for $C$. $\mathcal{M} \neq \emptyset$ since letting $\mathcal{M} = K$ and $\varphi$ be the identity gives an element. Then by Zorn's lemma there exists a maximal element $(L_m, \varphi_m) \in \mathcal{M}$. We want to show that this is in fact $L$. If not, let $a \in L \setminus L_m$. Then $\varphi_m$ makes $M$ into an $L_m$-algebra, and since $M$ is algebraically closed, $M_{a/Lm}$ admits a zero in $M$. By \ref{Prop 4, Feb 6}, Mor$_{L_m}(L_m(a), L_m) \neq \emptyset$. Taking $\varphi^{\text{contr}} \in \text{Mor}_{L_m}(L_m(a), L_m)$, the existence of $(L_m(a), \varphi^{\text{contr}})$ contradicts the maximality of $L_m$, closing the proof.
\end{proof}
