\section{February 10, 2017}

\begin{proof}
	(\textit{of \ref{Prop 2, Feb 8}}) We prove the loop $1 \Rightarrow 3 \Rightarrow 2 \Rightarrow 1$.

	$1 \Rightarrow 3$) It's enough to show this for algebraically closed $M$. For if we've shown the result for $\overline{M}$, the restriction $\im(\Mor_K(L, \overline{M}))$ to $M$ gives the result for general $M$.

	Now take $L/K$ the minimal splitting field for $P$, i.e. finite and satisfying $L = K(a_1, ..., a_k)$ for $a_i \in L$ the zeroes of $P$. Then we have $P(T) = (T - a_1)^{m_1} \cdots (T - a_k)^{m_k}$. Now take $f \in \Mor_K(L, M)$ for $M/K$ an algebraically closed extension. Then
	\[\prod_{i = 1}^k (T - a_i)^{m_i} = P(T) \stackrel{(1)}{=} f(P)(T) = \prod_{i = 1}^k (T - f(a_i))^{m_i}\]
	where (1) is true since $f$ fixes each $k \in K$, so in particular it fixes $P \in K[T]$. Then we have that $M \supset \{f(a_i) \colon a_i \text{ a zero for } P\} = \{a_i \colon \text{zeroes of } P\}$, so $M$ contains the zeroes of $P$.

	Then $f(L)$ is a subfield of $M$ generated by the $f(a_i)$, but these are the zeroes of $P$ in $M$, which are independent of $f$. Then each $f \in \Mor_K(L, M)$ are generated by the same elements, i.e. have the same image.

	$3 \Rightarrow 2$) Let $a \in L$ and choose an algebraic closure $\overline{L}$ of L (exists by \ref{Cor 3, Feb 3}). Then let $a = a_1, ..., a_k \in \overline{L}$ be the zeroes of $M_{a/K}$ in $\overline{L}$. By \ref{Prop 4, Feb 6} we have a one-to-one correspondence $\Mor_K{K(a), \overline{L}} \leftrightarrow \{\text{zeroes of $M_a/K$ in } M\}$. Then we may choose $\psi_i \colon K(a) \rightarrow M$ s.t. $\psi_i(a) = a_i$. Then as per \ref{Cor 5, Feb 6}, choose an extension $\widetilde{\psi}_i \colon L \rightarrow \overline{L}$ of $\psi_i$. Then $\widetilde{\psi}_i \in \Mor_K(L, \overline{L})$ with $a \mapsto a_i$. Then by assumption, the inclusion map $L \hookrightarrow \overline{L}$ has the same image as $\widetilde{\psi}_i$, so $a_i \in L$.

	$2 \Rightarrow 1$) Write $L = K(a_1, ..., a_n)$ for some $a_1, ..., a_n \in L$. Then letting $P = \prod_{i = 1}^n M_{a_i/K} \in K[T]$, we have that $L$ is the minimal splitting field of a polynomial $P \in K[T]$.
\end{proof}

\begin{fact} \label{Fact 1, Feb 10}
	Let $L/K$ be a finite extension. Then there exists a finite extension $N/L$ s.t. $N/K$ is normal.
\end{fact}

\begin{proof}
	Write $L = K(a_1,..., a_n)$ and put $P = \prod M_{a_i/K}$. The splitting field $N$ of $P$ is as needed. $N$ is the \textbf{normal closure} of $L/K$.
\end{proof}

\begin{fact} \label{Fact 2, Feb 10}
	If $M/L/K$ is a tower of finite extensions with $M$ normal over $K$, then $M$ is normal over $L$.
\end{fact}

\begin{proof}
	If $P \in K[T]$ is the polynomial such that $M$ is the splitting field of $K$, we have $P \in L[T]$.
\end{proof}

\begin{prop} \label{Prop 3, Feb 10}
	Let $L/K$, $M/K$ be extensions with $L/K$ finite. Then
	\[|\Mor_K(L,M)| \leq [L : K].\]
\end{prop}

\begin{proof}
	We have $L = K(a_1, ..., a_n)$ for some $a_1, ..., a_n \in L$. We induct on $n$.

	For $n = 1$, by \ref{Prop 4, Feb 6}, we have
	\begin{align*}
	|\Mor_K(L,M)| &= |\{x \in M \colon M_{a/K}(x) = 0\}| \\
	&\leq \deg(M_{a/K}) \quad \text{by \ref{Cor 1, Jan 18},} \\
	&= [L : K] \qquad \quad \text{by \ref{Prop 1, Jan 30}.}
	\end{align*}
	For induction, consider $K(a_1, ..., a_n)/K(a_1, ..., a_{n-1})/K$.  Then
	\begin{align*}
	\Mor_K(K(a_1, ..., a_n), M) &\stackrel{\text{Res}}{\rightarrow} \Mor_K(K(a_1, ..., a_{n-1}), M) \\
	\Mor_{K(a_1, ..., a_{n-1})}(K(a_1, ..., a_n), M) &\mapsto \{\varphi\}.
	\end{align*}
	Then
	\begin{align*}
	|\Mor_{K(a_1, ..., a_{n-1})}(K(a_1, ..., a_n), M)| &\leq [K(a_1, ..., a_n) : K(a_1, ..., a_{n-1})] \text{ by base case} \\
	|\Mor_K(K(a_1, ..., a_{n-1}), M)| &\leq [K(a_1, ..., a_{n-1}), K] \qquad \qquad \quad \text{ by hypothesis,} \\
	\Rightarrow [K(a_1, ..., a_n), M] &\leq [K(a_1, ..., a_n) : K(a_1, ..., a_{n-1})] \cdot [K(a_1, ..., a_{n-1}) : K] \text{ by \ref{Cor 8, Jan 27},} \\
	&= [K(a_1, ..., a_n) : K] \\
	&= [L : K]
	\end{align*}
	with $[K(a_1, ..., a_n), M] = |\Mor_K(K(a_1, ..., a_n), M)|$. This completes the proof.
\end{proof}

\begin{rmk}
	If we have equality in the base case, it descends through the inductive case. When is this true? The answer to this question motivates the following:
\end{rmk}

\begin{defn} \label{Defn 4, Feb 10}
	Let $L/K$ be a finite extension. An element $a \in L$ is called \textbf{separable over $K$} provided that all zeroes of $M_{a/K}$ in its splitting field are simple, i.e. have order 1.
\end{defn}

\begin{fact} \label{Fact 5, Feb 10}
	Let $M/L/K$ be a finite tower. If $a \in M$ is separable over $K$, then it is separable over $L$.
\end{fact}

\begin{proof}
	This is a quick exercise in definitions. We have $M_{a/L} \mid M_{a/K}$, so if $M_{a/K}$ splits simply in its splitting field, so must $M_{a/L}$ in the splitting field reduced to the field generated by its zeroes. Thus $a$ is separable over $L$.
\end{proof}
