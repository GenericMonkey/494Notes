\section{January 20, 2017}
\begin{prop}
    Let $R$ be a domain with factorization. Then $R$ has unique factorization if and
    only if every irreducible element is prime.
\end{prop}
\begin{proof}
    Say $R$ has unique factorization. Let $p$ be irreducible. We will show that
    $p$ is prime. Let $a,b \in R$ be such that $p \mid ab$. Using factorization,
    we write that $a = \zeta u_1u_2 \cdots u_k$, $b = \mu v_1\cdots v_l$, and that
    $\frac{ab}{p} = \alpha w_1w_2 \cdots w_m$. \\
    From the fact that $\frac{ab}{p}\cdot p = ab$, we have that $\alpha p w_1w_2\cdots w_m = \zeta\mu u_1\cdots u_kv_1 \cdots v_l$.
    By uniqueness of factorization, we have that these expressions are permutations
    upto multiplication by units, so we have that $p \sim u_i$ or $p \sim v_j$. In
    the first case, we have that $p \mid a$ and in the second, we have that $p \mid b$. \\
    Assume now that every irreducible is prime. We will show uniqueness by induction
    on the length of the factorization. \\
    Length = 0: Then, $a$ is a unit, so $a$ is not divisible by any irreducible. \\
    Assume now that the statement holds for a length $k$ factorization.
    Assume now that we have
    $$
    a = \zeta u_1 \cdots u_{k+1} = \mu v_1 \cdots v_l
    $$
    Now $u_{k+1}$ divides $v_1 \cdots v_l$. But since $u_{k+1}$ is prime, $u_{k+1} \mid v_j$
    for some $1 \leq j \leq l$. Reordering the factors, we have that
    $u_{k+1} \mid v_l$. Since $v_l$ is irreducible, this thus implies that if $u_{k+1}x = v_l$,
    $x$ must be a unit, and thus $v_l \sim u_{k+1}$. Furthermore, we can write that
    $\zeta u_1 \cdots u_k = \mu \left(\frac{v_l}{u_{k+1}}\right) v_1 \cdots v_{l-1}$. Applying
    the inductive hypothesis, we have $l -1 = k$ and the the first $k$ terms
    are simply a permutation of each other, while the last is simply a unit scaling
    away from the other. Thus, these are the same factorization.
\end{proof}
\begin{defn}
    A ring is called \textbf{principal}, if every ideal is principal. If,
    in addition, the ring is a domain, it is a \textbf{Prinicipal Ideal Domain}, or
    \textbf{PID}.
\end{defn}
\begin{ex}
    $\ZZ$ is a PID. $\Zn n$ are principal, any field is a PID. Quotients and products
    of principal rings are principals.
\end{ex}
\begin{thm}
    Every PID is a UFD
\end{thm}
\begin{proof}
    Let $R$ be a PID. We first show the existence of factorization. Say that
    $a \in R$, $a \neq 0$, $a \in R^\times$, and has no factorization. Then,
    $a$ is not irreducible, (as otherwise, it itself is a factorization). So,
    we can right that $a = a_0 = a_1b_1$. However, note that one of these must
    not have a factorization, as otherwise the product of their factorizations
    is a factorization of $R$. So, WoLOG assume that $a_1$ does not have a factorization.
    Again, $a_1$ is not irreducible, and we write $a_1 = a_2b_2$. Recursively
    applying this argument, we have that $a_0 = a_1a_2a_3 \dots$ an infinite sequence with
    $a_i \neq 0$, $a_i \notin R^\times$ and $a_i \not\sim a_{i+1}$, and $a_{i+1} \mid a_{i}$
    In terms of the ideals, we have that
    $$
    (a_0) \subsetneq (a_1) \subsetneq \dots
    $$
    Now, we consider $I = \bigcup\limits_{i=0}^{\infty}{(a_i)}$. Since we have a
    nested sequence of ideals, we have that $I$ itself is an ideal. We have that $R$
    is a principal ideal, which means that $I = (c)$ for some $c \in R$. This means
    though that $c \in I$, which means that $c \in (a_k)$ for some $k \in \NN$.
    However, this means that for any $n \geq k$, we have that $(c) \subset (a_k) \subset I = (c)$,
    meaning that each $(a_n) = (c)$. However, we assumed proper inclusion above,
    and this is thus a contradiction. Thus, we have factorization. \\
    Next, we verify uniqueness. We can show this by showing that every irreducible is
    prime. Let $u$ be an irreducible. We want to show it is prime, so assume that $u \mid ab$.
    Assume that $u \nmid a$. We will show that $u \mid b$. Since we have that $u \nmid a$,
    we have that $u \notin (a)$, so we have that $(a) \subsetneq (a, u)$. Since
    $R$ is principal, we have that $(u, a) = (d)$. Since $u \in (d)$ we have that $d \mid u$.
    However, since $u$ is irreducible, $d$ must be a unit, or $d \sim u$. We know because
    $(u) \subsetneq (u,a)$, we have that $d \sim u$ is impossible. Thus, assume that
    $d$ is a unit. This means though that $(u,a)  = R$. This means that we have $1 \in R$,
    so $1 = \alpha u + \beta a$ (this is what it means to be in $(u,a)$). Scaling
    this equality by $b$, we have that $b = b\alpha u + \beta (ab)$. Now, since $u \mid ab$
    and $u \mid u$, $u \mid b$, as desired.
\end{proof}
\begin{defn}
    A \textbf{Euclidean Domain} is a pair $(R, H)$, where $R$ is a domain, and
    $H: R \setminus \{0\} \rightarrow \NN$ is a function such that
    \begin{enumerate}
        \item $H(ab) \geq H(a)$
        \item Given $X,d \in R$ with $d \neq 0$, there are $q, r \in R$ such that
        \begin{enumerate}
            \item $X = qd + r$
            \item either $r = 0$ or $H(r) < H(d)$
        \end{enumerate}
    \end{enumerate}
\end{defn}
\begin{ex}
    If $F$ is a field, then $(F[T], deg)$ is a Euclidean domain.
\end{ex}
\begin{prop}
    Every Euclidean domain is a PID.
\end{prop}
\begin{proof}
    Let $I \subset R$ be an ideal. WoLOG, $I \neq \{0\}$. Choose $d \in I$ such
    that $H(d)$ is minimal. To show that $I = (d)$, let $a \in I$ and take $q,r \in R$
    such that $a = qd + r$. Then, $r = a - qd \in I$. If $r \neq 0$, we have that $H(r) < H(d)$,
    contradicting the minimality of $d$. Thus $r = 0$, and $a = qd$. Thus,
    every ideal is of the form $(d)$, showing it is principal.
\end{proof}
\begin{cor}
    If $F$ is a field, $F[T]$ is a PID, and hence a UFD.
\end{cor}
\begin{defn}
    Let $R$ be a UFD. We call $0 \neq P \in R[T]$ \textbf{primitive} if $a \in R$
    with $a \mid P$ means that $a \in R^\times$.
\end{defn}
\begin{rmk}
    This is equivalent to no irreducible $p \in R$ divides all the coefficients of
    $P$. Intuitively, we can say the ``gcd'' of the coefficients of $P$ is 1.
\end{rmk}
\begin{lemma}
    \textbf{(Gauss)} If $P,Q$ are primitive, so is $PQ$.
\end{lemma}
\begin{proof}
    Let $P \neq 0$, $Q \neq 0$ such that $PQ$ is not primitive. Let $p \in R$ prime
    such that $p \mid PQ$. Thus, $PQ$ becomes $0$ under the
    $R[T] \rightarrow (R/(p))[T]$. \\
    But $R / (p)$ is a domain by \ref{thm:ideals}, and thus, by \ref{polyringdomain}, we have
    that $(R/(p))[T]$ is also a domain. Thus, we have that either $P$ or $Q$ is
    0 in $(R / (p))[T]$, meaning it is also divisible $p$, as desired.
\end{proof}
