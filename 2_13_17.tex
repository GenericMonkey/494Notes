\section{February 13, 2017}

\begin{prop} \label{Prop 1, Feb 13}
	Let $L/K$ be a finite extension. Then the following are equivalent:
	\begin{itemize}
		\item every $a \in L$ is separable over $K$;
		\item $L = K(a_1, ..., a_n)$ for $a_i \in L$ separable over $K$;
		\item if $M/L$ is an extension s.t. $M/K$ is normal, then $|\Mor_K(L,M)| = [L : K]$.
	\end{itemize}
\end{prop}

\begin{proof}
	$1 \Rightarrow 2$) This is trivial. \\
	$2 \Rightarrow 3$) Apply the same argument as in Prop. 3, Feb 10 \ref{Prop 3, Feb 10}, replacing ``$\leq$'' with ``$=$''. We can do this since $|\{x \in M : M_{a/K}(x) = 0\}| = \deg(M_{a/K})$, where the desired roots exist since $M/K$ is normal and they are distinct since $a$ is separable.\\
	$3 \Rightarrow 1$) We prove the contrapositive: if $a \in L$ is not separable over $K$, then $|\Mor_K(L/M)| < [K(a) : K]$, then induct.
\end{proof}

\begin{defn} \label{Defn 2, Feb 13}
	A finite extension is called \textbf{separable} provided that it satisfies these conditions.
\end{defn}

\begin{cor} \label{Cor 3, Feb 13}
	Let $L/K$ be a finite extension. The set of all $a \in L$ separable over $K$ is a subfield of $L$.
\end{cor}

\begin{proof}
	Let $S \subset L$ be the above set, and $M := K(S) \subset L$ be the subfield generated by $S$. Now $[M : K] \leq [L : K] < \infty$ so there exist $a_1, ..., a_n \in S$ such that $M = K(a_1, ..., a_n)$. Then by Proposition 1 \ref{Prop 1, Feb 13} $M$ is separable, so $M \subset S$. Thus $M = S$.
\end{proof}

\begin{defn} \label{Defn 4, Feb 13}
	$M$ is called the \textbf{separable closure of $K$ in $L$}. $[M : K]$ is the \textbf{separability degree}, notated $[L : K]_{\text{sep}}$; $[L : M]$ is the \textbf{inseparability degree}, notated $[L : K]_{\text{insep}}$.
\end{defn}

\begin{prop} \label{Prop 5, Feb 13}
	(Primitive Element Theorem) Let $L/K$ be a finite separable extension. Then there exists $a \in L$ s.t. $L = K(a)$.
\end{prop}

\begin{proof}
	\begin{lemma}
		Let $F$ be an infinite field, $V$ be a finite-dimensional $F$-vector space, $W_1, ..., W_n \subset V$ proper subspaces. Then $V \neq \bigcup W_i$.
	\end{lemma}
	\begin{proof}[of Lemma]
		This was proved on MATH 494 Problem Set 6.
	\end{proof}
	If $K$ is finite, so is $L$, so $L^{\times}$ is cyclic (\ref{Prop 4, Feb 3}). Then any generator $a \in L^{\times}$ works. If $K$ if infinite, let $\overline{L}$ be an algebraic closure of $L$. Then let $\varphi_1, ..., \varphi_n : L \rightarrow \overline{L}$ be the elements of $\Mor_K(L, \overline{L})$, so $n = [L : K]$. For any $1 \leq i \neq j \leq n$ $\varphi_1 - \varphi_j$ is a nonzero morphism of $K$-vector spaces. Let $W_{i, j} \subsetneq L$ be its kernel. By Lemma 1.6, $L \neq \bigcup W_{i, j}$, so let $a \in L \setminus \left( \bigcup W_{i, j} \right)$, so $\varphi_i(a) \neq \varphi_j(a)$ for all $1 \leq i \neq j \leq n$. Then $\varphi_1|_{K(a)}, ..., \varphi_n|_{K(a)}$ are distinct elements of $\Mor_K(K(a), \overline{L})$, so $[L : K] = n \leq [K(a) : K] \leq [L : K]$ since $K(a) \subset L$ (see \ref{Prop 3, Feb 10}). Then $K(a) = L$.
\end{proof}

\begin{rmk}
	Let $L/K$ be a finite extension. Then every element of $\Mor_K(L, L)$ is invertible, and we can compose them. So $\Mor_K(L,L)$ is a group.
\end{rmk}

\begin{defn} \label{Defn 6, Feb 13}
	$\Mor_K(L,L) =: \Aut(L/K) =: \Aut_K(L)$.
\end{defn}

\begin{defn} \label{Defn 7, Feb 13}
	A finite extension $L/K$ is called \textbf{Galois} provided that $|\Aut(L/K)| = [L : K]$. In that case we write $\Gal(L/K) := \Aut(L/K)$ is its \textbf{Galois group.}
\end{defn}
