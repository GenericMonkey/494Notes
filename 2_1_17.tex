\section{February 1, 2017}
\begin{prop} \label{Prop 1, Feb 1}
    Let $M / L / K$ be a tower of extensions.
    \begin{enumerate}[1)]
        \item If $M / K$ is algebraic, then $M / L$ is algebraic.
        \item If $M / L$ and $L / K$ are algebraic, so is $M/K$.
    \end{enumerate}
\end{prop}

\begin{proof}
    Of 1). Let $a \in M$. There exist $P \neq 0 \in K[T]$ such that $P(a) = 0$. But
    $P$ can be viewed as an element of $L[T]$. \\
    Of 2). Let $a \in M$. Since $a$ is algebraic over $L$, there is $P \neq 0 \in L[T]$
    such that $P(a) = 0$. $P(T) = a_nT^n + \dots a_0, a_i \in L$. Each $a_i$ is
    algebraic over $K$, thus $[K(a_1, \dots a_n) : K] < \infty$ by \ref{Prop 2, Jan 30}. \\
    (This is implicitly inducting on that proposition, which said extending by a single
    element is still finite). \\
    Now, $P \in K(a_1 \dots a_0[T]$, so $a$ is algebraic over the field $K(a_1, \dots a_n)$.
    Thus, we have that (again by \ref{Prop 2, Jan 30}) that $[K(a_1, \dots a_n, a) : K(a_1, \dots a_n)] < \infty$
    By \ref{Cor 8, Jan 27}, $[K(a_1, \dots a_n, a) : K] = [K(a_1, \dots a_n, a) : K(a_1, \dots a_n)] \cdot
    [K(a_1, \dots a_n) : K] < \infty$. Thus, by \ref{Prop 2, Jan 30}, $M$ is algebraic.
\end{proof}

\begin{prop}\label{Prop 2, Feb 1}
    Let $P \neq 0 \in K[T]$ There exists a finite extension $L / K$ and $a \in L$
    such that $P(a) = 0$.
\end{prop}
\begin{proof}
    Let $Q$ be an irreducible factor of $P$. Let $L = K[T] / (Q)$, $a =$ image of $T$.
    Then $P(a) = 0$, as by coset operations this is exactly $P + (Q) = (Q) = 0$ in this
    quotient space. Since $Q$ is irreducible, it is prime (as $K[T]$ is a UFD). So,
    $L$ is a domain by \ref{Thm 8, Jan 6}. By \ref{Prop 4, Jan 27}, the set $1, T, \dots T^{n-1}$ is
    a basis for the $K$-vector space.
\end{proof}

\begin{rmk}
    Same is true for $P_1, \dots P_n \in K[T]$ by inducting on $n$.
\end{rmk}

\begin{cor} \label{Cor 3, Feb 1}
    Let $P \in K[T]$. There exists a finite extension $L / K$ $a_1 \dots a_k \in L$
    distinct, $m_1 \dots m_k \in \NN$ nonzero $c \in K$ such that
    $$
    P(T) = c(T - a_1)^{m_1} \cdots (T - a_k)^{m_k}
    $$
\end{cor}
\begin{proof}
    Induct on above proposition (\ref{Prop 2, Feb 1}) and \ref{Lemma 3, Jan 13}.
\end{proof}

\begin{thm} \label{Thm 4, Feb 1}
    Let $K$ be a field. There exists an algebraically closed extension $L / K$.
\end{thm}

\begin{lemma} \label{Lemma 5, Feb 1}
    Let $K$ be a field. There exists an extension $L / K$ such that for every
    $P \in K[T] \setminus K$ there is $a \in L$ with $P(a) = 0$.
\end{lemma}
\begin{proof}
    Let $S \in K[T] \setminus K$. Let $R = K[T_p \mid p \in S]$ (We construct polynomials
    with variables indexed by the set $S$. Thus, each polynomial gets an assoicated variable in this
    larger polynomial ring). Let $I$ be the ideal generated by $\{P(T_P) \mid p \in S \}$. For example,
    if $P = T^2 + 1$, then we consider $T_{T^2 + 1}^2 + 1$ as one generator of this ideal). Now,
    we do this for every polynomial in $K[T] \setminus K$). \\
    \textbf{Claim:} $I \neq R$. 
    \begin{proof}
        Assume that $I = R$. Then there are $r_1, \dots r_n \in R, P_1, \dots P_n \in S$
        such that
        $$
        1 = \sum{r_iP_i(T_{P_i})}
        $$
        By \ref{Prop 2, Feb 1}, there is a finite extension $L^\prime / K$ and $a_{P_1}, \dots
        a_{P_n} \in L^\prime$ such that $P_i(a_{P_i}) = 0$. For $P \in S$ not among the $P_1, \dots P_n$, choose
        $a_P \in L^\prime$ arbitrarily. We get a map $f:S \rightarrow L^\prime$ that sends
        $P \rightarrow a_P$. \\
        There exists a unique morphism $\phi: R \rightarrow L^\prime$ such that $\phi(T_P) = f(p)$ (This is
        from homework on the construction of the polynomials of too many variables). But,
        \begin{align*}
            \varphi(1) &= \varphi\left(\sum{r_iP_i(T_{P_i})}\right) \\
            &= \sum{r_i\varphi(P_i(T_{P_i}))} \\
            &= \sum{r_iP_i(\varphi(T_{P_i}))} \\
            &= \sum{r_iP_i(a_{P_i})} \\
            &= \sum{r_i0} \\
            &= 0 \neq 1
        \end{align*}
    \end{proof}
    Now, by \ref{Thm 2, Jan 9}, there exists a max ideal $I \subset \mathcal{M} \subset R$.
    Put $L = R / \mathcal{M}$ By \ref{Thm 8, Jan 6}, $L$ is a field. For every $P \in S$,
    let $a_P \in L$ be the image of $T_P$. Then $P(a_{P}) = 0$.
\end{proof}

\begin{proof}
    \textit{of Theorem \ref{Thm 4, Feb 1}}. We apply the above Lemma (\ref{Lemma 5, Feb 1}) inductively
    to obtain
    $$
    K = K_0 \subset K_1 \subset \dots
    $$
    Put $L = \bigcup_{n \geq 0}{K_n}$. Then $L$ is a field. To see that $L$ is algebraically
    closed, let $P \in L[T] \setminus L$. Write $P(T) = a_nT^n \dots a_0$.
    There exists $N$ such that $a_n \dots a_0 \in K_N$. Thus, $P \in K_N[T]$. By
    construction, we have that there is $a_P \in K_{N+1}$ such that $P(a_P) = 0$.
\end{proof}
