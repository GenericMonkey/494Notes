\section{February 15, 2017} %-Pranav
\begin{ex} \hspace{0.5cm}
    \begin{itemize}
        \item $\CC / \RR$ is Galois, $\gal(\CC /\RR) = \{1, \tau\}$, where
        $\tau(z) = \bar{z}$.
        \item $E /F$ a quadratic extension, char$(F) \neq 2$, is Galois.
        $\gal(E / F) = \{1, \tau\}, \tau(a) = -a$. \\
        This comes from the fact that in Homework, we showed that if $[E :F]$ is
        2, we have the existence of $\alpha \in E \setminus F$ such that $\alpha^2 \in F$.
        We have that $T^2 - a^2 \in F[T]$ is the minimal polynomial, and we know
        the roots of the minimal polynomial are in bijection with the morphisms
        from $E$ to $E$ in this case. Thus, the automorphisms are solely determine
        by how they act on $a$ and $-a$, hence the above 2 morphisms.
        \item $\FF_{q^m} / \FF_q$ is Galois with $\gal(\FF_{q^m} / \FF_q) \cong \Zn{m}$
        and is generated by $\frob_q(x) = x^q$.
        \item (Non-Example): $L = \FF_p(T), K = \FF_p(T^p) \subset L$. We note that
        $L = K(T)$ (adjoining $T$). For $P(x) = X^p - T^p \in K[X]$, we have that $T$ is
        a root. Thus, the minimal polynomial of $T$ in $K$ divides $X^p - T^p$, so we have
        $[L:K] \mid p$. As $p$ is prime, we have this index is $1$ or $p$, but as $L \neq K$,
        it cannot be 1. Thus, $[L:K] = p$. Moreover, this implies $L$ is the
        splitting field of $P$, because $L[T], P(X) = (X - T)^p$. \\
        Now, we characterize automorphims. Take $\varphi \in \aut(L / K)$. Then,
        we have $\phi(T)^p = \phi(T^p)$. Because $T^p \in K$ and $\varphi$ fixes $K$,
        we have that this equals $T^p$. Thus, $0 = \varphi(T)^p - T^p = (\varphi(T) - T)^p$.
        Thus, $\varphi(T) = T$, meaning, as every automorphism fixes both the base
        field and the element we adjoin, we have $\aut(L/K) = \{1\}$.
    \end{itemize}
\end{ex}
\noindent Recall, for $G$ a group, $X$ a set. \\
An \textbf{action} of $G$ on $X$ is a group homomorphism from $G \rightarrow \textrm{Bij}(X)$.
The action is called
\begin{itemize}
    \item \textbf{transitive} if $\forall \, x,y \in X, \exists \, g \in G$ such that $gx = y$.
    \item \textbf{faithful} if $\forall \, x \in X, gx = x \Rightarrow g = 1$.
    \item \textbf{simple} if $\exists \, x \in X, gx = x \Rightarrow g = 1$.
\end{itemize}
Some notation:
\begin{itemize}
    \item $G \cdot x = \{g \cdot x \mid g \in G \} \subset X$ is the \textbf{orbit} of $x$.
    \item $G_x = \{g \in G \mid gx = x \} \subset G$ is the \textbf{stabilizer} of $x$.
    \item $X^G = \{x \in X \mid gx = x \forall g \in G \}$
\end{itemize}
\noindent $G / G_x \cong G \cdot x$ by the map $g \mapsto g \cdot x$ (orbit-stabilizer) \\
If $y = gx$, then $Gg = gG_xg^{-1}$. \\
If this action is transitive and all stabilizers have the same size, the action is \textbf{simply transitive}
and $|G| = |X|$.
\setcounter{thm}{-1}
\begin{defn} \label{Def 0, Feb 15}
    If $L$ is a field, $\aut(L)$ is the group of field automorphisms of $L$.
\end{defn}
\begin{rmk}
    If $L / K$ is an extension, the $\aut(L/K) = \{\sigma \in \aut(L) \mid \sigma(k) = k \forall k \in K\}$.
    That is, the map restricts to the identity on $K$.
\end{rmk}
If $G \subset \aut(L)$, then $L^G$ is a subfield of $L$. Each $\sigma \in \aut(L)$
acts on $L[T]$ by $\sigma(a_nT^n + \dots a_0) = \sigma(a_n)T^n + \dots \sigma(a_0)$.
By \ref{Fact 1, Jan 13} we have $L \overset{\sigma}{\rightarrow} L \hookrightarrow L[T]$ that sends
$T \mapsto T$. If we have $K = L^G$, then $L[T]^G = K[T]$.
\begin{prop} \label{Prop 1, Feb 15}
    Let $L / K$ be a finite extension. The following are equivalent.
    \begin{enumerate}
        \item $L / K$ is Galois
        \item $L / K$ is normal and separable
        \item $K = L^{\aut(L/K)}$
        \item For every $a \in L, M_{a/K}(T) = \prod\limits_{\eta \in G \cdot a}{(T - \eta)}$ where $G = \aut(L/K)$
    \end{enumerate}
\end{prop}
\begin{proof}
    $2) \Rightarrow 1)$ Immediate from \ref{Prop 1, Feb 13}. \\
    $1) \Rightarrow 3)$ $|G| = [L:K] \geq [L : L^G] \geq |G|$ as $G \subset \mor_{L^G}(L, L)$
    and by \ref{Prop 3, Feb 10}. This string of inequalities thus implies that
    $[L:K] = [L: L^G]$ which implies that $K = L^G$. \\
    $3) \Rightarrow 4)$ Let $Q(T) = \prod\limits_{\eta \in G \cdot a}{(T - \eta)} \in L[T]$. Now
    we note that $Q(a) = 0$. Now, $\forall \sigma \in G$, we have that $\sigma(Q)$ simply permutes
    the ordering on the orbit, and thus $\sigma(Q) = Q$. This means that $Q \in K[T]$, as $K[T]$ is by
    assumption the field fixed by $\aut(L/K)$. Now, we also note that every $b \in G \cdot a$ is a zero of
    $M_{a/K}$, because $b = \sigma(a)$ for $\sigma \in $, and is thus a zero of $\sigma(M_{a/K})$.
    However, $\sigma$ also fixes $M_{a/K}$, which means $b$ is a zero of $M_{a/K}$. Now,
    this implies thus that $Q \mid M_{a/K}$ in $L[T]$. Since we know that both are
    now in $K[T]$, we also thus have that $Q \mid M_{a/K}$ in $K[T]$. However, by \ref{Prop 1, Jan 30},
    $M_{a/K}$ is irreducible, so we have $M_{a/K} = Q$, as desired. \\
    $4) \Rightarrow 2)$ By 4) $M_{a/K}$ splits completely over $L$ with distinct
    roots, so $L / K$ is normal and separable.
\end{proof}
