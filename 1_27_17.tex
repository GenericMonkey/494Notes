\section{January 27, 2017}

\begin{defn} \label{Defn 1, Jan 27}
\hspace{0.5cm}
\begin{enumerate}
\item[(i)] A \textbf{morphism of fields} is a morphism of rings whose source and target are fields
\item[(ii)] A \textbf{subfield} is a subring of a field that is a field
\item[(iii)] An \textbf{extension} of a field $k$ is a field $L$ containing $k$, denoted $L/k$.
\end{enumerate}
\end{defn}

\begin{rmk}
If $L/k$ is an extension, then $L$ is a $k$-algebra, and in particular a $k$-vector space. An extension $L/k$ is called \textit{finite} if $\dim_k(L) < \infty$. In this case define the \textit{degree} of $L/k$, denoted $[L : k] \stackrel{\text{def}}{=} \dim_k(L)$.\footnote{The same notation used for the indexes of subgroups is used here since field extensions in Galois theory correspond to subgroups of the Galois group of the extension (provided that it is Galois). See below, namely Feb. 13 and on.}
 \end{rmk}

\begin{fact} \label{Fact 2, Jan 27}
 If $k_1, k_2$ are subfields of $L$, $k_1 \cap k_2$ is a subfield.
\end{fact}

\begin{defn} \label{Defn 3, Jan 27}
\hspace{0.5cm}
\begin{enumerate}
\item[(i)] Every field has a smallest field, called the \textbf{prime subfield}, equal to $\bigcap_{k \subset L} k^{\text{field}}$, the intersection of all subfields of $L$. \\
\textit{Example:} The prime subfield of $\RR$ is $\QQ$. To see this, observe: any subfield of $\RR$ must contain 1 and 0, thus sums of the form $1 + \cdots + 1$, so it must contain $\NN$. Throwing in additive and then multiplicative inverses gives $\QQ$.
\item[(ii)] Given an extension $L/k$ and any subset $S \subset L$, then $k(S)$ is the smallest subfield of $L$ containing $k$ and $S$.
\end{enumerate}
\end{defn}

\begin{prop}\label{Prop 4, Jan 27}
Let $L$ be a ring and $k \subset L$ a subfield. If $L$ is a domain and $\dim_k(L) < \infty$ then $L$ is a field.
\end{prop}
\begin{proof}
Let $0 \neq a \in L$. The map $a \cdot \colon L \rightarrow L$ that sends $x$ to $ax$ is $k$-linear, and since $L$ is a domain, injective: if $ax = 0$ then since $a \neq 0$, $x = 0$. By rank-nullity, the dimension of the image of $a \cdot$ must be equal to that of the target, i.e. $a \cdot$ is surjective. Thus there exists $G \in L$ so that $aG = 1$, implying that, as desired, $L$ is a field.
\end{proof}

\begin{defn} \label{Defn 5, Jan 27}
Let $L/k$ be an extension, $L \supset E, F \supset k$ intermediate fields, each finite over $k$. The \textit{composite of E and F} is
\[E \cdot F = \left\{ \sum_{i = 1}^n e_i \cdot f_i \colon n \in \NN, e_i \in E, f_i \in F \right\}.\]
\end{defn}

\begin{prop} \label{Prop 6, Jan 27}
$EF$ is a field extension of $k$. It is finite and $[EF : k] \leq [E : k] \cdot [F : k]$.
\end{prop}
\begin{proof}
It is clear that $EF$ is a subring of $L$ containing $k$. Let $\{e_1, ..., e_n\}$ and $\{f_1, ..., f_m\}$ be, respectively, bases for $E/k$ and $F/k$. Then the set $\{e_i \cdot f_j \colon i \in \{1, ..., n\}, j \in \{1, ..., m\}\}$ spans the $k$-vector space $EF$. Thus by elementary results, $\dim_k(EF) \leq \#\{e_i \cdot f_j\} = [E : k] \cdot [F : k]$. In particular, $[EF : k] < \infty$. Applying \ref{Prop 4, Jan 27} above, we have that $EF$ is a field.
\end{proof}

\begin{prop} \label{Prop 7, Jan 27}
Let $L/k$ be a finite extension, $V$ a finite dimensional $L$-vector space. Then $V$ is finite dimensional as a $k$-vector space, with $\dim_k(V) = \dim_L(V) \cdot [L : k]$.
\end{prop}
\begin{proof}
Let $\{v_1, ..., v_n\}$ be a basis for $V$ over $L$, $\{\ell_1, ..., \ell_m\}$ be a basis for $L$ over $k$. We will show $\{\ell_i \cdot v_j \colon i \in \{1, ..., n\}, j \in \{1, ..., j\}\}$ is a basis for $V$ over $k$:
\begin{enumerate}
\item[(i)] Spanning: given $x \in V$ we may write $x \in \lambda_1v_1 + \cdots + \lambda_nv_n$ for $\lambda_i \in L$, with each $\lambda_i = \mu_{i,1}\ell_1 + \cdots + \mu_{i,m}\ell_m$. Then $x = \sum_i \sum_j \mu_{i, j} \ell_j v_i$ with the $\mu_{i,j} \in k$.
\item[(ii)] Linear independence: Write $\sum \mu_{i,j} \ell_j v_i = 0$. Then $0 = \sum_i \left( \sum_j\mu_{i,j} \ell_j \right) v_i$. By linear independence of the $v_i$ we have $\sum_j\mu_{i,j} \ell_j = 0$, by linear independence of the $\ell_j$ we have $\mu_{i,j} = 0$ for all $i,j$.
\end{enumerate}
\end{proof}

\begin{cor}\label{Cor 8, Jan 27}
If $M/L/k$ is a tower of finite extensions, then $[M : k] = [M : L] \cdot [L : k]$.
\end{cor}

\begin{defn} \label{Defn 9, Jan 27}
Let $L/k$ be an extension, and $a \in L$. If there is $0 \neq p \in k[T]$ such that $p(a) = 0$, then $a$ is \textbf{algebraic over \textit{k}}. Otherwise, $a$ is \textbf{transcendental over \textit{k}}.
\end{defn}

As per the above definitions, we would like to study the following construction: let $\text{ev} \colon k[T] \rightarrow L$, $p \mapsto p(a)$ be the evaluation morphism. If $a$ is transcendental, ev is injective. Further, since $L$ is a field, we have that ev factors uniquely through the fraction field $k(T)$\footnote{This notation was established on homework: $k(T)$ is the set of rational functions with coefficients in $k$.} and induces an isomorphism $k(T) \stackrel{\sim}{\rightarrow} k(a)$. If $a$ is algebraic, let $0 \neq I \subset k[T]$ be the kernel of ev. Since $k[T]$ is a Euclidean domain, hence a PID (see a proposition from Jan. 20), so there exists a unique monic polynomial $p \in k[T]$ such that $(p) = I$.

\begin{defn} \label{Defn 10, Jan 27}
$p$ is called the \textbf{minimal polynomial} of $a$ over $k$, $a/k$, written $M_{a/k}$. $\deg(a/k) := \deg(M_{a/k})$ is its \textbf{degree}.
\end{defn}

\begin{rmk}
$M_{a/k}$ is also the minimal polynomial of the endomorphism (linear operator that is not necessarily an isomorphism) $a \cdot \colon L \rightarrow L$.
\end{rmk}
